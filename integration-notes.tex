\documentclass{article}
\usepackage[utf8]{inputenc}

\usepackage{amsmath}
\usepackage{amssymb} 
\usepackage{amsthm}  
\usepackage{dsfont}
\usepackage{mathrsfs}
\usepackage{mathtools}

\usepackage{geometry}

\usepackage{hyperref}        

\usepackage[french]{babel}

\usepackage[shortlabels]{enumitem}

\newcommand{\indep}{\perp\!\!\! \perp}

\theoremstyle{definition} 
\newtheorem{definition}{Définition}

\theoremstyle{definition} 
\newtheorem{prop}{Proposition}

\theoremstyle{definition}
\newtheorem{coro}{Corollaire}

\theoremstyle{definition}
\newtheorem{example}{Exemple}

\theoremstyle{plain}
\newtheorem{theorem}{Théorème}

\theoremstyle{definition}
\newtheorem{remarque}{Remarque}


\begin{document}

\title{Notes Intégartion te Séries de Fourier}
\author{Yago iglesias}
\maketitle
\tableofcontents

\section{Théorie de la Mesure}

\subsection{Tribus}


\begin{definition}[Tribu]
	Soit $E$ un ensemble. Une tribu $\mathscr{A}$ sur $E$ est une partie de $\mathscr{P}(E)$ vérifiant:
	\begin{enumerate}
		\item $E \in \mathscr{A}$
		\item $A \in \mathscr{A} \implies A^c \in \mathscr{A}$
		\item $ \forall n \in \mathbb{N}, A_n \in \mathscr{A} \implies \bigcup\limits_{n \in \mathbb{N}} A_n \in \mathscr{A}$
	\end{enumerate}
\end{definition}


\begin{prop}
	Toute intsersection de tribus sur $E$ est une tribu sur $E$.
\end{prop}

\begin{proof}
	Soit $i \in I$, $\mathscr{A}_i \forall i$ une tribu sur $E$. On a:
	\begin{itemize}
		\item $\forall  i \in I, E \in \mathscr{A}_i \implies E \in \bigcap\limits_{i \in I} \mathscr{A}_i$
		\item $\forall i \in I, A \in \mathscr{A}_i \implies \forall i \in I, \, A^c \in \mathscr{A}_i
			      \implies A^c \in \bigcap\limits_{i \in I} \mathscr{A}_i$
		\item $\forall i \in I, \forall n \in \mathbb{N}, A_n \in \mathscr{A}_i \implies
			      \forall i \in I, \,\bigcup\limits_{n \in \mathbb{N}} A_n \in \mathscr{A}_i \implies
			      \bigcup\limits_{n \in \mathbb{N}} A_n \in \bigcap\limits_{i \in I} \mathscr{A}_i$
	\end{itemize}
\end{proof}

\begin{definition}
	Soit $\mathscr{C}$ un sous ensemble de $\mathscr{P}(E)$. On note la tribu
	engrndrée par $\mathscr{C}$, $\sigma(\mathscr{C})$ avec
	\begin{equation*}
		\sigma(\mathscr{C}) = \bigcap\limits_{\mathscr{A} \in \mathscr{A}, \mathscr{C} \subset \mathscr{A}} \mathscr{A}
	\end{equation*}
\end{definition}

\begin{remarque}
	$\sigma(\mathscr{C})$ est bien une tribu comme intersection non vide de tribus,
	car $\mathscr{P}(E)$ est une tribu sur $E$ qui contient $\mathscr{C}$.
\end{remarque}

\begin{definition}[Tribu borélienne]
	On note $\omega$ l'ensemble des ouverst ed $\mathbb{R}$. La tribu borélienne sur
	$\mathbb{R}$ est la tribu $\sigma(\omega)$, notée $\mathscr{B}(\mathbb{R})$.
\end{definition}


\begin{remarque}
	On peut ettendre cette définition à tout space topologique (ou moins fort, tout space métrique). En particulier
	$\mathbb{R}^d$.
\end{remarque}

\begin{remarque}
	$\mathscr{B}(\mathbb{R})$ est aussi engrendrée par :
	\begin{itemize}
		\item $\{ ]-\infty, a[\,,\, a \in \mathbb{R} \}$
		\item $\{ ]a, b[\,,\, a < b \in \mathbb{R} \}$
		\item $\{ ]a, b[\,,\, a < b \in \mathbb{Q} \}$
	\end{itemize}
\end{remarque}


\begin{definition}[Tribu produit]
	Si $(E_1, \mathscr{A}_1)$ et $(E_2, \mathscr{A}_2)$ sont deux espaces mesurables ( couple ensemble-tribu compatible)
	on note $\mathscr{A}_1 \otimes \mathscr{A}_2$ la tribu sur $E_1 \times E_2$ engendrée
	par les rectangles $A_1 \times A_2$ avec $A_1 \in \mathscr{A}_1$ et $A_2 \in \mathscr{A}_2$.

\end{definition}


\subsection{Mesures}

On se donne $(E, \mathscr{A})$ un espace mesurable.

\begin{definition}[Mesure]
	On dit que $\mu : \mathscr{A} \to \mathbb{R}^+\cup \{+\infty\}$ est une mesure sur $(E, \mathscr{A})$ si:
	\begin{enumerate}
		\item $\mu(\emptyset) = 0$
		\item $\mu$ est $\sigma$-additive, c'est à dire que si $(A_n)_{n \in \mathbb{N}}$ est une suite d'éléments de $\mathscr{A}$
		      tels que $A_i \cap A_j = \emptyset$ pour $i \neq j$ (deux à deux disjoints), alors
		      \begin{equation*}
			      \mu\left(\bigcup\limits_{n \in \mathbb{N}} A_n\right) = \sum\limits_{n \in \mathbb{N}} \mu(A_n)
		      \end{equation*}
	\end{enumerate}
\end{definition}

\begin{remarque}
	On appelle meurables les ensembles qui sont dans $\mathscr{A}$.
\end{remarque}

\begin{remarque}
	Comme $\mu (A_n) \in \mathbb{R}^+ \cup \{+\infty\}$, la somme $\sum\limits_{n \in \mathbb{N}} \mu(A_n)$ est bien définie.
\end{remarque}

\begin{remarque}
	$\mu$ mesurable donne l'additivité (finie). Cependant, la reciproque est fausse.
	\begin{example}
		Soit $m: \mathscr{P}(\mathbb{N}) \to \mathbb{R}^+$
		\begin{equation*}
			m(A) = \left\{
			\begin{array}{ll}
				0       & \text{ si } A \text{ est fini } \\
				+\infty & \text{ sinon }
			\end{array}
			\right.
		\end{equation*}
		est additive mais pas $\sigma$-additive.
	\end{example}
\end{remarque}

\begin{example}
	Sur $(\mathbb{R}, \mathscr{B}(\mathbb{R}))$, on a la mesure de
	Dirac en $x_0 \in \mathbb{R}$, notée $\delta_{x_0}$, définie par:
	\begin{equation*}
		\delta_{x_0}(A) = \left\{
		\begin{array}{ll}
			1 & \text{ si } x_0 \in A \\
			0 & \text{ sinon }
		\end{array}
		\right.
	\end{equation*}
\end{example}

\begin{example}
	Sur $\mathbb{R}, \mathscr{B}(\mathbb{R})$, on a la mesure de comptable, notée $\nu$, définie par:
	\begin{equation*}
		\nu(A) = \left\{
		\begin{array}{ll}
			\#A     & \text{ si } A \text{ est fini } \\
			+\infty & \text{ sinon }
		\end{array}
		\right.
	\end{equation*}
\end{example}

\begin{proof}
	\begin{enumerate}
		\item Le espace de départ est bien une tribu car $\mathscr{P}(\mathbb{R})$ est une tribu sur $\mathbb{R}$.Le
		      espace d'arrivée est bien $\mathbb{R}^+ \cup \{+\infty\}$ car $\#A \in \mathbb{R}^+ \cup \{+\infty\}$.
		\item $\nu(\emptyset) = \#\emptyset = 0$
		\item Si $(A_n)_{n \in \mathbb{N}}$ est une suite de boréliens disjoints:
		      \begin{equation*}
			      \mu(\bigcup\limits_{n \in \mathbb{N}} A_n) = \left\{\begin{array}{ll}
				      \#A     & \text{ si } \bigcup\limits_{n \in \mathbb{N}} A_n \text{ est fini } \\
				      +\infty & \text{ sinon }
			      \end{array}
			      \right.
		      \end{equation*}
		      $\bigcup\limits_{n \in \mathbb{N}} A_n$ est fini si $\exists k ,\, A_k$ infini (cas 1) ou si les elements sont
		      finis mais tous non vides a partir d'un certain rang (cas 2).
		      \begin{enumerate}
			      \item Cas 1 : $\sum\limits_{n \in \mathbb{N}} \nu(A_n) \geq \nu(A_k) = +\infty$.
			      \item Cas 2 : $\sum\limits_{n \in \mathbb{N}} \nu(A_n) = +\infty$ car
			            $\forall n \in \mathbb{N}, \, \nu(A_n) \in \mathbb{N}$ et $\nu(A_n)$ ne stationne pas en 0l Donc il existe une suite
			            infinie d'éléments non vides, donc tels que $\nu(A_n) \geq 1$. Donc la some diverge.
		      \end{enumerate}
		      Par contre si $\bigcup\limits_{n \in \mathbb{N}} A_n$ est fini, alors $A_n = \emptyset$ à partir d'un certain rang.
		      Donc le cardinal de $\bigcup\limits_{n \in \mathbb{N}} A_n$ est fini et $\sum\limits_{n \in \mathbb{N}} \nu(A_n) = \nu(\bigcup\limits_{n \in \mathbb{N}} A_n)$ car ils
		      sont deux a deux disjoints.
	\end{enumerate}
\end{proof}

\begin{prop}[Propriétés élémentaires]\label{prop:mesure:elementaire}
	Nous avons 5 propriétés élémentaires:
	\begin{enumerate}
		\item Croissance. Si $A$ et $B$ mesurables, avec $A \subset B$, alors $\mu(B) = \mu(A) + \mu(B \setminus A)$. De plus,
		      si $\mu(B)$ est finie, alors $\mu(B\setminus A) = \mu(B) - \mu(A)$.
		\item Crible. Si $A$ et $B$ mesurables
		      \[\mu(A\cup B) + \mu(A \cap B) = \mu(A) + \mu(B)\]
		\item Continuité croissante. Soit $A_n$ une suite croissante d'ensembles mesurables $A_n \subset A_{n+1}$,
		      alors
		      \[\mu(\bigcup\limits_{n \in \mathbb{N}} A_n) = \lim\limits_{n \to \infty} \mu(A_n)\]
		\item Continuité décroissante. Soit $A_n$ une suite décroissante d'ensembles mesurables $A_{n+1}
			      \subset A_n$, telle que $\mu(A_0) < +\infty$, alors:
		      \[\mu(\bigcap\limits_{n \in \mathbb{N}} A_n) = \lim\limits_{n \to \infty} \mu(A_n)\]
		\item Sous-aditivité. Soit $(A_n)_{n \in \mathbb{N}}$ une suite d'ensembles mesurables, alors:
		      \[\mu(\bigcup\limits_{n \in \mathbb{N}} A_n) \leq \sum\limits_{n \in \mathbb{N}} \mu(A_n)\]
	\end{enumerate}
\end{prop}

\begin{proof}
	Nous allons démontrer les propriétés dans l'ordre.
	\begin{enumerate}
		\item Soit $B = A \cup (B \setminus A)$, alors $A$ et $B \setminus A$ sont disjoints. Donc
		      \[\mu(B) = \mu(A) + \mu(B \setminus A)\]
		      Donc $\mu(B) \geq \mu(A)$. Donc $\mu(B) < \infty$ alors $\mu(A)< \infty$ et $\mu(B \setminus A) = \mu(B) - \mu(A)$.
		\item \begin{eqnarray*}
			      \mu(A) + \mu (B) &=& \mu( A \setminus (A \cap B)) + \mu(A \cap B) + \mu(A \setminus B) \\
			      \mu(A\cup B)&=& \mu(A \setminus (A \cap B)) + \mu(A \cap B) + \mu(B \setminus (A \cap B))
		      \end{eqnarray*}
		\item On pose $B_0 = A_0$ et $B_n = A_n \setminus A_{n-1}$ pour $n \geq 1$. On a que les $B_n$ sont disjoints et
		      $\bigcup\limits_{n \in \mathbb{N}} B_n = \bigcup\limits_{n \in \mathbb{N}} A_n$. Donc
		      \begin{eqnarray*}
			      $\bigcup\limits_{n \in \mathbb{N}} \mu(A_n) &=& \bigcup\limits_{n \in \mathbb{N}} \mu(B_n)\\
			      &=& \sum\limits_{n \in \mathbb{N}} \mu(B_n) \text{par }\quad \sigma\text{-additivité}
		      \end{eqnarray*}
		      Si l'un des $A_n$ vérifie $\mu(A_n) = +\infty$, alors $\mu(\bigcup\limits_{n \in \mathbb{N}} A_n) = +\infty$
		      et l'égalité est donc vraie.
		      \\
		      Si tous les $A_n$ sont finis, on a :
		      \begin{eqnarray*}
			      \mu(B_n) &=& \mu(A_n) - \mu(A_{n-1}) \\
			      \mu(\bigcup\limits_{n \in \mathbb{N}} A_n) &=& \sum\limits_{n \in \mathbb{N}} \left( \mu(A_n) - \mu(A_{n-1}) \right)+ \mu(A_0) \\
			      &=& \lim\limits_{n \to \infty} \mu(A_0) + \sum\limits_{n = 1}^N \left( \mu(A_n} - \mu(A_{n-1}) \right) \\
			      &=& \lim\limits_{n \to \infty} \mu(A_n)
		      \end{eqnarray*}
		\item On définit $C_n = A_0 \setminus A_n $, alors les $C_n$ sont croissants et mesurables. Donc
		      \begin{eqnarray*}
			      \mu(\bigcup\limits_{n \in \mathbb{N}} C_n) &=& \lim\limits_{n \to \infty} \mu(C_n) \\
			      &=& \lim\limits_{n \to \infty} \mu(A_0\setminus A_n) \\
			      &=& \lim\limits_{n \to \infty} \mu(A_0) - \mu(A_n) \\
			      &=& \mu(A_0) - \lim\limits_{n \to \infty} \mu(A_n)
		      \end{eqnarray*}
		      Or $\bigcup\limits_{n \in \mathbb{N}} C_n = A_0 \setminus \bigcup\limits_{n \in \mathbb{N}} A_n$ et donc
		      $\mu(\bigcup\limits_{n \in \mathbb{N}} C_n) = \mu(A_0) - \mu(\bigcup\limits_{n \in \mathbb{N}} A_n)$.
		\item On pose $D_n = A_n \cap \left( \bigcup\limits_{i=0}^{n-1} A_i \right)^c$. Les $D_n$ sont disjoints et mesurables par
		      $\sigma$-additivité.
		      \begin{eqnarray*}
			      \mu(\bigcup\limits_{n \in \mathbb{N}} A_n) &=& \mu(\bigcup\limits_{n \in \mathbb{N}} D_n) \\
			      &=& \sum\limits_{n \in \mathbb{N}} \mu(D_n) \quad \text{ par } \sigma\text{-additivité}
		      \end{eqnarray*}
		      et $D_n \subset A_n$ donc $\mu(D_n) \leq \mu(A_n) \leq \sum\limits_{n \in \mathbb{N}} \mu(A_n)$.
	\end{enumerate}
\end{proof}

\begin{remarque}
	Dans le (4)\ref{prop:mesure:elementaire}, l'hypothèse $\mu(A_0) < +\infty$ est nécessaire. En effet, soit
	$\nu$ la mesure de comptage. Soit
	\[ A_n = \{ n, n+1, n+2, \dots \} \]
	On a $\bigcup\limits_{n \in \mathbb{N}} A_n = \emptyset$ et $\nu(A_n) = +\infty$ pour tout $n \in \mathbb{N}$.
\end{remarque}

\begin{definition}[Vocabulaire des mesures]
	Soit $\mu$ une mesure sur $(E, \mathscr{A})$.
	\begin{itemize}
		\item $(E,A\mu)$ est un space mesuré si $(E,\mathscr{A})$ est un espace mesurable et $\mu$ est une mesure sur $ (E, \mathscr{A} ) $.
		\item $\mu$ est dite finie si $\mu(E) < +\infty$.
		\item $\mu$ est une probabilité si $\mu(E) = 1$.
		\item $\mu$ est $\sigma$-finie si $E = \bigcup\limits_{n \in \mathbb{N}} A_n$ avec $A_n \in \mathscr{A}$ et $\mu(A_n) < +\infty$.
		\item On dit que $x$ est un atome de $\mu$ si $\mu(\{x\}) > 0$.
		\item $\mu$ est diffuse si elle n'a pas d'atomes.
	\end{itemize}
\end{definition}

\subsection{Fonctions mesurables}

\begin{definition}
	Soient $(E, \mathscr{A})$ et $(F, \mathscr{B})$ deux espaces mesurables. Une fonction $f: E \to F$ est dite mesurable si
	$f^{-1}(B) \in \mathscr{A}$ pour tout $B \in \mathscr{B}$.
	Dans le cas ou $E$ et $F$ sont munis de leur tribus boréliennes (si elles existent), on dit que $f$ est borélienne.
\end{definition}


\begin{prop}
	Si $f: (E, \mathscr{A}) \to (F, \mathscr{B})$ et $g: (F, \mathscr{B}) \to (G, \mathscr{C})$
	sont mesurables, alors $g \circ f : (E, \mathscr{A}) \to (G, \mathscr{C})$ est mesurable.
\end{prop}

\begin{proof}
	Soitn $C \in \mathscr{C}$. On a que $g^{-1}(C) \in \mathscr{B}$ car $g$ est mesurable et
	$f^{-1}(g^{-1}(C)) \in \mathscr{A}$ car $f$ est mesurable.
	Comme $f^{-1}(g^{-1}(C)) = (g \circ f)^{-1}(C)$, on a que $g \circ f$ est mesurable.
\end{proof}


\begin{prop}
	Pour que $f : (E, \mathscr{A}) \to (F, \mathscr{B})$ soit mesurable, il suffit qu'il existe $\mathscr{C} \subset \mathscr{B}$ telle que:
	\begin{enumerate}
		\item $f^{-1}(C) \in \mathscr{A}$ pour tout $C \in \mathscr{C}$.
		\item $\sigma(\mathscr{C}) = \mathscr{B}$
	\end{enumerate}
\end{prop}

\begin{proof}
	Posons $\mathscr{G} = \{ B \in \mathscr{B} \, | \, f^{-1}(B) \in \mathscr{A} \}$. On a que $\mathscr{C} \subset \mathscr{G}$.
	Montonrs qu $\mathscr{G}$ est une tribu sur $F$.
	\begin{enumerate}
		\item $f^{-1}(\emptyset) = \emptyset \in \mathscr{A}$ donc $\emptyset \in \mathscr{G}$.
		\item Soit $B \in \mathscr{G}$, alors $f^{-1}(B^c) = (f^{-1}(B))^c \in \mathscr{A}$ donc $B^c \in \mathscr{G}$.
		\item Soit $(B_n)_{n \in \mathbb{N}}$ une suite d'éléments de $\mathscr{G}$, alors
		      $f^{-1}(\bigcup\limits_{n \in \mathbb{N}} B_n) = \bigcup\limits_{n \in \mathbb{N}} f^{-1}(B_n) \in \mathscr{A}$ donc
		      $\bigcup\limits_{n \in \mathbb{N}} B_n \in \mathscr{G}$.
	\end{enumerate}
	Ainsi $\mathscr{G}$ est une tribu sur $F$ et $\mathscr{C} \subset \mathscr{G}$.
	On a donc
	\[ \sigma(\mathscr{B}) = \sigma(\mathscr{C}) \subset \sigma(\mathscr{G}) = \mathscr{G} \]
	Donc $\mathscr{B} \subset \mathscr{G} \subset \mathscr{B}$, donc $\mathscr{B} = \mathscr{G}$.
	On a donc que $f$ est mesurable.
\end{proof}


\begin{example}[Application]
	Si $F, \mathscr{B}) = (\mathbb{R}, \mathscr{B}(\mathbb{R}))$, on peut prendre $\mathscr{C} = \{ ]-\infty,\   t[ \, | \, t \in \mathbb{R} \}$. On a que
				$\sigma(\mathscr{C}) = \mathscr{B}(\mathbb{R})$. Et donc il suffit d'étudier la mesurabilité de $f^{-1}(]-\infty, t[)$.
\end{example}


\end{document}
