\documentclass{article}
\usepackage[utf8]{inputenc}
\usepackage[T1]{fontenc}

\usepackage{amsmath}
\usepackage{amssymb} 
\usepackage{amsthm}  
\usepackage{dsfont}
\usepackage{mathrsfs}
\usepackage{mathtools}

\usepackage{geometry}

\usepackage{hyperref}        

\usepackage[french]{babel}

\usepackage[shortlabels]{enumitem}

\usepackage{fancyhdr}

\fancypagestyle{toc}{%
\fancyhf{}%
\fancyhead[L]{\rightmark}%
\fancyhead[R]{\thepage}%
}

\pagestyle{toc}

\newcommand{\indep}{\perp\!\!\! \perp}

\newcommand{\Rbar}{\overline{\mathbb{R}}}
\newcommand{\Rbarp}{\overline{\mathbb{R^+}}}

\newcommand{\R}{\mathbb{R}}
\newcommand{\Rn}{\mathbb{R}^n}
\newcommand{\Rp}{\mathbb{R}^p}

\newcommand{\N}{\mathbb{N}}
\newcommand{\Ns}{\mathbb{N}^*}

\newcommand{\triA}{\mathscr{A}}
\newcommand{\triM}{\mathscr{M}}
\newcommand{\triC}{\mathscr{C}}
\newcommand{\bor}{\mathscr{B}}

\theoremstyle{plain}
\newtheorem{theorem}{Théorème}[section]
\newtheorem{coro}[theorem]{Corollaire}
\newtheorem{lemma}{Lemme}[section]
\newtheorem{prop}{Proposition}[section]

\theoremstyle{definition} 
\newtheorem{definition}{Définition}[section]
\newtheorem{example}{Exemple}[subsection]
\newtheorem{exercice}{Exercice}[subsection]

\theoremstyle{plain}
\newtheorem{remarque}{Remarque}[subsection]


\begin{document}
\begin{titlepage}
	\newcommand{\HRule}{\rule{\linewidth}{0.5mm}}
	\center

	\HRule\\[0.4cm]

	\textsc{\Large Intégration et séries de Fourier}\\[0.5cm]
	\textsc{\large Un ensemble compréhensible de notes de cours}\\[0.5cm]

	\HRule\\[1.5cm]

	{\large\textit{Auteur}}\\
	Yago \textsc{Iglesias}


	\vfill\vfill\vfill

	{\large\today}

	\vfill

\end{titlepage}

\tableofcontents

\section{Introduction}

Ce document est un recueil de notes de cours sur l'intégration niveau L3. Il est
basé sur les cours de M.~\textsc{Cyrille Lucas} à Université Paris Cité, cependant toute
erreur ou inexactitude est de ma responsabilité. Tout futur contributeur
peut se retouver dans la section contributeurs du répertoire
\href{https://github.com/Yag000/integration-notes/graphs/contributors}{GitHub}.
\vspace{0.5cm}

Les notes portent sur la théorie de la mesure, l'integration et les séries de Fourier.
\vspace{0.5cm}

Toute erreur ou remarque est la bienvenue.
Sentez vous libres de contribuer à ce document par le biais de \href{https://github.com/Yag000/integration-notes}{GitHub},
où vous pouvez trouver le code source de ce document et une version pdf à jour.
Si vous n'etes pas familiers avec \textit{Git} ou \LaTeX , vous pouvez toujours me contacter
par \href{mailto: yago.iglesias.vazquez@gmail.com}{mail}.


\section{Théorie de la Mesure}


\subsection{Tribus}


\begin{definition}[Tribu]
	Soit $E$ un ensemble. Une tribu $\mathscr{A}$ sur $E$ est une partie de $\mathscr{P}(E)$ vérifiant:
	\begin{enumerate}
		\item $E \in \mathscr{A}$
		\item $A \in \mathscr{A} \implies A^c \in \mathscr{A}$
		\item $ \forall n \in \mathbb{N}, A_n \in \mathscr{A} \implies \bigcup\limits_{n \in \mathbb{N}} A_n \in \mathscr{A}$
	\end{enumerate}
\end{definition}


\begin{prop}
	Toute intsersection de tribus sur $E$ est une tribu sur $E$.
\end{prop}

\begin{proof}
	Soit $i \in I$, $\mathscr{A}_i \forall i$ une tribu sur $E$. On a:
	\begin{itemize}
		\item $\forall  i \in I, E \in \mathscr{A}_i \implies E \in \bigcap\limits_{i \in I} \mathscr{A}_i$
		\item $\forall i \in I, A \in \mathscr{A}_i \implies \forall i \in I, \, A^c \in \mathscr{A}_i
			      \implies A^c \in \bigcap\limits_{i \in I} \mathscr{A}_i$
		\item $\forall i \in I, \forall n \in \mathbb{N}, A_n \in \mathscr{A}_i \implies
			      \forall i \in I, \,\bigcup\limits_{n \in \mathbb{N}} A_n \in \mathscr{A}_i \implies
			      \bigcup\limits_{n \in \mathbb{N}} A_n \in \bigcap\limits_{i \in I} \mathscr{A}_i$
	\end{itemize}
\end{proof}

\begin{definition}
	Soit $\mathscr{C}$ un sous ensemble de $\mathscr{P}(E)$. On note la tribu
	engrndrée par $\mathscr{C}$, $\sigma(\mathscr{C})$ avec
	\begin{equation*}
		\sigma(\mathscr{C}) = \bigcap\limits_{\mathscr{A} \in \mathscr{A}, \mathscr{C} \subset \mathscr{A}} \mathscr{A}
	\end{equation*}
\end{definition}

\begin{remarque}
	$\sigma(\mathscr{C})$ est bien une tribu comme intersection non vide de tribus,
	car $\mathscr{P}(E)$ est une tribu sur $E$ qui contient $\mathscr{C}$.
\end{remarque}

\begin{definition}[Tribu borélienne]
	On note $\omega$ l'ensemble des ouverst ed $\mathbb{R}$. La tribu borélienne sur
	$\mathbb{R}$ est la tribu $\sigma(\omega)$, notée $\mathscr{B}(\mathbb{R})$.
\end{definition}


\begin{remarque}
	On peut ettendre cette définition à tout space topologique (ou moins fort, tout space métrique). En particulier
	$\mathbb{R}^d$.
\end{remarque}

\begin{remarque}
	$\mathscr{B}(\mathbb{R})$ est aussi engrendrée par :
	\begin{itemize}
		\item $\{ ]-\infty, a[\,,\, a \in \mathbb{R} \}$
		\item $\{ ]a, b[\,,\, a < b \in \mathbb{R} \}$
		\item $\{ ]a, b[\,,\, a < b \in \mathbb{Q} \}$
	\end{itemize}
\end{remarque}


\begin{definition}[Tribu produit]
	Si $(E_1, \mathscr{A}_1)$ et $(E_2, \mathscr{A}_2)$ sont deux espaces mesurables ( couple ensemble-tribu compatible)
	on note $\mathscr{A}_1 \otimes \mathscr{A}_2$ la tribu sur $E_1 \times E_2$ engendrée
	par les rectangles $A_1 \times A_2$ avec $A_1 \in \mathscr{A}_1$ et $A_2 \in \mathscr{A}_2$.

\end{definition}




\subsection{Mesures}

On se donne $(E, \mathscr{A})$ un espace mesurable.

\begin{definition}[Mesure]
	On dit que $\mu : \mathscr{A} \to \mathbb{R}^+\cup \{+\infty\}$ est une mesure sur $(E, \mathscr{A})$ si:
	\begin{enumerate}
		\item $\mu(\emptyset) = 0$
		\item $\mu$ est $\sigma$-additive, c'est à dire que si $(A_n)_{n \in \mathbb{N}}$ est une suite d'éléments de $\mathscr{A}$
		      tels que $A_i \cap A_j = \emptyset$ pour $i \neq j$ (deux à deux disjoints), alors
		      \begin{equation*}
			      \mu\left(\bigcup\limits_{n \in \mathbb{N}} A_n\right) = \sum\limits_{n \in \mathbb{N}} \mu(A_n)
		      \end{equation*}
	\end{enumerate}
\end{definition}

\begin{remarque}
	On appelle mesurables les ensembles qui sont dans $\mathscr{A}$.
\end{remarque}

\begin{remarque}
	Comme $\mu (A_n) \in \mathbb{R}^+ \cup \{+\infty\}$, la somme $\sum\limits_{n \in \mathbb{N}} \mu(A_n)$ est bien définie.
\end{remarque}

\begin{remarque}
	$\mu$ mesurable donne l'additivité (finie). Cependant, la réciproque est fausse.
	\begin{example}
		Soit $m: \mathscr{P}(\mathbb{N}) \to \mathbb{R}^+ \cup \{+\infty\}$
		\begin{equation*}
			m(A) = \left\{
			\begin{array}{ll}
				0       & \text{ si } A \text{ est fini } \\
				+\infty & \text{ sinon }
			\end{array}
			\right.
		\end{equation*}
		est additive mais pas $\sigma$-additive.
	\end{example}
\end{remarque}

\begin{example}
	Sur $(\mathbb{R}, \mathscr{B}(\mathbb{R}))$, on a la mesure de
	Dirac en $x_0 \in \mathbb{R}$, notée $\delta_{x_0}$, définie par:
	\begin{equation*}
		\delta_{x_0}(A) = \left\{
		\begin{array}{ll}
			1 & \text{ si } x_0 \in A \\
			0 & \text{ sinon }
		\end{array}
		\right.
	\end{equation*}
\end{example}

\begin{example}
	Sur $(\mathbb{R}, \mathscr{B}(\mathbb{R}))$, on a la mesure de comptage, notée $\nu$, définie par:
	\begin{equation*}
		\nu(A) = \left\{
		\begin{array}{ll}
			\#A     & \text{ si } A \text{ est fini } \\
			+\infty & \text{ sinon }
		\end{array}
		\right.
	\end{equation*}
\end{example}

\begin{proof}
	\begin{enumerate}
		\item L'espace de départ est bien une tribu car $\mathscr{P}(\mathbb{R})$ est une tribu sur $\mathbb{R}$.
		      L'espace d'arrivée est bien $\mathbb{R}^+ \cup \{+\infty\}$ car $\#A \in \mathbb{R}^+ \cup \{+\infty\}$.
		\item $\nu(\emptyset) = \#\emptyset = 0$
		\item Si $(A_n)_{n \in \mathbb{N}}$ est une suite de boréliens deux à deux disjoints:
		      \begin{equation*}
			      \mu(\bigcup\limits_{n \in \mathbb{N}} A_n) = \left\{\begin{array}{ll}
				      \#A     & \text{ si } \bigcup\limits_{n \in \mathbb{N}} A_n \text{ est fini } \\
				      +\infty & \text{ sinon }
			      \end{array}
			      \right.
		      \end{equation*}
		      $\bigcup\limits_{n \in \mathbb{N}} A_n$ est fini si $\exists k ,\, A_k$ infini (cas 1) ou si les éléments sont
		      finis mais tous non vides à partir d'un certain rang (cas 2).
		      \begin{enumerate}
			      \item Cas 1 : $\sum\limits_{n \in \mathbb{N}} \nu(A_n) \geq \nu(A_k) = +\infty$.
			      \item Cas 2 : $\sum\limits_{n \in \mathbb{N}} \nu(A_n) = +\infty$ car
			            $\forall n \in \mathbb{N}, \, \nu(A_n) \in \mathbb{N}$ et $\nu(A_n)$ ne stationne pas en 0. Donc il existe une suite
			            infinie d'éléments non vides, donc tels que $\nu(A_n) \geq 1$. Donc la somme diverge.
		      \end{enumerate}
		      Par contre si $\bigcup\limits_{n \in \mathbb{N}} A_n$ est fini, alors $A_n = \emptyset$ à partir d'un certain rang.
		      Donc le cardinal de $\bigcup\limits_{n \in \mathbb{N}} A_n$ est fini et $\sum\limits_{n \in \mathbb{N}} \nu(A_n) = \nu(\bigcup\limits_{n \in \mathbb{N}} A_n)$ car ils
		      sont deux a deux disjoints.
	\end{enumerate}
\end{proof}

\begin{prop}[Propriétés élémentaires]\label{prop:mesure:elementaire}
	Nous avons 5 propriétés élémentaires:
	\begin{enumerate}
		\item Croissance. Si $A$ et $B$ mesurables, avec $A \subset B$, alors $\mu(B) = \mu(A) + \mu(B \setminus A)$ et $\mu(A) \leq \mu(B)$. De plus,
		      si $\mu(B)$ est finie, alors $\mu(B\setminus A) = \mu(B) - \mu(A)$.
		\item Crible. Si $A$ et $B$ mesurables
		      \[\mu(A\cup B) + \mu(A \cap B) = \mu(A) + \mu(B)\]
		\item Continuité croissante. Soit $A_n$ une suite croissante d'ensembles mesurables $A_n \subset A_{n+1}$,
		      alors
		      \[\mu(\bigcup\limits_{n \in \mathbb{N}} A_n) = \lim\limits_{n \to \infty} \mu(A_n)\]
		\item Continuité décroissante. Soit $A_n$ une suite décroissante d'ensembles mesurables $A_{n+1}
			      \subset A_n$, telle que $\mu(A_0) < +\infty$, alors:
		      \[\mu(\bigcap\limits_{n \in \mathbb{N}} A_n) = \lim\limits_{n \to \infty} \mu(A_n)\]
		\item Sous-additivité. Soit $(A_n)_{n \in \mathbb{N}}$ une suite d'ensembles mesurables, alors:
		      \[\mu(\bigcup\limits_{n \in \mathbb{N}} A_n) \leq \sum\limits_{n \in \mathbb{N}} \mu(A_n)\]
	\end{enumerate}
\end{prop}

\begin{proof}
	Nous allons démontrer les propriétés dans l'ordre.
	\begin{enumerate}
		\item Soit $B = A \cup (B \setminus A)$, alors $A$ et $B \setminus A$ sont disjoints. Donc
		      \[\mu(B) = \mu(A) + \mu(B \setminus A)\]
		      Donc $\mu(B) \geq \mu(A)$. Donc $\mu(B) < \infty$ alors $\mu(A)< \infty$ et $\mu(B \setminus A) = \mu(B) - \mu(A)$.
		\item \begin{eqnarray*}
			      \mu(A) + \mu (B) &=& \mu( A \setminus (A \cap B)) + \mu(A \cap B) + \mu(B \setminus (A \cap B)) + \mu(A \cap B) \\
			      \mu(A\cup B)&=& \mu(A \setminus (A \cap B)) + \mu(A \cap B) + \mu(B \setminus (A \cap B))
		      \end{eqnarray*}
		\item On pose $B_0 = A_0$ et $B_n = A_n \setminus A_{n-1}$ pour $n \geq 1$. On a que les $B_n$ sont disjoints et
		      $\bigcup\limits_{n \in \mathbb{N}} B_n = \bigcup\limits_{n \in \mathbb{N}} A_n$. Donc
		      \begin{eqnarray*}
			      \bigcup\limits_{n \in \mathbb{N}} \mu(A_n) &=& \bigcup\limits_{n \in \mathbb{N}} \mu(B_n) \\
			      &=& \sum\limits_{n \in \mathbb{N}} \mu(B_n) \quad \text{par } \sigma\text{-additivité}
		      \end{eqnarray*}
		      Si l'un des $A_n$ vérifie $\mu(A_n) = +\infty$, alors $\mu(\bigcup\limits_{n \in \mathbb{N}} A_n) = +\infty$
		      et l'égalité est donc vraie.
		      \\
		      Si tous les $A_n$ sont finis, on a :
		      \begin{eqnarray*}
			      \mu(B_n) &=& \mu(A_n) - \mu(A_{n-1}) \\
			      \mu(\bigcup\limits_{n \in \mathbb{N}} A_n) &=& \sum\limits_{n \in \mathbb{N}} \left( \mu(A_n) - \mu(A_{n-1}) \right)+ \mu(A_0) \\
			      &=& \lim\limits_{N \to \infty} \mu(A_0) + \sum\limits_{n = 1}^N \left( \mu(A_n) - \mu(A_{n-1}) \right) \\
			      &=& \lim\limits_{N \to \infty} \mu(A_N)
		      \end{eqnarray*}
		\item On définit $C_n = A_0 \setminus A_n $, alors les $C_n$ sont croissants et mesurables. Donc
		      \begin{eqnarray*}
			      \mu(\bigcup\limits_{n \in \mathbb{N}} C_n) &=& \lim\limits_{n \to \infty} \mu(C_n) \\
			      &=& \lim\limits_{n \to \infty} \mu(A_0\setminus A_n) \\
			      &=& \lim\limits_{n \to \infty} \mu(A_0) - \mu(A_n) \\
			      &=& \mu(A_0) - \lim\limits_{n \to \infty} \mu(A_n)
		      \end{eqnarray*}
		      Or $\bigcup\limits_{n \in \mathbb{N}} C_n = A_0 \setminus \bigcap\limits_{n \in \mathbb{N}} A_n$ et donc
		      $\mu(\bigcup\limits_{n \in \mathbb{N}} C_n) = \mu(A_0) - \mu(\bigcap\limits_{n \in \mathbb{N}} A_n)$.
		\item On pose $D_n = A_n \cap \left( \bigcup\limits_{i=0}^{n-1} A_i \right)^c$. Les $D_n$ sont disjoints et mesurables par
		      $\sigma$-additivité.
		      \begin{eqnarray*}
			      \mu(\bigcup\limits_{n \in \mathbb{N}} A_n) &=& \mu(\bigcup\limits_{n \in \mathbb{N}} D_n) \\
			      &=& \sum\limits_{n \in \mathbb{N}} \mu(D_n) \quad \text{ par } \sigma\text{-additivité}
		      \end{eqnarray*}
		      et $D_n \subset A_n$ donc $\mu(D_n) \leq \mu(A_n) \leq \sum\limits_{n \in \mathbb{N}} \mu(A_n)$.
	\end{enumerate}
\end{proof}

\begin{remarque}
	Dans le (4) de la proposition \ref{prop:mesure:elementaire}, l'hypothèse $\mu(A_0) < +\infty$ est nécessaire. En effet, soit
	$\nu$ la mesure de comptage. Soit
	\[ A_n = \{ n, n+1, n+2, \dots \} \]
	On a $\bigcap\limits_{n \in \mathbb{N}} A_n = \emptyset$ et $\nu(A_n) = +\infty$ pour tout $n \in \mathbb{N}$.
\end{remarque}

\begin{definition}[Vocabulaire des mesures]
	Soit $\mu$ une mesure sur $(E, \mathscr{A})$.
	\begin{itemize}
		\item $(E, \mathscr{A}, \mu)$ est un espace mesuré si $(E,\mathscr{A})$ est un espace mesurable et $\mu$ est une mesure sur $ (E, \mathscr{A} ) $.
		\item $\mu$ est dite finie si $\mu(E) < +\infty$.
		\item $\mu$ est une probabilité si $\mu(E) = 1$.
		\item $\mu$ est $\sigma$-finie si $E = \bigcup\limits_{n \in \mathbb{N}} A_n$ avec $A_n \in \mathscr{A}$ et $\mu(A_n) < +\infty$.
		\item On dit que $x$ est un atome de $\mu$ si $\mu(\{x\}) > 0$.
		\item $\mu$ est diffuse si elle n'a pas d'atomes.
	\end{itemize}
\end{definition}

\subsection{Fonctions mesurables}

\begin{definition}
	Soient $(E, \mathscr{A})$ et $(F, \mathscr{B})$ deux espaces mesurables. Une fonction $f: E \to F$ est dite mesurable si
	$f^{-1}(B) \in \mathscr{A}$ pour tout $B \in \mathscr{B}$.
	Dans le cas ou $E$ et $F$ sont munis de leur tribus boréliennes (si elles existent), on dit que $f$ est borélienne.
\end{definition}


\begin{prop}
	Si $f: (E, \mathscr{A}) \to (F, \mathscr{B})$ et $g: (F, \mathscr{B}) \to (G, \mathscr{C})$
	sont mesurables, alors $g \circ f : (E, \mathscr{A}) \to (G, \mathscr{C})$ est mesurable.
\end{prop}

\begin{proof}
	Soitn $C \in \mathscr{C}$. On a que $g^{-1}(C) \in \mathscr{B}$ car $g$ est mesurable et
	$f^{-1}(g^{-1}(C)) \in \mathscr{A}$ car $f$ est mesurable.
	Comme $f^{-1}(g^{-1}(C)) = (g \circ f)^{-1}(C)$, on a que $g \circ f$ est mesurable.
\end{proof}


\begin{prop}
	Pour que $f : (E, \mathscr{A}) \to (F, \mathscr{B})$ soit mesurable, il suffit qu'il existe $\mathscr{C} \subset \mathscr{B}$ telle que:
	\begin{enumerate}
		\item $f^{-1}(C) \in \mathscr{A}$ pour tout $C \in \mathscr{C}$.
		\item $\sigma(\mathscr{C}) = \mathscr{B}$
	\end{enumerate}
\end{prop}

\begin{proof}
	Posons $\mathscr{G} = \{ B \in \mathscr{B} \, | \, f^{-1}(B) \in \mathscr{A} \}$. On a que $\mathscr{C} \subset \mathscr{G}$.
	Montonrs qu $\mathscr{G}$ est une tribu sur $F$.
	\begin{enumerate}
		\item $f^{-1}(\emptyset) = \emptyset \in \mathscr{A}$ donc $\emptyset \in \mathscr{G}$.
		\item Soit $B \in \mathscr{G}$, alors $f^{-1}(B^c) = (f^{-1}(B))^c \in \mathscr{A}$ donc $B^c \in \mathscr{G}$.
		\item Soit $(B_n)_{n \in \mathbb{N}}$ une suite d'éléments de $\mathscr{G}$, alors
		      $f^{-1}(\bigcup\limits_{n \in \mathbb{N}} B_n) = \bigcup\limits_{n \in \mathbb{N}} f^{-1}(B_n) \in \mathscr{A}$ donc
		      $\bigcup\limits_{n \in \mathbb{N}} B_n \in \mathscr{G}$.
	\end{enumerate}
	Ainsi $\mathscr{G}$ est une tribu sur $F$ et $\mathscr{C} \subset \mathscr{G}$.
	On a donc
	\[ \sigma(\mathscr{B}) = \sigma(\mathscr{C}) \subset \sigma(\mathscr{G}) = \mathscr{G} \]
	Donc $\mathscr{B} \subset \mathscr{G} \subset \mathscr{B}$, donc $\mathscr{B} = \mathscr{G}$.
	On a donc que $f$ est mesurable.
\end{proof}


\begin{example}[Application]
	Si $F, \mathscr{B}) = (\mathbb{R}, \mathscr{B}(\mathbb{R}))$, on peut prendre $\mathscr{C} = \{ ]-\infty,\   t[ \, | \, t \in \mathbb{R} \}$. On a que
				$\sigma(\mathscr{C}) = \mathscr{B}(\mathbb{R})$. Et donc il suffit d'étudier la mesurabilité de $f^{-1}(]-\infty, t[)$.
\end{example}


\begin{remarque}
	Si $f : (E,\mathscr{B}(E)) \rightarrow (F,\mathscr{B}(F))$, alors $f$ continue $\implies f$ mesurable.
\end{remarque}


\begin{prop}
	La fonction $\mathscr{1}_A : (E,\mathscr{A}) \rightarrow (\mathbb{R},\mathscr{B}(\mathbb{R}))$ est messurable si et seulement si $A \in \mathscr{A}$.
\end{prop}

\begin{proof}
	$f^{-1}(]-\infty, t] = \emptyset si t < 0, A^c si t \in [0,1[ E si t \geq 1$
	donc $f$ messirable ssi $A^c\in \mathscr{A}$ ssi $A\in \mathscr{A}$
\end{proof}

\begin{prop}
	$f_1 : (E,\mathscr{A}) \rightarrow (F_1,\mathscr{B_1}$
	$f_2 : (E,\mathscr{A})) \rightarrow (F_2,\mathscr{B_2}))$,
	$g: (E \times E,\mathscr{A} \oplus \mathscr{A}) \rightarrow (F_1 \times F_2,\mathscr{B_1} \oplus \mathscr{B_2})$
	$x\mapsto (f_1(x), f_2(x))$
	alors $g$ est mesurable si $f_1$ et $f_2$ le sont
\end{prop}

\begin{proof}
	$\mathscr{B_1} \oplus \mathscr{B_2} = \sigma \left\{ B_1 \times B_2, B_1 \in \mathscr{B_1}, B_2 \in \mathscr{B_2} \right\}...$
	$g^{-1}(B_1 \times B_2) = {x \in E, f_1(x) \in B_1 et f_2(x) \in B_2  }$
	$f_1^{-1}(B_1) \cap f_2{-1}(B_2)$
	donc $g^{-1}(B_1 \times B_2) \in \mathscr{A}$
\end{proof}



\begin{prop}
	Si $f$ et $g (E,\mathscr{A}) \rightarrow (F,\mathscr{B}$
	sont mesurables, alors les foncntions suivantes sont mesurables:
	\begin{itemize}
		\item $f + g$
		\item $fg$
		\item $\inf(f,g)$
		\item $f^+ = \sup(f,0)$
		\item $f^- = \sup(-f,0)$
	\end{itemize}
\end{prop}

\begin{proof}
	$(x,y) \mapsto x + y$
	$(\mathbb{R}^2, \mathscr{B}(\mathbb{R}) \oplus \mathscr{B}(\mathbb{R})  \mathscr{B} \rightarrow (R,   \mathscr{B}(\mathbb{R})) $
	est continue donc mesurable. De meme pour le reste.
\end{proof}


\begin{definition}
	On note $\Rbar = \R \cup \{ -\infty, \infty \}$
	On note $\Rbarp = \R^+ \cup \{  \infty \}$
	Dans $\Rbarp\ \forall a \in \Rbarp: $
	\begin{itemize}
		\item $a + (+\infty) = +\infty$
		\item
		      $a * (+\infty) = \left\{ \begin{array}{cc}
                      0       & \text{si} \  a = 0 \\
                      +\infty & \text{sinon}
			      \end{array}\right.$
	\end{itemize}
\end{definition}

\begin{definition}
	Si $a_n$ est une suite dans $\Rbar$. On définit
	\begin{itemize}
		\item $\sup a_n =
			      \left\{ \begin{array}{cc}
				      +\infty         & \text{si} \ +\infty \ \text{est dans la suite}   \\
				      +\infty         & \text{si} \ \forall M > 0 \ \exists N,\, a_n > M \\
				      \sup a_n \in \R & \text{si} \ a_n\  \text{est majoré}
			      \end{array}\right.$

		\item De même pour le $\inf$
		\item $\limsup a_n = notation chiante = \lim\limits_{n \rightarrow \infty}\downarrow \sup\limits_{k \geq n } a_k \in \Rbar$
		\item $\liminf a_n = notation chiante = \lim\limits_{n \to \infty}\uparrow \inf\limits_{k \geq n } a_k \in \Rbar$
	\end{itemize}
\end{definition}


\begin{remarque}
	On travaillera avec $\bor(\Rbar)$, les boreliens de $\Rbar$ qui est
	$$\sigma (\left\{[-\infty, a], a \in \R\right\})$$
\end{remarque}


\begin{remarque}
	$\limsup a_n$ est la plus grande valeur d'adherance de la suite $a_n$
	$\liminf a_n$ est la plus petite valeur d'adherance de la suite $a_n$
\end{remarque}

\begin{prop}
	Si $f_n$ est une suite de fonction messurables
	$$f_n (E,\triA) \rightarrow (\mathbb{R},\bor(\R))$$
	alors les fonctions suivantes sont mesurables:
	\begin{itemize}
		\item $\sup f_n x\mapsto f_n(x) $
		\item $\inf f_n$
		\item $\limsup f_n$
		\item $\liminf f_n$
	\end{itemize}
	En particulier si $f_n$ converge simplement dans $\Rbar$ alors
	$\lim f_n = \limsup f_n$ est mesurable. De plus $\{ x \in E, f_n(a) \text{converge}\}$ est mesurable.
\end{prop}

\begin{proof}
	$f(x) = \inf f_n (x)$
	\begin{eqnarray*}
		f^{-1}([-\infty, a[) &=& \{x \in E, \inf f_n < a\} \\
		&=& \{x \in E, \exists N \in \mathbb{N}, f_N (x) < a \}\\
		&=& \bigcup\limits_{n\in \mathbb{N}}f_N^{-1}([-\infty, a[))\\
		&=& \in \mathscr{A}
	\end{eqnarray*}
	de meme pour $\sup$

	\begin{eqnarray*}
		\liminf f_n &=& \lim\limits_{n \to \infty}\inf\limits_{k \geq n } a_k\\
		&=& \sup _{n\in N} \inf_{k \geq k} f_n\\
		&=& \in \mathscr{A}
	\end{eqnarray*}

	donc $\liminf$ et $\limsup$ sont mesurables.

	\begin{eqnarray*}
		\{x \in E, f_n \  \text{converge} \}  &=& (\liminf f_n - \limsup f_n)^{-1}(\{0\})\\
	\end{eqnarray*}
	et $\{ 0 \} \in \bor(\R)$ donc $\left\{ f_n \in E, f_n(x) \ \text{converge} \ \right\} \in \mathscr{A}$.
\end{proof}









\section{Intégration par rapport à une mesure}

\subsection{Intégrale d'une fonction mesurable positive}


On travaille avec $(E,\triA,\mu)$ un espace mesuré.

\begin{definition}[Fonction étagée]
	Soit $f: (E, \triA) \to (\R, \bor(\R))$ une fonction mesurable. On dit que $f$ est une fonction étagée
	si elle prend un nombre fini de valeurs.
	Si $\alpha_1, \dots, \alpha_n$ sont les valeurs prises par $f$, on note
	$A_i = f^{-1}(\{\alpha_i\})$ et on a
	$$ f = \sum_{i=1}^n \alpha_i \1_{A_i} $$
	on dit que c'est l'écriture canonique de $f$.
\end{definition}

\begin{remarque}
	$\1_{\Q}$ n'est pas étagée.
\end{remarque}


\begin{definition}[Intégrale d'une fonction positive]
	Si $f = \sum\limits_{i=1}^n \alpha_i \1_{A_i}$ est une fonction étagée positive, on définit
	$$ \int f d\mu = \sum_{i=1}^n \alpha_i \mu(A_i) \in [0, +\infty] $$
	Avec la convention $0 \cdot \infty = 0$.
\end{definition}


\begin{remarque}
	La valeur de $\int f d\mu$ ne dépend pas de l'écriture canonique de $f$, i.e. si
	$$ f = \sum_{i=1}^n \alpha_i \1_{A_i} = \sum_{j=1}^m \beta_j \1_{B_j} $$
	alors
	$$ \sum_{i=1}^n \alpha_i \mu(A_i) = \sum_{j=1}^m \beta_j \mu(B_j) $$
\end{remarque} %TODO: Add proof ?

\begin{prop}[Linéarité et croissance]
	Soient $f,g$ deux fonctions étagées positives.
	\begin{enumerate}
		\item $  \forall a, b \in \R, \int (af + bg) d\mu = a \int f d\mu + b \int g d\mu$
		\item Si $f \leq g$ alors $\int f d\mu \leq \int g d\mu$
	\end{enumerate}
\end{prop}

\begin{proof}
	\begin{itemize}
		\item $ f = \sum_{i=1}^n \alpha_i \1_{A_i} $ et $ g = \sum_{j=1}^m \alpha'_j \1_{A'_j} $
		      On introduit $B_{ij} = A_i \cap A'_j$, $\beta_{ik} = \alpha_i$ et $\beta'_{ik} = \alpha'_k$.
		      On a alors
		      $$ f = \sum_{i=1}^n \sum_{j=1}^m \beta_{ij} \1_{B_{ij}}, \ \int f \mu = \sum_{i,j} \beta_{ij} \mu(B_{ij}) $$
		      et
		      $$ g = \sum_{i=1}^n \sum_{j=1}^m \beta'_{ij} \1_{B_{ij}}, \ \int g \mu = \sum_{i,j} \beta'_{ij} \mu(B_{ij} $$
		      On a alors
		      \begin{eqnarray*}
			      \int (af + bg) d\mu & = & \int \left( \sum_{i=1}^n \sum_{j=1}^m (a\beta_{ij} + b\beta'_{ij}) \1_{B_{ij}} \right) d\mu \\
			      & = & \sum_{i,j} (a\beta_{ij} + b\beta'_{ij}) \mu(B_{ij}) \\
			      & = & a \sum_{i,j} \beta_{ij} \mu(B_{ij}) + b \sum_{i,j} \beta'_{ij} \mu(B_{ij}) \\
			      & = & a \int f d\mu + b \int g d\mu
		      \end{eqnarray*}
		\item Si $f \leq g$ alors $g - f $ est étagée positive.
		      \begin{eqnarray*}
			      \int g d\mu & = & \int (f + (g-f)) d\mu \\
			      & = & \int f d\mu + \underbrace{\int (g-f)}_{\geq 0} d\mu
		      \end{eqnarray*}
		      Donc $\int g d\mu \geq \int f d\mu$.
	\end{itemize}
\end{proof}

Soit  $\mathcal{E}_+$ l'ensemble des fonctions étagées positives.


\begin{definition}[Intégrale d'une fonction mesurable positive]
	Soit $f: (E, \triA) \to (\Rbarp, \bor(\Rbarp))$ une fonction mesurable positive.
	On définit
	$$ \int f d\mu = \sup\limits_{\substack{g \in \mathcal{E}_+ \\ g \leq f}} \int g d\mu $$
\end{definition}

\begin{remarque}
	L'ensemble en question n'est pas vide car $0 \in \mathcal{E}_+$ et $0 \leq f$. Et donc
	$$ \int f d\mu \geq \int 0 d\mu = 0 $$
\end{remarque}

\begin{remarque}
	Si $g$ est étagée positive,
	$$\sup\limits_{\substack{h \in \mathcal{E}_+ \\ h \leq g}} \int h d\mu \leq \int g d\mu \text{ intégrale définie précédemment}$$
	et de plus $h \in \mathcal{E}_+$ et $h \leq g$ et donc
	$$ \int g d\mu \leq \sup\limits_{\substack{h \in \mathcal{E}_+ \\ h \leq g}} \int h d\mu $$
	et donc on a l'égalité entre les deux définitions.
\end{remarque}

\begin{remarque}
	On notera:
	\begin{itemize}
		\item $\int f d\mu$
		\item $\int f(x) d\mu(x)$
		\item $\int f(x) \mu(dx)$
		\item $\int\limits_E f(x) \mu(dx)$
		\item $\int\limits_{x\in E} f(x) d\mu(x)$
		\item Et même $\mu(f)$
	\end{itemize}
\end{remarque}


\begin{prop}[Croissance et séparation] \label{prop:croi-et-sep}
	\begin{itemize}
		\item Si $f,g$ mesurables positives $f \leq g$ alors $\int f d\mu \leq \int g d\mu$
		\item Si $\mu(\left\{x : f(x) > 0\right\}) = 0$ alors $\int f d\mu = 0$
	\end{itemize}
\end{prop}

\begin{proof}
	\begin{itemize}
		\item $\left\{ h \in \mathcal{E}_+ : h \leq f \right\} \subset \left\{ h \in \mathcal{E}_+ : h \leq g \right\}$ donc
		      $$ \int f d\mu \leq \int g d\mu $$
		\item Soit $h$ étagée positive telle que $h \leq f$.
		      $h^{-1}(\overline{\R^{+*}}) \subset f^{-1}(\overline{\R^{+*}})$ et donc
		      $$ \mu(h^{-1}(\overline{\R^{+*}})) \leq \mu(f^{-1}(\overline{\R^{+*}})) = 0 $$
		      Donc $h = 0 \cdot \1_{A_i} + \sum_{j=1}^n \alpha_j \1_{A_j}$
		      Alors $A_i \subset h^{-1}(\{0\})$ et donc $\mu(A_i) = 0$.
		      Donc $\int h d\mu = 0$.
		      et donc $\int f d\mu = 0$.
	\end{itemize}
\end{proof}


\begin{theorem}[Convergence monotone] \label{thm:convergence_monotone}
	Soit $(f_n)_{n \in \N}$ une suite croissante de fonctions mesurables positives.\\
	On note $f = \lim\limits_{n \to \infty}\uparrow f_n$.
	Alors on a
	$$ \int f d\mu = \lim\limits_{n \to \infty} \int f_n d\mu $$
\end{theorem}

\begin{remarque}
	$f_n$ suite croissante de fonctions (et pas suite de fonctions croissantes ...).
	$$ \forall n \in \N, f_n \leq f_{n+1} $$
\end{remarque}


\begin{proof}
	\begin{itemize}
		\item $$f_n (x) \leq f_{n+1}(x)$$ et donc $\int f_n d\mu \leq \int f_{n+1} d\mu$.
		      et finalement $$\lim\limits_{n \to \infty} \int f_n d\mu \leq \int f d\mu$$.
		\item Soit $h \in \mathcal{E}_+$ telle que $h \leq f$, montrons que $\int h d\mu \leq \lim\limits_{n \to \infty} \int f_n d\mu$.\\
		      Soit $a \in [0,1[$
		      $$ E_n = \left\{ x \in E :  ah(x) \leq f_n(x) \right\} $$
		      $E_n$ est mesurable car $E_n = (ah-f_n)^{-1}(\R^-)$.\\
		      On a $f_n \to f$ et donc $a < 1$ et donc $ah < f$ et donc pour un
		      $n$ assez grand, $f_n \geq ah$.\\
		      Donc $E = \bigcup_{n \in \N} E_n$.\\
		      Or $ f_n \geq ah \1_{E_n} $ et donc
		      \begin{eqnarray*}
			      \int f_n d\mu \geq \int ah \1_{E_n} d\mu &=& \sum_{i=1}^k a \alpha_i \mu (A_i \cap E_n) \\
			      \text{car} \ ah\1_{E_n} 0 \sum_{i=1}^k a \alpha_i \1_{A_i \cap E_n} \  \text{est étagée positive}& \\
			      &=& a \sum_{i=1}^k \alpha_i \mu(A_i \cap E_n) \\
		      \end{eqnarray*}
		      or $E_n$ est une suite croissante d'ensembles avec $\bigcup_{n \in \N} E_n = E$ et donc
		      \begin{eqnarray*}
			      \lim\limits_{n \to \infty} \mu(A_i \cap E_n) &=& \mu(A_i)\\
			      \lim \limits_{n \to \infty} \int f_n d\mu &\geq& a \int h d\mu
		      \end{eqnarray*}
		      Comme c'est vrai pour tout $a \in [0,1[$, on a
		      \begin{eqnarray*}
			      \lim \limits_{n \to \infty} \int f_n d\mu &\geq&  \int h d\mu\\
			      \lim \limits_{n \to \infty} \int f_n d\mu &\geq& \sup\limits_{\substack{h \in \mathcal{E}_+ \\ h \leq f}} \int h d\mu = \int f d\mu
		      \end{eqnarray*}
	\end{itemize}
\end{proof}


\begin{example}[Contre-exemple à la convergence monotone pour une suite non croissante]
	$f_n = \1_{[n, \infty[}$. $f_n \to 0, \ \forall x $.\\
	Et on a  $\int f_n d\lambda = \infty$ or $\int 0 d\lambda = 0$.
\end{example}

\begin{prop}
	Soit $f$ mesurable positive.\\
	Alors il existe $f_n$ suite croissante de fonctions étagées positives telles
	$$\forall x \in E, f_n(x) \to f(x)$$
\end{prop}


\begin{proof} %TODO: Add graphics
	$$n \geq 1, i \in \left\{0, \dots, 2^n - 1\right\}$$
	\begin{eqnarray*}
		A_{n} &=& \left\{ x \in E : f(x) \geq n \right\} \in \triA \\
		B_{n,i} &=& \left\{ x \in E : \frac{i}{2^n} \leq f(x) < \frac{i+1}{2^n} \right\} \in \triA \\
		f_n & = & n \1_{A_n} + \sum_{i=0}^{2^n - 1} \frac{i}{2^n} \1_{B_{n,i}}
	\end{eqnarray*}

	En général, $A_n$ et $B_{n,i}$ ne sont pas des intervalles.

	Par construction $f_n \leq fn$ et $f_n$ est une suite croissante, c'est pour cette raison qu'on a subdivisé avec $2^n$ intervalles, et pas $n$ intervalles.\\
	A-t-on $f_n \to f$ ?

	\begin{eqnarray*}
		f_n(x) =  2^{-1} i_n(x)\underbrace{\1_{B_n,i_n(x)}(x)}_{=1} &\text{ Si } \1_n(x) \text{ est tel que } x \in B_{n,i_n(x)} & \\
		& 2^{-n}i_n(x) \leq f_n(x) \leq 2^{-n}(i_n(x) + 1) &\\
		& \text{donc } 2_{-n}i_n(x) \to f(x) &
	\end{eqnarray*}
	Et donc $f_n(x) \to f(x)$
\end{proof}

\begin{remarque}
	Les fonctions données par la proposition précédente vérifient les hypothèses du théorème \ref{thm:convergence_monotone}, donc
	$$ \int f_n d\mu \to \int f d\mu $$
\end{remarque}


\begin{prop}[Linéarité]
	Soient $f,g$ mesurables positives et $a, b\geq 0$.
	$$ \int (af + bg) d\mu = a\int f d\mu + b\int g d\mu $$
\end{prop}


\begin{proof}
	$$f_n \in \mathcal{E}_+, f_n \uparrow \to f$$
	$$ g_n \in \mathcal{E}_+, g_n \uparrow \to g$$
	$$ \underbrace{\int a f_n + b g_n \,d\mu}_{\text{Par } \ref{thm:convergence_monotone} \to \int a f + b g\,d\mu}
		= a \underbrace{\int f_n d\mu}_{ \text{Par } \ref{thm:convergence_monotone} \to \int f d\mu}
		+ b \underbrace{\int g_n d\mu}_{ \text{Par } \ref{thm:convergence_monotone} \to \int g d\mu} $$
\end{proof}



\begin{prop} [TCM pour les suites]
	Soit $f_n$ une suite de fonctions mesurables positives.\\
	Alors,
	$$ \int \sum\limits_n^{\infty} f_n d\mu = \sum\limits_n^{\infty} \int f_n d\mu $$
\end{prop}

\begin{proof}
	On regarde $S_n(x) = \sum\limits_{k=1}^n f_k(x)$ est une suite de fonctions mesurables positives.
	Et $S_{n+1} - S_n = f_{n+1} > 0 $ et donc $S_n$ est une suite croissante de fonctions mesurables positives.
	D'après le \ref{thm:convergence_monotone}, on a
	\begin{eqnarray*}
		\int \lim\limits_{n \to \infty} S_n (x)d\mu (x)&=& \lim\limits_{n \to \infty} \int S_n(x) d\mu(x) \\
		\int \sum\limits_{n = 1}^{\infty} S_n (x)d\mu (x)&=& \lim\limits_{n \to \infty} \int \sum\limits_{k = 1}^n S_n(x) d\mu(x) \\
		&=& \lim\limits_{n \to \infty} \sum\limits_{k = 1}^n\int  f_k(x) d\mu(x) \\
		&=& \sum\limits_{n = 1}^{\infty} \int  f_k(x) d\mu(x) \\
	\end{eqnarray*}
\end{proof}


\begin{definition}[$\mu$-presque partout]
	Soit $P$ une propriété sur $x \in E$ ($P(x) \in \{Vrai, Faux\}$)\\
	On dit que $P$ est vraie $\mu$-presque partout ($\mu-pp$) si
	$$ \left\{x \in E : P(x)  = Faux \right\} \in \triA \text{ et } \mu\left(\left\{x \in E : P(x)  = Faux \right\}\right) = 0 $$
	ou s'il existe $B \in \triA$ tel que $\left\{ x \in E : P(x) = Faux \right\} \subset B$ et $\mu(B) = 0$
\end{definition}

\begin{definition}[Mesure à densité]
	Soit $f$ une fonction mesurable positive. \\
	On définit
	\begin{eqnarray*}
		\nu : \triA & \to &\Rbarp                  \\
		A     & \mapsto & \int \1_A f d\mu \\
		&& = \int_A fd\mu
	\end{eqnarray*}
	Alors $\nu$ est une mesure sur $(E, \triA)$ et on l'appelle mesure à densité par rapport à $\mu$.
\end{definition}

\begin{proof}
	\begin{itemize}
		\item L'ensemble de départ est mesurable.
		\item $\nu(\emptyset) = \int \1_{\emptyset} f d\mu = \int 0 d \mu = 0$
		\item $A_n$ suite d'ensembles disjoints on regarde
		      \begin{eqnarray*}
			      \nu\left(\bigcup A_n\right) &=& \int \1_{\bigcup A_n} f d\mu \\
			      &=& \int \left( \sum_{k=1}^{\infty} \1_{A_k} \right) f d\mu \\
			      &=& \int \sum_{k=1}^{\infty} \1_{A_k} f d\mu \\
			      &=& \sum_{k=1}^{\infty} \int \1_{A_k} f d\mu
		      \end{eqnarray*}
		      d'après le TCM pour les séries, car $\forall k \geq 1, \ \1_{A_k}f$ est mesurable positive.
	\end{itemize}
\end{proof}

\begin{remarque} \label{rem:transfert}
	On veut veut, pour $g$ mesurable positive, montrer que
	$$ \int g d\nu = \int g f d\mu $$
\end{remarque}

\begin{proof}
	\begin{itemize}
		\item Vrai pour $\1_A$ ?
		\item Vrai pour $g$ étagée positive ?
		\item Vrai pour  $g$ étagée positive ?
	\end{itemize}

	\begin{itemize}
		\item Si $g = \1_A$, avec $A \in \mathscr{F}$, alors
		      \begin{eqnarray*}
			      \int g d \nu &=& \int \1_A d\nu  = \nu(A) \\
			      &=& \int \1_A f d\mu = \int g f d\mu
		      \end{eqnarray*}
		\item Si $g = \sum_{i=1}^n \alpha_i \1_{A_i}$, alors
		      \begin{eqnarray*}
			      \int g d \nu = \sum_{i=1}^n \alpha_i \1_{A_i} \nu(A_i)  &=& \sum_{i=1}^n \alpha_i \int f \1_{A_i} d \mu \\
			      &=&\int \left(\sum_{i=1}^n \alpha_i \1_{A_i}\right) f d \mu \\
			      &=& \int g f d \mu
		      \end{eqnarray*}
		\item Si $g$ est étagée positive, alors
		      On prend $g_n \uparrow g$ une suite de fonctions étagées positives.
		      Alors \begin{eqnarray*}
			      \int g d \nu &=& \lim\limits_{n \to \infty} \int g_n d \nu  \text{ par le théorème \ref{thm:convergence_monotone}}\\
			      &=& \lim\limits_{n \to \infty} \int g_n f d \mu
		      \end{eqnarray*}
		      On remarque que $g_n f$ est une suite croissante de fonctions mesurables positives et donc
		      on peut appliquer le théorème \ref{thm:convergence_monotone} et donc
		      \begin{eqnarray*}
			      \lim\limits_{n \to \infty} \int g_n f d \mu & = & \int  \lim\limits_{n \to \infty} g_n f d \mu \\
			      &=& \int g f d \mu
		      \end{eqnarray*}
	\end{itemize}
\end{proof}

\begin{example}[la normale sur $\R$]
	(Même proba sur  $(\R, \bor(\R))$, qui est la loi de $X ~ \mathscr{N}(0,1)$)
	\begin{eqnarray*}
		\nu(B) = \Pro_\lambda(B) &= &\Pro(X \in B)  \\
		&=& \int \1_B(x) \frac{1}{\sqrt{2\pi}} e^{-\frac{X^2}{2}} d \lambda(x)\\
		&=&\int \1_B f d\lambda
	\end{eqnarray*}

	Donc $\nu$ a la même densité que $f$ par rapport à la mesure de Lebesgue:
	$$f (x) = \frac{1}{\sqrt{2\pi}} e^{-\frac{X^2}{2}} $$
\end{example}




\begin{prop}[Inégalité de Markov]
	Soit $f$ une fonction mesurable positive.
	\begin{itemize}
		\item Pour tout $a > 0$ on a
		      $$ \mu\left(\left\{ x \in E : f(x) \geq a \right\}\right) \leq \frac{1}{a} \int f d\mu $$
		\item Si $\int f d\mu < \infty$ alors $f < \infty \ \mu-pp$
		\item $\int f d\mu = 0$ si et seulement si $f = 0 \ \mu-pp$
		\item $f=g \ \mu-pp \implies \int f d\mu = \int g d\mu$
	\end{itemize}
\end{prop}

\begin{proof}
	\begin{itemize}
		\item $f \geq a \1_{f \geq a}$
		      $$\int f d \mu \geq a \mu \left( \left\{ x \in E : f(x) \geq a \right\} \right) $$
		\item $A_k = \left\{ x \in E : f(x) \geq a \right\}$
		      On a $A_{\infty} = \bigcup_{k=0}^{\infty} A_k$ intersection décroissante.
		      \begin{eqnarray*}
			      \mu(A_1) &=& 1 \int f d \mu \text{ d'après  le (1)}\\
			      &<& \infty \text{ par hypothèse}
		      \end{eqnarray*}
		      donc \begin{eqnarray*}
			      \mu {A_{\infty}} &=& \lim\limits_{n \to \infty} \mu(A_n) \\
			      \mu(A_{n} &\leq& \frac{1}{n} \int f d u
		      \end{eqnarray*}

		      Donc $\mu (An) \to 0$ et donc $\mu(A_{\infty}) = 0$.
		\item $f= 0 \ \mu-pp \implies \int f d\mu = 0$  par  \ref{prop:croi-et-sep}.\\
		      $\leftarrow$ Soit  $f$ mesurable positive telle que $\int f d\mu = 0$.\\
		      $B_n = \left\{ x \in E : f(x) \geq \frac{1}{n} \right\}$\\
		      $\bigcup B_n$ est une suite croissante. \\
		      $\bigcup\limits_{n \in \N} = \left\{ x \in E : f(x) > 0 \right\}$\\

		      \begin{eqnarray*} %TODO: Add prof
		      \end{eqnarray*}

	\end{itemize}
\end{proof}


\begin{theorem}[Lemme de Fatou]
	Soit $f_n$ est une suite de fonctions mesurables positives.\\
	Alors
	$$ \int \liminf f_n d\mu \leq \liminf \int f_n d\mu $$
\end{theorem}

\begin{proof}
	$\liminf f_n = \lim\limits_{n \to \infty} \uparrow \inf_{k \geq n} f_k$\\
	On regarde $g_n = \inf_{k \geq n} f_k$ est une suite croissante de fonctions mesurables positives.\\
	Donc d'après le théorème \ref{thm:convergence_monotone}, on a
	$$ \lim\limits_{n \to \infty} \int g_n d\mu = \int \lim\limits_{n \to \infty} g_n d\mu = \int \liminf f_n d\mu $$
	Si $p \geq n$, alors $g_n \leq f_p$ et donc
	$$ \int g_n d\mu \leq \int f_p d\mu $$
	et donc
	$$ \forall p \geq n, \int f_p d\mu \geq \int g_p d\mu $$
	Et donc $\forall n \in \N$:
	$$ \inf_{p \geq n} \int f_p d\mu \geq \int g_n d\mu $$
	Et en passant à la limite, on
	$$ \liminf \int f_n d\mu \geq \int \liminf f_n d\mu $$
\end{proof}


\begin{example} $([0,1], \bor([0,1]),\lambda)$\\
	\begin{eqnarray*}
		f_{2k} &=& \1_{[\frac{1}{2}, 1]}\\
		f_{2k+1} &=& \1_{[0,\frac{1}{2}]}\\
		\liminf f_n &=& \1_{\{\frac{1}{2}\}}\\
		\int f_n d \lambda &=& \frac{1}{2}
	\end{eqnarray*}
	$$ \frac{1}{2} = \liminf \int f_n d \lambda \geq \int \liminf f_n d \lambda = 0 $$
\end{example}


\begin{theorem}[Égalité Reinmann-Lebesgue sur un segment, fonctions positives]
	Soit $f$ mesurable positive:
	$ f ([a,b], \bor([a,b])) \to \R, \bor(\R)$.\\
	On suppose $f$ Reinmann intégrable sur $[a,b]$.\\
	$$ \int_a^b f(x) dx = \int_{[a,b]} f d\lambda $$
\end{theorem}

\begin{proof} %TODO: Add proof





\end{proof}

\subsection{Exemples de calculs d'intégrales}

\begin{example}
	La mesure de Dirac
	$$ \int f d \delta_x = f(x) $$
\end{example}

\begin{proof}
	\begin{itemize}
		\item $f$ indicatrice
		\item $f$ étagée
		\item $f$ mesurable positive
	\end{itemize}
	En utilisant ce schéma, on peut montrer l'égalité, comme pour la remarque \ref{rem:transfert}. %TODO: Add details
\end{proof}

\begin{example} %TODO Add \nu_c definition
	Mesure de comptage:
	$$ \int f d \nu_c = \sum_{n \in \N} f(n)$$
\end{example}



\subsection{Fonctions intégrables}

On travaille sur $(E, \triA, \mu)$.

\begin{definition}
	Soit $f$ mesurable, $f : (E, \triA) \to (\R, \bor(\R))$.\\
	On dit que $f$ est intégrable par rapport à $\mu$ si :
	$$\int f^+ d\mu < +\infty \text{ et } \int f^- d\mu < +\infty$$
	où $f^+ = \max(f,0)$ et $f^- = -\min(f,0)$.
	On définit alors
	$$ \int f d\mu = \int f^+ d\mu - \int f^- d\mu $$
\end{definition}

\begin{remarque}
	De façon équivalente, on dit que $f$ est intégrable si
	$$ \int |f| d\mu < +\infty$$
\end{remarque}

\begin{remarque}
	$$ f = f^+ - f^- $$
	$$ |f| = f^+ + f^- $$
\end{remarque}

\begin{remarque}
	$f^+$ et $f^-$ sont mesurables positives, donc $\int f^+d\mu$ et $\int f^- d\mu$ sont définies.
\end{remarque}

\begin{remarque}
	Si $f \geq 0$ on retrouve bien la définition originale.
\end{remarque}

\begin{remarque}
	Par contre, si $\int f^+ d\mu = +\infty$ on dit que $f$ n'est pas intégrable, même si elle est positive.
\end{remarque}


\begin{definition}
	On note $\Li$ l'espace des fonctions intégrables.
\end{definition}

\begin{prop}[Propriétés de l'intégrale]
	\begin{itemize}
		\item Inégalité triangulaire. $\left| \int f d\mu \right| \leq \int |f| d \mu$.
		\item Linéarité: $\Li$ est un espace vectoriel.
		\item Croissance: si $f,g \in \Li$ et $f\leq g$ alors $\int f d\mu \leq \int g d\mu$
		\item Si $f,g \in \Li$ $f = g \mupp$ alors $\int f d\mu = \int g d\mu$.
	\end{itemize}
\end{prop}

\begin{proof}
	\begin{itemize}
		\item  \begin{eqnarray*}
			      |\int f d\mu| &=& \left|\int f^+ d\mu - \int f^- d\mu \right| \\
			      &\leq& \left|\int f^+ d\mu \right| + \left|\int f^- d\mu \right| \\
			      &\leq& \int f^+ d\mu + \int f^- d\mu  \\
			      \text{linéarité fonctions positives } &\leq& \int f^+ f^- d\mu  \\
			      &\leq& \int |f| d\mu
		      \end{eqnarray*}
		\item %TODO
		\item $f \leq g$, $g = f + (g-f)$
		      $$\int g d\mu = \int f d\mu + \int (g-f) d \mu$$
		      Or $(g-f)^- = 0$, donc $\int (g-f) d\mu > 0$.
		      Donc $\int f d\mu \leq \int g d\mu$.
		\item Si $f = g$ $\mupp$, alors $f^+ = g^+$ $\mupp$ et $f^- = g^-$ $\mu$pp.\\
		      Donc $\int f d\mu = \int f^+ \mu + \int f^-  d\mu = \int g^+ \mu + \int g^-  d\mu = \int g d\mu$
	\end{itemize}
\end{proof}

\begin{definition}[Intégrale des fonctions complexes]
	$$ F: (E, \triA) \to (\C, \bor(\C)) \text{ mesurable.}$$
	Ce qui est équivalent à dire que $Re(f)$ et $Im(f)$ sont mesurables.\\
	On dit que $f$ est intégrable et on note
	$$ f\in \LiC$$
	si $$\int |f| d\mu < +\infty$$
	On pose
	$$ \int f d\mu = \int Re(f) d\mu + \int Im(f) d\mu $$
\end{definition}


\begin{prop}
	\begin{itemize}
		\item Inégalité triangulaire. $\left| \int f d\mu \right| \leq \int |f| d \mu$.
		\item Linéarité: $\LiC$ est un espace vectoriel.
		\item Si $f,g \in \Li$ $f = g \mupp$ alors $\int f d\mu = \int g d\mu$.
	\end{itemize}
\end{prop}


\begin{proof}
	\begin{itemize}
		\item $$\forall b \in \C, \ |b| = \sup_{a_1, a_2 \in \R \\ a_1^2 + a_2^2 = 1 } a_1 Re(b) + a_2 Im(b)$$
		      % TODO
		\item
	\end{itemize}
\end{proof}



\begin{theorem}[Convergence dominée]
	Soit $f_n$ une suite de fonctions dans $\Li$ avec:
	\begin{itemize}
		\item Il existe $f$ mesurable à valeurs dans $\R$ telle que $f_n \to f, \ \mupp$.
		\item Il existe $g : (E,A) \to (\R^+, \bor(\R^+))$ mesurable positive avec $\int g d\mu < +\infty$ telle que
		      $$\forall n \in\N,\ |f_n(x)| \leq g(x)\ \mupp$$
	\end{itemize}

	Alors $f \in \Li$ et
	$$\lim_{n\to \infty} \int f_n d\mu= \int f d \mu$$
	et de plus
	$$ \int |f_n - f | d \mu \to 0 $$
\end{theorem}


\begin{example}[Contre-exemple sans domination]
	$$f_n =\frac{1}{n} \1_{[0,n[} \to 0 \ \mupp$$
	$$ \forall n, \ \int f_n d\mu = 1 \to 1 \neq 0 = \int 0 d\mu$$
\end{example}

\subsection{Intégrales dépendant d'un paramètre}

\begin{theorem}[Continuité sous le signe intégrale]
	Soit $f : U \times E \to \R$ où $(U,D)$ est un espace métrique.\\
	On suppose:
	\begin{itemize}
		\item $\forall u \in U,\ f (u, \cdot) : (E, \triA) \to (\R, \bor(\R))$ mesurable.
		\item $\mu-pp$ (en $x$) $f(\cdot, x) : U  \to \R$ est continue en $u_0 \in U$.
		\item Il existe $g$ intégrable telle que
		      $$\forall u \in U, \ \left| f(u,x) \right| \leq g(x) \mu-pp \text{ en } x$$
	\end{itemize}
	Alors la fonction
	$$F (u) = \int f(u,x) d\mu(x)$$
	est bien définie pour tout $u \in U$ et elle est continue en $u_0 \in U$.
\end{theorem}


\begin{proof}
	Il faut montrer que si $u_n$ et une suite avec $u_n \to u_0, \ n > 0$, alors $F(u_n) \to F(u_0)$.\\
	F est bien définie car $\forall u \in U, f(u,\cdot)$ mesurable et $|f(u,\cdot)| \leq g(\cdot)$ qui est intégrable.\\

	Posons $f_n(x) = f(u_n,x)$ et $f_n(x) \to f(u_0,x)$.\\
	$$|f_n(x)|= |f(u_n, x)| \leq g(x) \mu-pp$$
	D'après le TCD , $\underbrace{\int f_n(x)d\mu(x)}_{=F(u_n)} \to  \underbrace{\int f(u_0, x)d\mu(x)}_{= F(u_0)}$
\end{proof}

\begin{example}
	Soit $\mu$ une mesure diffuse sur $(\R, \bor(\R))$ et $\phi \in \Li(\R, \bor(\R), \mu)$.
	On définit \begin{eqnarray*}
		F(u) &=&  \int_{]-\infty, u]} \phi(x)d\mu(x)\\
		& = & \int \phi \1_{]-\infty, u]}(x)d\mu(x)
	\end{eqnarray*}
	alors $F$ est continue.

	\begin{proof}
		On pose $f(u,x) = \phi(x)\1_{]-\infty, u]}(x)$
		Il suffit de vérifier que $f(\cdot, x)$ est continue et qu'elle est dominée. Le reste est trivial: \\
		\begin{itemize}
			\item $u \mapsto f(u,x) = \left\{
				      \begin{array}{l}
					      \phi(x) \text{ si } u \geq x \\
					      0 \text{ sinon}
				      \end{array}
				      \right.$ \\
			      est continue en $\R\setminus\{x\}$. Comme $\mu$ est diffuse, $\mu \{x\} = 0$ et donc $f(\cdot, x)$ est continue $\mu-pp$.
			\item $|f(u,x)| = |\phi(x)\1_{]-\infty, u]}(x)| \leq |\phi(x)|$ qui est intégrable.
		\end{itemize}
	\end{proof}
\end{example}


\begin{theorem}[Dérivation sous le signe intégrale]
	On suppose que $U = I$ est un intervalle ouvert de $\R$ et $u_0 \in I$.\\
	$$ f: I\times E \to \R$$
	\begin{itemize}
		\item $\forall u \in I, \ f(u, \cdot) \in \Li_R(E,\triA, \mu)$
		\item $\mu-pp,\  f(\cdot, x)$ est dérivable en $u_0 \in I$ de dérivée $\diffp{f}{u} (u_0, x)$.
		\item Il existe $g\in \Li$ telle que
		      $$ \forall u\in I |f(u,x) - f(u_0,x)| \leq g(x)|u-u_0| \mu-pp$$
	\end{itemize}

	Alors $F(u) = \int f(u,x)d\mu$ est dérivable au point $u_0$ et sa dérivée est $F'(u_0) = \int \diffp{f}{u}(u_0,x)d\mu(x)$
\end{theorem}

\begin{remarque}
	La fonction $\diffp{f}{u}(u_0,\cdot)$ n'est définie que $\mu-pp$. Il suffit de
	la prolonger n'importe comment et cela suffit a définir $\int \diffp{f}{u}(u_0,x)d\mu(x)$.
	Comme le prolongement se fait sur un ensemble de mesure nulle, cela ne change pas la valeur de l'intégrale,
	c'est pour cela que l'on peut donner une liberté absolue pour le prolongement, par exemple en lui donnant la valeur 0.
\end{remarque}

\begin{proof}
	Soit $u_n \to u_0, \ n > 0, \ u_n \neq u_0$. \\
	On regarde
	\begin{eqnarray*}
		\frac{F(u_n)-F(u_0)}{u_n-u_0} &=& \frac{1}{u_n-u_0} \int f(u_n,x) -(u_0,x)df\mu(x) \\
		&=& \int \frac{f(u_n,x) - f(u_0,x)}{u_n-u_0}d\mu(x)
	\end{eqnarray*}
	\begin{itemize}
		\item $\frac{f(u_n,x)- f(u_0,x)}{u_n-u_0} \to \diffp{f}{u}(u_0,x)$
		\item $\left|\frac{f(u_n,x)- f(u_0,x)}{u_n-u_0} \right| \leq g(x)$
	\end{itemize}
	Donc d'après le TCD on a %TDOD: Add ref
	$$ \frac{F(u_n)- F(u_0)}{u_n-u_0} \to \int \diffp{f}{u}(u_0,x) d\mu(x)$$

\end{proof}

\begin{remarque}
	On peu changer les hypothèses 2 et 3 pour avoir une forme plus pratique:
	\begin{itemize}
		\item $\mu-pp\  f(\cdot, x)$ est dérivable sur $I$.
		\item Il existe $g\in \Li$ telle que
		      $$ \forall u\in I \left|\diffp{f}{u}(u,x)\right| \leq g(x) \mu-pp$$
		      Ceci implique , par le théorème des accroissements finis, la majoration de l'hypothèse 3.
	\end{itemize}
	et dans e cas on a que $F$ est dérivable sur $I$ tout entier.
\end{remarque}

\begin{remarque}
	Si $f$ est à valeurs complexes cela marche aussi.
\end{remarque}


\begin{example}[Trasformée de Fourier]
	Si $\phi \in \Li$ on définie sa transformée de Fourier:
	$\hat{\phi} = \int e^{iux}\phi(x) d\lambda(x)$
	alors $\hat{\phi}$ est bien définie dans $\R$ et continue (par le théorème de continuité sous le signe intégrale).\\
	Si de plus on a $\int |x\phi(x)|d\lambda(x)<\infty$ alors $\hat{\phi}$ est dérivable sur $\R$ de dérivée:
	$$\hat{\phi}'(x)= \int ixe^{iux}\phi(x)d\lambda(x)$$
	%TODO add proof
\end{example}



\begin{theorem}[Régularité $C^k$ sous le signe intégrale]
	Soit $f: I \times E \to \R$ où $I$ est un intervalle ouvert de $\R$\\

	\begin{itemize}
		\item $\forall u \in I, \ f(u,\cdot) \in \Li(E,\triA,\mu)$
		\item $\mu-pp u \ \mapsto f(u,x)$ est $C^k$ sur $I$.
		\item Il existe $g_k\in \Li$ telle que
		      $$ \forall u\in I \left|\diffp[i]{f}{u}(u,x)\right| \leq g_k(x) \mu-pp$$
	\end{itemize}

	Alors $F(u) = \int f(u,x)d\mu$ est $C^k$ sur $I$, avec $F^{(i)}(u) = \int \diffp[i]{f}{u}(u,x)d\mu(x)$

\end{theorem}





\end{document}
