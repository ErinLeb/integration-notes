\subsection{Construction de la mesure produit}

\begin{theorem}
	Soient $\mu$ et $\nu$ deux mesures $\sigma$-finies sur $(E, \triA)$ et $(F, \triB)$ respectivement.

	\begin{itemize}
		\item Il existe une unique mesure $\sigma$-finie sur ($E \times F, \triA \xor \triB)$ telle que $\forall A \in \triA, B \in \triB, \mu \xor \nu(A \times B) = \mu(A) \nu(B)$.
		\item $\forall C \in \triA \xor \triB. $
		      $$ \mu \xor \nu(C) = \int_E \nu(C_x) d \mu(x) = \int_F \mu(C^y) d \nu(y) $$
	\end{itemize}
\end{theorem}

\begin{proof}
	\begin{itemize}
		\item  Unicité:\\
		      $\mu$ et $\nu$ sont $\sigma$-finies, on se donne donc $E_n \uparrow E$ et $F_n \uparrow F$ tels que $\mu(E_n) < +\infty$ et $\nu(F_n) < +\infty$.\\
		      On pose $C_n = E_n \times F_n$ et on a que $C_n \uparrow E \times F$.\\
		      Soient $m_1$ et $m_2$ deux mesures qui vérifient les propriétés de la proposition.\\
		      \begin{eqnarray*}
			      m_1(E_n \times F_n) &=& \mu(E_n) \nu(F_n) < +\infty \\
			      & = & m_2(E_n \times F_n)
		      \end{eqnarray*}

		      \begin{itemize}
			      \item $m_1$ et $m_2$ coincident sur $\set{A \times B \mid A \in \triA, B \in \triB}$ qui engendre $\triA \xor \triB$ et stable par intersections finies.
			      \item $m_1 (C_n) = m_2(C_n) < +\infty$ et $C_n \uparrow E \times F$ donc $m_1 = m_2$ d'après le théorème d'unicité des mesures.
		      \end{itemize}
		      %TODO: It's really long, idk if it's worth it. If someone wants to do it, go ahead.
	\end{itemize}
\end{proof}

\begin{remarque}
	On peu itérer $\mu_1 \xor \mu_2 \xor \dots \xor \mu_n = \mu_1 \xor (\mu_2 \xor (\dots \xor (\mu_{n-1} \xor \mu_n)\dots)$ et c'est associatif.
	$$ \mu_1 \xor \mu_2 \xor \dots \xor \mu_n(A_1 \times \dots \times A_n) = \prod_{i=1}^n \mu_i(A_i) $$
\end{remarque}


\begin{example}
	La mesure de Lebesgue sur $\R^n$ est le produit des mesures de Lebesgue sur $\R$:

	$$\lambda_n = \underbrace{\lambda \xor \dots \xor \lambda}_{n \text{ fois}}$$
	et elle vérifie:
	$$\lambda_n([a_1, b_1] \times \dots \times [a_n, b_n]) = \prod_{i=1}^n (b_i - a_i)$$
\end{example}

\begin{remarque}
	Si $X \indep Y$, alors $\mathbb{P}_{X,Y} = \mathbb{P}_X \xor \mathbb{P}_Y$.
	%TODO: Add random probability stuff
\end{remarque}
