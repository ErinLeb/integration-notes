\subsection{L'espace de Hilbert $L^2$}

\begin{definition}
	On définit pour $f,g \in L^2(E,\triA, \mu)$.
	$$\sprod{f}{g} = \int fg d \mu$$
	est un profit scalaire sur $L^2(E, \triA, \mu)$.
	\begin{itemize}
		\item $(f,g) \mapsto \sprod{f}{g}$ est bilinéaire
		\item $(f,g) \mapsto \sprod{f}{g}$ est symétrique
		\item $\sprod{f}{f} > 0$ et $\sprod{f}{f} = 0 \mssi f = 0$ (presque partout)
	\end{itemize}
\end{definition}


\begin{proof}
	\begin{itemize}
		\item par linéarité de l'intégrale
		\item évident
		\item $ \int f^2 d \mu = 0 \iff f^2 = 0 \mu pp (2.1.12)$
	\end{itemize}
\end{proof}


\begin{remarque}
	En particulier, $\norm{f}_2 = \sqrt{\sprod{f}{f}}$ est une norme sur l'espace vectoriel $L^2_\R(E, \triA, \mu)$.
\end{remarque}

\begin{definition}
	On dit que $(E, \sprod{\cdot}{\cdot})$  est un espace de Hilbert si $E$ est muni d'un produit scalaire $\sprod{\cdot}{\cdot}$ et
	si $E$ est complet pour la norme induite $\norm{\cdot} = \sqrt{\sprod{\cdot}{\cdot}^2}$.
\end{definition}

\begin{theorem}[Riesz]
	$L^2(E, \triA, \mu)$ est un espace de Hilbert.
\end{theorem}

\begin{proof}
	Soit $f_n$ une suite de Cauchy dans $L^2(E, \triA, \mu)$, alors il existe une sous-suite $f_{k_n}$ telle que
	$$\norm{f_{k_{n+1}} - f_{k_n}}_2 \leq 2^{-n}$$
	On pose $g_n = f_{k_n}$
	\begin{eqnarray*}
		\int \left( \sum\limits_{n=1}^\infty \abs{g_{n+1} - g_n} \right)^2 d \mu &=& \lim_{N \to \infty} \int \left( \sum\limits_{n=1}^N \abs{g_{n+1} - g_n} \right)^2 d \mu \\
		&=& \lim_{N \to \infty} \int \left( \sum\limits_{1 \leq i, j \leq N} \abs{g_{i+1} - g_i} \abs{g_{j+1} - g_j} \right) d \mu \\
		&=& \lim_{N \to \infty}   \sum\limits_{1 \leq i, j \leq N} \int\abs{g_{i+1} - g_i} \abs{g_{j+1} - g_j}  d \mu \\
		&\leq& \lim_{N \to \infty} \sum\limits_{1 \leq i, j \leq N} \norm{g_{i+1} - g_i}_2 \norm{g_{j+1} - g_j}_2 \\
		&\leq& \lim_{N \to \infty} \left( \sum\limits_{n=1}^N \norm{g_{n+1} - g_n}_2 \right)^2 \\
		&<& \infty
	\end{eqnarray*}
car $\norm{g_{n+1} - g_n}_2 \leq 2^{-n}$.
En particulier, $\left( \sum\limits_{n=1}^\infty \abs{g_{n+1} - g_n} \right)^2$ est $\mu-intégrable$, donc finie $\mu-pp$ donc $ \sum\limits_{n=1}^\infty \abs{g_{n+1} - g_n} $ est finie $\mu-pp$.
$ \sum\limits_{n=1}^\infty g_{n+1}(x) - g_n(x)$ est absolument convergente $\mu-pp$ (en x).
%todo finish proof (gl)
\end{proof}

\begin{example}
	$(\N, \mathcal{P}(\N), \mu)$ où $\mu$ est la mesure de comptage.
	$$l^2(\N) = \set{ a \in \R^\N \mid \sum a_n^2 < \infty} = L^2(\N, \mathcal{P}(\N), \mu)$$
	$$\sprod{a}{b} = \sum a_n b_n$$
	$$\norm{a}_2 = \sqrt{\sum a_n^2}$$
\end{example}

\begin{example}
	$ E = \R/\Z ( = B_1)$
	On regarde $\mu$ la mesure de Lebesgue induite sur $[0,1]$.
	$$ L^2(\R/\Z, \bor (\R/\Z), \mu) = \set{ f: \R \to \R, \  1-\text{périodique} \mid \int_{[0,1]} f^2 d\lambda < \infty } $$
	C'est dans cet espace qu'on fera des séries de Fourier.
\end{example}

\begin{remarque}
	Si les fonctions sont à valeurs complexes, on définit le produit scalaire par:
	$$\sprod{f}{g} = \int f \overline{g} d \mu$$
	Cela munit $L^2(E, \triA, \mu)$ d'une structure de espace de Hilbert complexe.
\end{remarque}
