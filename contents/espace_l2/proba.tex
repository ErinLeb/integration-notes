\subsection{Lien avec les probabilités}




Soit $(\Omega, \triF, \Pro)$ espace probabilisé.

$X : (\Omega,    \triF) \to (\R,  \bor(\R)) $ mesurable "variable aléatoire".

C'est quoi la loi?

\begin{eqnarray*}
    \Pro_x: \bor (\R) &\to& [0,1] \\
	B &\mapsto& \Pro(X  \in B)
\end{eqnarray*}

$(\Pro_x)$ est une mesure sur $(\R,  \bor(\R)).$

Soit $X$ une variable aléatoire réelle positive\\
\begin{eqnarray*}
	\E[X] &=& \int_0^{+\infty} \Pro (X> t) d t\\
	&=& \int_0^{+\infty}\int_\R \1_{X(\omega)>t}d \Pro (\omega) d t\\
	&=& \int_{\R^+} \int_{\R} \1_{X(\omega)>t}d \Pro (\omega) d \lambda(t) \\ %TODO add remark
	&=& \int_{\Omega} \int_{\R^+} \1_{X(\omega)>t}d \lambda(t) d \Pro (\omega)  \\%TODO add underbrace
	&=& \int_{\Omega} X(\omega) d \Pr(\omega)
\end{eqnarray*}

De plus,


\begin{eqnarray*}
	\E[X] &=& \int_{\R^+} \Pro_X(]t, +\infty[) d \lambda (t) \\
	&=& \int_{\R^+} \int_{\R^+} \1_{X(\omega)>t}d \Pro_X (\omega) d \lambda(t) \\
	&=& \int_{\R^+} \int_{\R^+} \1_{X(\omega)>t}d \lambda(t) d\Pro_X (\omega)  \\ %TODO underbrace
	&=& \int_{\R^+} x d\Pro_X (x)
\end{eqnarray*}

En particulier, si $\Pro_X = fd\lambda$, "X est à densité par rapport à Lebesgue", alors $\forall g$ mesurable positive,
$$\int g(x) d \Pro_X(x) = \int g(x) f(x) d \lambda(x)$$
et donc ("formule de transfert") :
$$ \E[g(X)] = \int g(x) f(x) d \lambda(x)$$

\begin{example}
	Soit $\mu$ une mesure et $f$ une fonction mesurable positive.\\
	Si $\nu(A) = \int f \1_A d \mu$, alors $\nu$ est la mesure de densité $f$ par rapport à $\mu$.
\end{example}


\subsubsection{Sur l'indépendance}

$X \indep Y, \quad \forall A,B \in \bor (\R)$

\begin{eqnarray*}
	\Pro_{(X,Y)}(A,B)= \Pro (X \in A \text{ et } Y \in B)
	&=& \Pro(X\in A) \Pro(Y\in B)\\
	&=& \Pro_X(A) \Pro_Y( B)
\end{eqnarray*}


$\Pro_{(X,Y)} \to [0,1]$ mesure sur $\R \times \R$ telle que
$$\forall A,B\Pro_{(X,Y)}(A\times B) = \Pro (X\in A \text{ et } Y \in B)$$

donc $X \indep Y\iff \Pro_{(X,Y)} = \Pro_x \xor \Pro_y$.

De même pour $(X_1, \dots, X_n)$, ils sont indépendants \ssi $\Pro_{(X_1,.., X_n)} =\Pro_{x_1} \xor ... \xor  \Pro_{x_n} $

