\subsection{Définitions, premières propriétés}

\begin{definition}
	Soit $H$ un $\R$ ou $\C$-espace vectoriel.
	Un produit scalaire est une application $\sprod{\cdot}{\cdot} : H \times H \to \R$ ou $\C$ telle que
	\begin{itemize}
		\item $\forall y \in H, x \mapsto \sprod{x}{y}$ est linéaire
		\item $\forall x,y \in H, \sprod{x}{y} = \overline{\sprod{y}{x}}$
		\item $\forall x \in H, \sprod{x}{x} \geq 0$ et $\sprod{x}{x} = 0 \iff x = 0$
	\end{itemize}

	Si $\sprod{\cdot}{\cdot}$ vérifie seulement les deux premières propriétés, on dit que c'est un produit hermitien.
\end{definition}

\begin{remarque}
	$\sprod{x}{\lambda y} = \overline{\lambda} \sprod{x}{y} $.
	$y \mapsto \sprod{x}{y}$ est antilinéaire.
	$(x,y) \mapsto \sprod{x}{y}$ dite sesquilinéaire (linéaire en la première variable et antilinéaire en la seconde).
	On note en général $\norm{x}_2 = \sqrt{\sprod{x}{x}}$. On montrera que c'est une norme.
\end{remarque}

\begin{remarque}
	\begin{eqnarray*}
		\norm{x + y}_2^2 &=& \sprod{x + y}{x + y}\\
		&=& \sprod{x}{x} + \sprod{x}{y} + \sprod{y}{x} + \sprod{y}{y}\\
		&=& \norm{x}_2^2 + 2\Re \sprod{x}{y} + \norm{y}_2^2
	\end{eqnarray*}
\end{remarque}

\begin{prop}[Inégalité de Cauchy-Schwarz]
	Soit $H$ un espace de Hilbert, alors
	$$ \forall x,y \in H, \abs{\sprod{x}{y}} \leq \norm{x}_2 \norm{y}_2 $$
\end{prop}

\begin{proof}
	\begin{itemize}
		\item Si $\norm y = 0, \ 0 \leq 0$.
		\item Sinon on regarde
		      $$ 0 \leq \norm{x + ty}^2 = \norm{x}^2 + 2t\Re \sprod{x}{y} + t^2 \norm{y}^2 $$
		      qui est un polynôme du second degré en $t$.
		      On pose $\Delta = 4\Re \sprod{x}{y}^2 - 4\norm{x}^2\norm{y}^2 \leq 0$.
		      Donc $\abs{\Re \sprod{x}{y}} \leq \norm x \norm y$.

		      Si on est dans un $\R$-espace vectoriel, on a fini. Sinon:
		      \begin{eqnarray*}
			      \sprod x y &=& \abs{\sprod x y} e^{i\theta} \\
			      \abs{\sprod x y} &=& e^{-i\theta} \sprod x y \\
			      &=&    \sprod {e^{-i\theta} x} y
		      \end{eqnarray*}
		      Or $\abs{\sprod x y} = \abs {\Re ( \sprod {e^{-i\theta} x} y )} \leq \norm {e^{-i\theta} x} \norm y = \norm x \norm y$.
	\end{itemize}
\end{proof}


\begin{coro}
	$\norm{x} = \sqrt{\sprod{x}{x}}$ définit une norme sur $H$ si $\sprod{\cdot}{\cdot}$ est un produit scalaire.
\end{coro}

\begin{proof}
	\begin{itemize}
		\item Par définition $\norm{x} \geq 0$.
		\item \begin{eqnarray*}
			      \norm{x + y}^2 &=& \sprod{x+y}{x+y} \\
			      &=&\norm x^2 + 2\Re \sprod{x}{y} + \norm{y}^2 \\
			      &\leq& \norm{x}^2 + 2\abs{\sprod{x}{y}} + \norm{y}^2 \\
			      &\leq& \norm{x}^2 + 2\norm{x}\norm{y} + \norm{y}^2 \text{ par Cauchy-Schwarz} \\
			      &\leq& (\norm{x} + \norm{y})^2
		      \end{eqnarray*}
		\item Soit $\lambda$ un scalaire:
		      \begin{eqnarray*}
			      \norm{\lambda x}^2 &=& \sprod{\lambda x}{\lambda x} \\
			      &=&\lambda \overline \lambda \sprod{x}{x} \\
			      &=& \abs{\lambda}^2 \norm{x}^2 \\
			      \norm{\lambda x} &=& \abs{\lambda} \norm{x}
		      \end{eqnarray*}
		\item $\norm{x} = 0 \implies \sprod{x}{x} = 0 \implies x = 0$ par définition du produit scalaire.
	\end{itemize}
\end{proof}

\begin{prop}[Identités remarquables]
	Soient $x,y \in H$ et $\sprod{\cdot}{\cdot}$ un produit scalaire sur $H$ et $H$ un $\C$-espace vectoriel.

	\begin{itemize}
		\item Identité de Polarisation:
		      $$\sprod{x}{y} = \frac{1}{4} \left( \norm{x+y}^2 - \norm{x-y}^2 + i\norm{x+iy}^2 - i\norm{x-iy}^2 \right)$$
		\item Dans un $\R$-espace vectoriel, on a:
		      $$\sprod{x}{y} = \frac{1}{4} \left( \norm{x+y}^2 - \norm{x-y}^2 \right)$$
		\item Identité de la médiane / du parallélogramme:
		      $$\norm{x+y}^2 + \norm{x-y}^2 = 2\left( \norm{x}^2 + \norm{y}^2 \right)$$
		\item Pythagore: Si $\sprod{x}{y} = 0$, alors
		      $$ \norm{x+y}^2 = \norm{x}^2 + \norm{y}^2$$
	\end{itemize}
\end{prop}

\begin{definition}
	On dit que $x$ et $y$ sont orthogonaux si $\sprod{x}{y} = 0$.
\end{definition}

\begin{proof}
	$$\norm{x+y}^2  = \norm{x}^2 + 2\Re \sprod{x}{y} + \norm{y}^2 $$
	$$\norm{x-y}^2  = \norm{x}^2 - 2\Re \sprod{x}{y} + \norm{y}^2 $$

	Par différence $\Re \left(\sprod{x}{y}\right) = \frac{1}{4} \left( \norm{x+y}^2 - \norm{x-y}^2 \right)$
	\begin{itemize}
		\item
		      \begin{eqnarray*}
			      \norm{x+y}^2  &=& \norm{x}^2 + 2\underbrace{\Re \sprod{x}{y} }_{= \Re (-i\sprod{x}{y}) = \Im \sprod{x}{y}} + \norm{y}^2 \\
			      \norm{x-iy}^2 &=& \norm{x}^2 - 2\Im \sprod{x}{y} + \norm{y}^2 \\
			      \Im \sprod{x}{y} &=& \frac{1}{4} \left( \norm{x+iy}^2 - \norm{x-iy}^2 \right)\\
			      \sprod{x}{y} &=& \Re \sprod{x}{y} + i \Im \sprod{x}{y}
		      \end{eqnarray*}
		      D'où l'identité de polarisation.
		\item On démontre la deuxième identité et¡n utilisant la différence, car sur un $\R$-espace vectoriel, $\Re \left(\sprod{x}{y}\right) = \sprod{x}{y}$.
		\item L'identité de la médiane est immédiate en additionnant les deux premières équations.
		\item Et enfin, si $\sprod{x}{y} = 0$, alors $\norm{x+y}^2 = \norm{x}^2 + \norm{y}^2$.
	\end{itemize} %TODO: Maybe add figure
\end{proof}


\begin{remarque}
	Dans un $\C$-espace vectoriel, le théorème de Pythagore n'admet pas de réciproque.
\end{remarque}

\begin{definition}
	$H$ muni de son produit scalaire est un espace de Hilbert s'il est complet pour la norme associée au produit scalaire.
\end{definition}

