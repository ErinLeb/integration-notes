\subsection{Projection orthogonale dans un espace de Hilbert}

\begin{definition}
	$F$ un sous espace de $H$, on définit l'othogonal de $F$ par $F^\perp = \{y \in H, \forall x \in F, \sprod{x}{y} = 0\}$.
\end{definition}

\begin{prop}
	$F, G$ deux sous-espaces vectoriels de $H$, alors:
	\begin{itemize}
		\item $F^\perp$ est fermé.
		\item Si $F \subset G$ alors $G^\perp \subset F^\perp$.
		\item $F^\perp = \overline{F}^\perp$.
	\end{itemize}
\end{prop}


\begin{proof}
	\begin{itemize}
		\item Soit $x_n \in F^\perp$ tel que $x_n \to x$, montrons que $x \in F^\perp$.
		      Soit $y \in F$, montrons que $\sprod{x}{y} = 0$.
		      $$ \abs{\sprod{x_n-x}{y}} \leq  \underbrace{\norm{x_n-x}}_{\to 0} \norm{y} $$
		      $$ \underbrace{\sprod{x_n}{y}}_{=0} - \sprod{x}{y}  \to 0 $$
		      donc la suite stationne en 0, donc $\sprod{x}{y} = 0$.

		\item Soit $x \in G^\perp$, montrons que $x \in F^\perp$.\\
		      Soit $y \in F$ montrons que $\sprod{x}{y} = 0$. Or $F \subset G$ donc $y \in G$ et donc $\sprod{x}{y} = 0$.
		\item Par le 2 $\overline{F}^\perp \subset F^\perp$.\\
		      Montrons que $F^\perp \subset \overline{F}^\perp$.\\
		      Soit $x \in F^\perp$, montrons que $x \in \overline{F}^\perp$.\\
		      Soit $y \in \overline{F}$, montrons que $\sprod{x}{y} = 0$.\\
		      Soit $y_n \in F$ tel que $y_n \to y$, on a
		      $$ \abs{\sprod{x}{y- y_n}} \leq \norm{x} \norm{y-y_n} \to 0 $$
		      or
		      $$  \abs{\sprod{x}{y- y_n}} = \sprod{x}{y} -  \underbrace{\sprod{x}{y_n}}_{=0} $$
		      Donc $\sprod{x}{y} = 0$.
	\end{itemize}
\end{proof}

\begin{remarque}
	D'après Cauchy-Schwarz,
	$$ \forall x \in H, \  x \mapsto \sprod{x}{y} \text{ est continue} $$
	$$ \forall y \in H, \  y \mapsto \sprod{x}{y} \text{ est continue} $$
\end{remarque}

\begin{theorem}[Projection orthogonale sur un sous espace fermé]
	Soit $F$ un sous espace fermé de $H$.

	\begin{itemize}
		\item $\forall x \in H, \exists ! y \in F, \norm{x-y} = \dist{x} {F}$.\\
		      Où $\dist{x} {F} = \inf_{y \in F} \norm{x-y}$.\\
		      On note $y = p_F(x)$ cet unique point, cela définit une application $p_F : H \to F$.
		\item $p_F(x)$ est l'unique vecteur de $F$ tel que $x - p_F(x) \in F^\perp$.
		\item $\forall x \in H,\ \norm{x}^2 = \norm{p_F(x)}^2 + \norm{x - p_F(x)}^2$.
		\item $p_F$ est linéaire et continue.
		\item $F$ et $F^\perp$ sont dits supplémentaires orthogonaux.\\
		      $$ H = F + F^\perp \text{ et } F \cap F^\perp = \{0\} $$
		      Et $$\forall x \in F , \forall y \in F^\perp, \sprod{x}{y} = 0$$
		      On note cela $H = F \oplus F^\perp$.
	\end{itemize}
\end{theorem}


\begin{proof}

	\begin{itemize}
		\item Soit $y_n \in F$ tel que $\norm{x-y_n} \to \dist{x}{F}$. Comme $H$ est complet, il suffit de montrer que $y_n$ est de Cauchy.\\
		      On prend $a = x - y_n$ et $b = x - y_m$.
		      \begin{eqnarray*}
			      \norm{a + b}^2  + \norm{a - b}^2 & = & 2 \norm{a}^2 + 2 \norm{b}^2\\
			      \norm {a-b}^2 & = & 2 \left( \norm{a}^2 +  \norm{b}^2 \right) -  \norm{a+b}^2\\
			      \norm {y_n - y_m}^2 & = & 2 \left( \norm{x-y_n}^2 +  \norm{x-y_m}^2 \right) - 4 \norm{x - \frac{y_n + y_m}{2}}^2 \\
			      & \leq & 2 \left( \norm{x-y_n}^2 +  \norm{x-y_m}^2 \right) - 4 {\dist x F}^2  %TODO: Underlines
		      \end{eqnarray*}

		      Si $N$ est assez grand pour que $\forall n \geq N, \abs {\norm{x-y_n}^2 - \dist{x}{F}^2} \leq\frac{\varepsilon}{4}$.\\
		      Si $n,m \geq N$ on a
		      \begin{eqnarray*}
			      \norm {y_n - y_m}^2 & \leq & 2 \left( \dist{x}{F}^2 +   \frac \varepsilon 4 + \dist{x}{F}^2 +   \frac \varepsilon 4 \right) - 4 {\dist x F}^2 \\
			      &\leq & \varepsilon
		      \end{eqnarray*}
		      $y_n$ est donc de Cauchy et donc converge vers un point $y \in H$, $y_n \in F$ donc $y \in F$ car $F$ est fermé.\\

		      Unicité: Supposons que $y$ et $y'$ soient deux points de $F$ qui réalisent le minimum.\\
		      $$ \norm{y - y'}^2 = 2 \left( {\dist x F}^2 + {\dist x F}^2 \right) - 4 {\dist x F}^2 \leq 0 $$
		      Donc $y = y'$.

		\item Montrons que $x - p_F(x) \in F^\perp$.\\
		      Soit $z \in F$, montrons que $\sprod{x - p_F(x)}{z} = 0$.\\
		      On pose $y = p_F(x) \in F$. Soit $t\in \Rp$, on a $y + t z \in F$ Donc
		      \begin{eqnarray*}
			      {\dist x F}^2 & \leq & \norm{x - (y + t z)}^2\\
			      & \leq & \norm{x - tz - y}^2\\
			      & = & \norm{x - y}^2 - 2\Re \sprod{x - y}{tz} + t^2 \norm{z}^2\\
			      & \leq & {\dist x F}^2 - 2t \Re \sprod{x - y}{z} + t^2 \norm{z}^2
		      \end{eqnarray*}

		      Donc $2t\Re \sprod{x - y}{z} + t^2 \norm{z}^2 \geq 0$ pour tout $t \in \Rp$.
		      En particulier pour $t = 0$ on a $\Re \sprod{x - y}{z} = 0$ \\
		      En utilisant $itz$ à la place de $tz$ on trove $\Im \sprod{x - y}{z} = 0$\\

		      Unicité: Supposons $y$ et $y'$ deux points de $F$ tels que $\forall z \in F, \sprod{x - y}{z} = 0$ et $\sprod{x - y'}{z} = 0$.\\
		      $\forall x \in F, \sprod{y - y'}{z} = 0$, or $y - y' \in F$ donc $\sprod{y - y'}{y - y'} = 0$ donc $y = y'$.
		\item D'après 2, $x - p_F(x) \perp p_F(x)$, donc par Pythagore,
		      $$ \norm{x}^2 = \norm{p_F(x)}^2 + \norm{x - p_F(x)}^2 $$
		\item Montrons que $p:F( x + \lambda y) = p_F(x) + \lambda p_F(y)$.\\
		      Montrons que $p_F(x) + \lambda p_F(y)$ vérifie 2.\\
		      Montrons que $x + \lambda y - p_F(x) - \lambda p_F(y) \in F^\perp$.\\
		      $$ \underbrace{x - p_F(x)}_{\in F^\perp} + \lambda ( \underbrace{y - p_F(y)}_{\in F^\perp}) \in F^\perp $$
		      Montrons que $p_F$ est continue.\\
		      Comme $p_F$ est linéaire, il suffit de montrer que $p_F$ est continue en 0.\\
		      $p_F$ continue \ssi $\exists M, \ \norm{p_F(x)} \leq M \norm{x}$.\\

		      D'après 3, $\norm{p_F(x)}^2 = \norm{x}^2 - \norm{x - p_F(x)}^2 \leq \norm{x}^2$. Donc $p_F$ est continue et
		      $$\normop{p_F(x)} \leq 1 $$ %TODO: command for this

		\item $x = \underbrace{p_F(x)}_{\in F} + \underbrace{(x - p_F(x))}_{\in F^\perp \text { d'après 2}}$.
	\end{itemize}
\end{proof}

\begin{remarque}
	Si $F$ est de dimension finie, $(e_1, \ldots, e_n)$ une base orthonormée de $F$, alors $p_F(x) = \sum_{i=1}^n \sprod{x}{e_i} e_i$.
\end{remarque}

\begin{remarque}
	Si $F$ n'est pas fermé rien ne fonctionne.
\end{remarque}

\begin{coro}
	\begin{itemize}
		\item Si $F$ est fermé, alors ${F^\perp}^\perp = F$.
		\item $F$ est dense \ssi $F^\perp = \set 0$.
	\end{itemize}
\end{coro}

\begin{proof}
	\begin{itemize}
		\item C'est le point 5 du théorème.
		\item $\overline{F}^\perp = F^\perp$

		      \begin{itemize}
			      \item Si $F$ est dense, $\overline{F}^\perp = H^\perp = \set 0$.
			      \item Sinon $\overline{F} \subsetneq H$  et $ \overline{F} \oplus \overline{F}^\perp = H$.
			            Donc $\overline{F}^\perp \neq \set 0$.
		      \end{itemize}
	\end{itemize}

\end{proof}
