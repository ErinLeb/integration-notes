\subsection{Bases orthonormales}


\begin{definition}
	On dit que $H$ est séparable s'il existe une partie dénombrable dense de $H$. Ou encore une suite de points dense dans $H$.
\end{definition}


\begin{example}
	$\R$ est séparable ($\Q$ est dense dans $\R$ et dénombrable).
\end{example}

\begin{example}
	$\Ld$ est séparable.
\end{example}

\begin{proof}
	On a vu que les fonctions continues à support compact sont denses dans $\Ld$.


	Toute fonction continue à support compact peut être approchée par une fonction étagée à valeurs rationnelles,
	avec pour ensemble de niveaux des pavés à coordonnées rationnelles.
	$$ \sum_{k=-n^2}^{n^2} f_{\epsilon} \left(\frac k n\right) \1_{[\frac k n, \frac {k+1} n[} $$
	avec $f_{\epsilon}$ un rationnel $\frac {\epsilon} 4$-proche de $f\left(\frac k n\right)$.

	\begin{itemize}
		\item approche $f$ en $\norm{\cdot}_2$.
	\end{itemize}

	%TODO
\end{proof}

\begin{example}
	$\Ld(\N)$ et $\Q^{(\N)}$ l'ensemble des suites nulles à partir d'un certain rang à valeurs rationnelles.
\end{example}

\begin{definition}[Base orthonormale / hilbertienne]
	On dit que $(e_n)_{n\in \N}$ est une base orthonormale/hilbertienne de $H$ si
	\begin{itemize}
		\item $\norm {e_n} = 1, \ \sprod {e_i} {e_j} = \delta_{ij} = \begin{cases} 1 & \text{si } i=j \\ 0 & \text{sinon} \end{cases}$
		\item L'ensemble des combinaisons linéaires finies d'éléments $(e_n)_{n\in \N}$ est dense dans $H$.
	\end{itemize}
\end{definition}


\begin{example}
	$\ld(\N)$  et $\sprod  a b = \sum\limits_{k=1}^n a_k \overline{b_k}$.
	$e_k = (0, \ldots, 0, 1, 0, \ldots)$ avec $1$ en position $k$.
	$$ \sprod {e_i} {e_j} = \sum e_i \overline{e_j} = \delta_{ij} $$

	L'ensemble des combinaisons linéaires finies d'éléments de $(e_n)_{n\in \N}$ est $\C^{(\N)}$ l'ensemble des suites nulles à partir d'un certain rang, donc dense dans $\Ld(\N)$.
\end{example}

\begin{example}
	$$\ld_{\C}([0,1], \bor([0,1]), \lambda)$$
	$$\sprod f g = \int_{[0,1]} f \overline g d \lambda$$

	on pose \begin{align*}
		e_n : [0,1] & \to \C               \\
		t           & \mapsto e^{2ni\pi t}
	\end{align*}

	\begin{eqnarray*}
		\sprod {e_n} {e_m} &=& \int e^{2ni\pi t} e^{-2mi\pi t} d \lambda (t) \\
		&=& \int  e^{2i\pi (n-m) t} d \lambda (t) \\
		&=& \begin{cases} 1 & \text{si } n=m \\ 0 & \text{sinon} \end{cases}
	\end{eqnarray*}

	car $\int_0^1 cos (2 \pi t) d t = 0 = \int_0^1 sin(2 \pi t) d t$.
\end{example}


\begin{theorem}
	Si $H$ est un espace de Hilbert séparable, alors $H$ admet une base hilbertienne.
\end{theorem}

\begin{proof}
	Soit $A$ une partie dénombrable dense de $H$.

	\begin{itemize}
		\item On construit avec $A$ une famille orthonormale.
		\item On vérifie que c'est une base hilbertienne.
	\end{itemize}

	$A = \set { a_n }_{n\in \N}$.

	\begin{itemize}
		\item Montrons le premier point:\\

		      Montrons que $\forall n \in \N, \exists p \leq n \exists f_0, \dots, f_p \subset \set {a_0, \dots, a_n}$ tel que $(f_0, \dots, f_p)$ est libre dans $H$ et $\vect {f_0, \dots, f_p} = \vect {a_0, \dots, a_n}$.

		      On le montre par récurrence sur $n$.

		      \begin{itemize}
			      \item $n=0$ : $f_0 = a_0$ si $a_0 \neq 0$. $(a_0)$ est libre, sinon $p = -1$ (la famille est vide).
			      \item Hérédité :
			            \begin{itemize}
				            \item Si $a_{n+1} \in \vect {a_0, \dots , a_n}$ on garde $(f_0, \dots, f_p)$ convient.
				            \item Si $a_{n+1} \notin \vect {a_0, \dots , a_n}$, alors $(f_0, \dots, f_n, a_{n+1})$ convient.
			            \end{itemize}
		      \end{itemize}

		      On travaille avec $f_0, \dots, f_p$.
		\item Soit $F$ un sous espace vectoriel fermé. Supposons que $(e_1, \dots, e_n)$ une base orthonormale de $F$.

		      \begin{eqnarray*}
			      p_F: H & \to & H \\
			      x & \mapsto & \sum_{k=1}^n \sprod x {e_k} e_k
		      \end{eqnarray*}
		      vérifie $x - p_F \in F^{\perp}$. Donc c'est bien la projection orthogonale sur $F$.

		      Posons $F_n = \vect {f_0, \dots, f_n}$.
		      alors:
		      \begin{eqnarray*}
			      e_0 &=& \frac {f_0} {\norm {f_0}} \\
			      e_1 &=& \frac {f_1 - p_{F_0} (f_1)} {\norm {f_1 - p_{F_0} (f_1)}} \\
			      &\vdots& \\
			      e_n &=& \frac {f_n - p_{F_{n-1}} (f_n)} {\norm {f_n - p_{F_{n-1}} (f_n)}}\\
			      &=& \frac {f_n - \sum\limits_{k=0}^{n-1} \sprod {f_n} {e_k} e_k} {\norm {f_n - \sum\limits_{k=0}^{n-1} \sprod {f_n} {e_k} e_k}} \\
			      &\vdots&
		      \end{eqnarray*}

		      On vérifie que $\norm {e_n} = 1$ et par récurrence $\forall n \in \N, \forall i < n, \sprod {e_n} {e_i} = 0$.


		      Or $e_n \in F_n$ et $\vect {(e_n)_{n\in \N}} \supset A$ Et $\vect {(e_n)_{n\in \N}}$ est l'ensemble des combinaisons linéaires finies. Or $A$ est dense dans $H$, donc
		      $\vect {(e_n)_{n\in \N}} = H$.
	\end{itemize}
\end{proof}


\begin{prop}[Inégalité de Bessel]
	Soit $(e_n)_{n\in \N}$ une famille orthonormale de $H$.
	Alors $\forall x \in H, \sum_{n=0}^{\infty} \abs{\sprod x {e_n}}^2 \leq \norm x^2$.
\end{prop}


\begin{proof}
	%TODO
\end{proof}


\begin{prop}[Identité de Parseval]
	$H$ espace de Hilbert séparable, $(e_n)_{n\in \N}$ une base hilbertienne de $H$,
	alors:
	$$ \forall x \in H, x = \sum_{n=0}^{\infty} \sprod x {e_n} e_n $$
	$$ {\norm x}^2 = \sum_{n=0}^{\infty} \abs{\sprod x {e_n}}^2 $$
	$$ \forall x,y \in H, \sprod x y = \sum_{n=0}^{\infty} \sprod x {e_n} \overline{\sprod y {e_n}} $$
\end{prop}

\begin{proof}
	%TODO
\end{proof}



\begin{prop}[Unicité des coefficients]
	Sous les mêmes hypothèses, si
	$$ \sum_{n=0}^{k} \lambda_n e_n \to_{k \to \infty} x $$
	alors $\forall n \in \N, \lambda_n = \sprod x {e_n}$.

	On dit que $(\lambda_n)$ est la suite des coefficients de $x$ dans la base (hilbertienne) $(e_n)$.
\end{prop}


\begin{proof}
	%TODO
\end{proof}

