\subsection{Intégrale d'une fonction mesurable positive}


On travaille avec $(E,\triA,\mu)$ un espace mesuré.

\begin{definition}[Fonction étagée]
	Soit $f: E, \triA \to \R, \bor(\R)$ une fonction mesurable. On dit que $f$ est une fonction étagée
	si elle prend un nombre fini de valeurs.
	Si $\alpha_1, \dots, \alpha_n$ sont les valeurs prises par $f$, on note
	$A_i = f^{-1}(\{\alpha_1\})$ et on a
	$$ f = \sum_{i=1}^n \alpha_i \1_{A_i} $$
	on dit que c'est l'écriture canonique de $f$.
\end{definition}

\begin{remarque}
	$\1_{\Q}$ n'est pas étagée.
\end{remarque}


\begin{definition}[Intégrale d'une fonction positive]
	Si $f = \sum\limits_{i=1}^n \alpha_i \1_{A_i}$ est une fonction étagée positive, on définit
	$$ \int f d\mu = \sum_{i=1}^n \alpha_i \mu(A_i) \in [0, +\infty] $$
	Avec la convention $0 \cdot \infty = 0$.
\end{definition}


\begin{remarque}
	La valeur de $\int f d\mu$ ne dépend pas de l'écriture canonique de $f$, i.e. si
	$$ f = \sum_{i=1}^n \alpha_i \1_{A_i} = \sum_{j=1}^m \beta_j \1_{B_j} $$
	alors
	$$ \sum_{i=1}^n \alpha_i \mu(A_i) = \sum_{j=1}^m \beta_j \mu(B_j) $$
\end{remarque} %TODO: Add proof ?

\begin{prop}[Linéarité et croissance]
	Soient $f,g$ deux fonctions étagées positives.
	\begin{enumerate}
		\item $  \forall a, b \in \R, \int (af + bg) d\mu = a \int f d\mu + b \int g d\mu$
		\item Si $f \leq g$ alors $\int f d\mu \leq \int g d\mu$
	\end{enumerate}
\end{prop}

\begin{proof}
	\begin{itemize}
		\item $ f = \sum_{i=1}^n \alpha_i \1_{A_i} $ et $ g = \sum_{j=1}^m \alpha'_j \1_{A'_j} $
		      On introduit $B_{ij} = A_i \cap A'_j$, $\beta_{ik} = \alpha_i$ et $\beta'_{ik} = \alpha'_k$.
		      On a alors
		      $$ f = \sum_{i=1}^n \sum_{j=1}^m \beta_{ij} \1_{B_{ij}} $$ % TODO: Show explicit integral
		      et
		      $$ g = \sum_{i=1}^n \sum_{j=1}^m \beta'_{ij} \1_{B_{ij}} $$ % TODO: Show explicit integral
		      On a alors
		      \begin{eqnarray*}
			      \int (af + bg) d\mu & = & \int \left( \sum_{i=1}^n \sum_{j=1}^m (a\beta_{ij} + b\beta'_{ij}) \1_{B_{ij}} \right) d\mu \\
			      & = & \sum_{i,j} (a\beta_{ij} + b\beta'_{ij}) \mu(B_{ij}) \\
			      & = & a \sum_{i,j} \beta_{ij} \mu(B_{ij}) + b \sum_{i,j} \beta'_{ij} \mu(B_{ij}) \\
			      & = & a \int f d\mu + b \int g d\mu
		      \end{eqnarray*}
		\item Si $f \leq g$ alors $g - f $ est étagée positive.
		      \begin{eqnarray*}
			      \int g d\mu & = & \int (f + (g-f)) d\mu \\
			      & = & \int f d\mu + \underbrace{\int (g-f)}_{\geq 0} d\mu
		      \end{eqnarray*}
		      Donc $\int g d\mu \geq \int f d\mu$.
	\end{itemize}
\end{proof}

Soit  $\mathcal{E}_+$ l'ensemble des fonctions étagées positives.


\begin{definition}[Intégrale d'une fonction mesurable positive]
	Soit $f: E, \triA \to \Rbarp, \bor(\Rbarp)$ une fonction mesurable positive.
	On définit
	$$ \int f d\mu = \sup\limits_{\substack{g \in \mathcal{E}_+ \\ g \leq f}} \int g d\mu $$
\end{definition}

\begin{remarque}
	L'ensemble en question n'est pas vide car $0 \in \mathcal{E}_+$ et $0 \leq f$. Et donc
	$$ \int f d\mu \geq \int 0 d\mu = 0 $$
\end{remarque}

\begin{remarque}
	Si $g$ est étagée positive,
	$$\sup\limits_{\substack{h \in \mathcal{E}_+ \\ h \leq g}} \int h d\mu \leq \int g d\mu \text{ intégrale définie précédemment}$$
	et de plus $h \in \mathcal{E}_+$ et $h \leq g$ et donc
	$$ \int g d\mu \leq \sup\limits_{\substack{h \in \mathcal{E}_+ \\ h \leq g}} \int h d\mu $$
	et donc on a l'égalité entre les deux définitions.
\end{remarque}

\begin{remarque}
	On notera:
	\begin{itemize}
		\item $\int f d\mu$
		\item $\int f(x) d\mu(x)$
		\item $\int f(x) \mu(dx)$
		\item $\int\limits_E f(x) \mu(dx)$
		\item $\int\limits_{x\in E} f(x) d\mu(x)$
		\item Et même $\mu(f)$
	\end{itemize}
\end{remarque}


\begin{prop}[Croissance et séparation]
	\begin{itemize}
		\item Si $f,g$ mesurables positives $f \leq g$ alors $\int f d\mu \leq \int g d\mu$
		\item Si $\mu(\left\{x : f(x) > 0\right\}) = 0$ alors $\int f d\mu = 0$
	\end{itemize}
\end{prop}

\begin{proof}
	\begin{itemize}
		\item $\left\{ h \in \mathcal{E}_+ : h \leq f \right\} \subset \left\{ h \in \mathcal{E}_+ : h \leq g \right\}$ donc
		      $$ \int f d\mu \leq \int g d\mu $$
		\item Soit $h$ étagée positive telle que $h \leq f$.
		      $h^{-1}(\overline{\R^{+*}}) \subset f^{-1}(\overline{\R^{+*}})$ et donc
		      $$ \mu(h^{-1}(\overline{\R^{+*}})) \leq \mu(f^{-1}(\overline{\R^{+*}})) = 0 $$
		      Donc $h = 0 \cdot \1_{A_i} + \sum_{j=1}^n \alpha_j \1_{A_j}$
		      Alors $A_i \subset h^{-1}(\{0\})$ et donc $\mu(A_i) = 0$.
		      Donc $\int h d\mu = 0$.
		      et donc $\int f d\mu = 0$.
	\end{itemize}
\end{proof}


\begin{theorem}[Convergence monotone]
	Soit $(f_n)_{n \in \N}$ une suite corissante de fonctions mesurables positives.\\
	On note $f = \lim\limits_{n \to \infty}\uparrow f_n$.
	Alors on a
	$$ \int f d\mu = \lim\limits_{n \to \infty} \int f_n d\mu $$
\end{theorem}

\begin{remarque}
	$f_n$ suite croissante de fonctions (et pas suite de fonctions croissantes ...).
	$$ \forall n \in \N, f_n \leq f_{n+1} $$
\end{remarque}


\begin{proof}
	\begin{itemize}
		\item $$f_n (x) \leq f_{n+1}(x)$$ et donc $\int f_n d\mu \leq \int f_{n+1} d\mu$.
		      et finalement $$\lim\limits_{n \to \infty} \int f_n d\mu \leq \int f d\mu$$.
		\item Soit $h \in \mathcal{E}_+$ telle que $h \leq f$, montrons que $\int h d\mu \leq \lim\limits_{n \to \infty} \int f_n d\mu$.\\
		      Soit $a \in [0,1[$
		      $$ E_n = \left\{ x \in E :  ah(x) \leq f_n(x) \right\} $$
		      $E_n$ est mesurable car $E_n = (ah-f_n)^{-1}(\R^-)$.\\
		      On a $f_n \to f$ et donc $a < 1$ et donc $ah < f$ et donc pour un
		      $n$ assez grand, $f_n \geq ah$.\\
		      Donc $E = \bigcup_{n \in \N} E_n$.\\
		      Or $ f_n \geq ah \1_{E_n} $ et donc
		      \begin{eqnarray*}
			      \int f_n d\mu \geq \int ah \1_{E_n} d\mu &=& \sum_{i=1}^k a \alpha_i \mu (A_i \cap E_n) \\
			      \text{car} \ ah\1_{E_n} 0 \sum_{i=1}^k a \alpha_i \1_{A_i \cap E_n} \  \text{est étagée positive}& \\
			      &=& a \sum_{i=1}^k \alpha_i \mu(A_i \cap E_n) \\
		      \end{eqnarray*}
		      or $E_n$ est une suite croissante d'ensembles avec $\bigcup_{n \in \N} E_n = E$ et donc
		      \begin{eqnarray*}
			      \lim\limits_{n \to \infty} \mu(A_i \cap E_n) &=& \mu(A_i)\\
			      \lim \limits_{n \to \infty} \int f_n d\mu &\geq& a \int h d\mu
		      \end{eqnarray*}
		      Comme c'est vrai pour tout $a \in [0,1[$, on a
		      \begin{eqnarray*}
			      \lim \limits_{n \to \infty} \int f_n d\mu &\geq&  \int h d\mu\\
			      \lim \limits_{n \to \infty} \int f_n d\mu &\geq& \sup\limits_{\substack{h \in \mathcal{E}_+ \\ h \leq f}} \int h d\mu = \int f d\mu
		      \end{eqnarray*}
	\end{itemize}
\end{proof}


\begin{example}[ Contre-exemple à la convergence monotone pour une suite non croissante]
	$f_n = \1_{[n, \infty[}$. $f_n \to 0, \ \forall x $.\\
	Et on a  $\int f_n d\lambda = \infty$ or $\int 0 d\lambda = 0$.
\end{example}

\begin{prop}
	Soit $f$ mesurable positive.\\
	Alors il existe $f_n$ suite croissante de fonctions étagées positives telles
	$$\forall x \in E, f_n(x) \to f(x)$$
\end{prop}


\begin{proof} %TODO: Add graphics
	$$n \geq 1, i \in \left\{0, \dots, 2^n - 1\right\}$$
	\begin{eqnarray*}
		A_{n} &=& \left\{ x \in E : f(x) \geq n \right\} \in \triA \\
		B_{n,i} &=& \left\{ x \in E : \frac{i}{2^n} \leq f(x) < \frac{i+1}{2^n} \right\} \in \triA \\
		f_n & = & n \1_{A_n} + \sum_{i=0}^{2^n - 1} \frac{i}{2^n} \1_{B_{n,i}}
	\end{eqnarray*}

	En général, $A_n$ et $B_{n,i}$ ne sont pas des intervalles.

	Par construction $f_n \leq fn$ et $f_n$ est une suite croissante, c'est pour cette raison qu'on a subdivisé avec $2^n$ intervalles, et oas $n$ intervalles.\\
	A-t-on $f_n \to f$ ?

	TODO
\end{proof}


\begin{prop}[Linéarité]
	Soient $f,g$ mesurables positives et $a, b\geq 0$.
	$$ \int (af + bg) d\mu = a\int f d\mu + b\int g d\mu $$
\end{prop}


\begin{proof}
	$$f_n \in \mathcal{E}_+, f_n \uparrow \to f$$
	$$ g_n \in \mathcal{E}_+, g_n \uparrow \to g$$
	TODO
\end{proof}

