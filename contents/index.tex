\section{Introduction}

Ce document est un recueil de notes de cours sur l'intégration niveau L3. Il est
basé sur les cours de M.~\textsc{Cyrille Lucas} à Université Paris Cité, cependant toute
erreur ou inexactitude est de ma responsabilité.
Ce document a été rédigé principalement par \textsc{Yago Iglesias}, mais plusieurs contributeurs
peuvent être retrouvés dans la section contributeurs du répertoire
\href{https://github.com/Yag000/integration-notes/graphs/contributors}{GitHub}. Un remerciement
particulier est adressé à \textsc{Erin Le Boulc’h} pour sa participation active à la correction
de ce document.
\vspace{0.5cm}

Les notes portent sur la théorie de la mesure, l'integration et les séries de Fourier.
\vspace{0.5cm}

Toute erreur signalée ou remarque est la bienvenue.
Sentez-vous libres de contribuer à ce document par le biais de \href{https://github.com/Yag000/integration-notes}{GitHub},
où vous pouvez trouver le code source de ce document et une version pdf à jour.
Si vous n'êtes pas familiers avec \textit{Git} ou \LaTeX, vous pouvez toujours me contacter
par \href{mailto: yago.iglesias.vazquez@gmail.com}{mail}.


\section{Théorie de la Mesure}


\subsection{Tribus}


\begin{definition}[Tribu]
	Soit $E$ un ensemble. Une tribu $\mathscr{A}$ sur $E$ est une partie de $\mathscr{P}(E)$ vérifiant:
	\begin{enumerate}
		\item $E \in \mathscr{A}$
		\item $A \in \mathscr{A} \implies A^c \in \mathscr{A}$
		\item $ \forall n \in \mathbb{N}, A_n \in \mathscr{A} \implies \bigcup\limits_{n \in \mathbb{N}} A_n \in \mathscr{A}$
	\end{enumerate}
\end{definition}


\begin{prop}
	Toute intsersection de tribus sur $E$ est une tribu sur $E$.
\end{prop}

\begin{proof}
	Soit $i \in I$, $\mathscr{A}_i \forall i$ une tribu sur $E$. On a:
	\begin{itemize}
		\item $\forall  i \in I, E \in \mathscr{A}_i \implies E \in \bigcap\limits_{i \in I} \mathscr{A}_i$
		\item $\forall i \in I, A \in \mathscr{A}_i \implies \forall i \in I, \, A^c \in \mathscr{A}_i
			      \implies A^c \in \bigcap\limits_{i \in I} \mathscr{A}_i$
		\item $\forall i \in I, \forall n \in \mathbb{N}, A_n \in \mathscr{A}_i \implies
			      \forall i \in I, \,\bigcup\limits_{n \in \mathbb{N}} A_n \in \mathscr{A}_i \implies
			      \bigcup\limits_{n \in \mathbb{N}} A_n \in \bigcap\limits_{i \in I} \mathscr{A}_i$
	\end{itemize}
\end{proof}

\begin{definition}
	Soit $\mathscr{C}$ un sous ensemble de $\mathscr{P}(E)$. On note la tribu
	engrndrée par $\mathscr{C}$, $\sigma(\mathscr{C})$ avec
	\begin{equation*}
		\sigma(\mathscr{C}) = \bigcap\limits_{\mathscr{A} \in \mathscr{A}, \mathscr{C} \subset \mathscr{A}} \mathscr{A}
	\end{equation*}
\end{definition}

\begin{remarque}
	$\sigma(\mathscr{C})$ est bien une tribu comme intersection non vide de tribus,
	car $\mathscr{P}(E)$ est une tribu sur $E$ qui contient $\mathscr{C}$.
\end{remarque}

\begin{definition}[Tribu borélienne]
	On note $\omega$ l'ensemble des ouverst ed $\mathbb{R}$. La tribu borélienne sur
	$\mathbb{R}$ est la tribu $\sigma(\omega)$, notée $\mathscr{B}(\mathbb{R})$.
\end{definition}


\begin{remarque}
	On peut ettendre cette définition à tout space topologique (ou moins fort, tout space métrique). En particulier
	$\mathbb{R}^d$.
\end{remarque}

\begin{remarque}
	$\mathscr{B}(\mathbb{R})$ est aussi engrendrée par :
	\begin{itemize}
		\item $\{ ]-\infty, a[\,,\, a \in \mathbb{R} \}$
		\item $\{ ]a, b[\,,\, a < b \in \mathbb{R} \}$
		\item $\{ ]a, b[\,,\, a < b \in \mathbb{Q} \}$
	\end{itemize}
\end{remarque}


\begin{definition}[Tribu produit]
	Si $(E_1, \mathscr{A}_1)$ et $(E_2, \mathscr{A}_2)$ sont deux espaces mesurables ( couple ensemble-tribu compatible)
	on note $\mathscr{A}_1 \otimes \mathscr{A}_2$ la tribu sur $E_1 \times E_2$ engendrée
	par les rectangles $A_1 \times A_2$ avec $A_1 \in \mathscr{A}_1$ et $A_2 \in \mathscr{A}_2$.

\end{definition}




\subsection{Mesures}

On se donne $(E, \mathscr{A})$ un espace mesurable.

\begin{definition}[Mesure]
	On dit que $\mu : \mathscr{A} \to \mathbb{R}^+\cup \{+\infty\}$ est une mesure sur $(E, \mathscr{A})$ si:
	\begin{enumerate}
		\item $\mu(\emptyset) = 0$
		\item $\mu$ est $\sigma$-additive, c'est à dire que si $(A_n)_{n \in \mathbb{N}}$ est une suite d'éléments de $\mathscr{A}$
		      tels que $A_i \cap A_j = \emptyset$ pour $i \neq j$ (deux à deux disjoints), alors
		      \begin{equation*}
			      \mu\left(\bigcup\limits_{n \in \mathbb{N}} A_n\right) = \sum\limits_{n \in \mathbb{N}} \mu(A_n)
		      \end{equation*}
	\end{enumerate}
\end{definition}

\begin{remarque}
	On appelle mesurables les ensembles qui sont dans $\mathscr{A}$.
\end{remarque}

\begin{remarque}
	Comme $\mu (A_n) \in \mathbb{R}^+ \cup \{+\infty\}$, la somme $\sum\limits_{n \in \mathbb{N}} \mu(A_n)$ est bien définie.
\end{remarque}

\begin{remarque}
	$\mu$ mesurable donne l'additivité (finie). Cependant, la réciproque est fausse.
	\begin{example}
		Soit $m: \mathscr{P}(\mathbb{N}) \to \mathbb{R}^+ \cup \{+\infty\}$
		\begin{equation*}
			m(A) = \left\{
			\begin{array}{ll}
				0       & \text{ si } A \text{ est fini } \\
				+\infty & \text{ sinon }
			\end{array}
			\right.
		\end{equation*}
		est additive mais pas $\sigma$-additive.
	\end{example}
\end{remarque}

\begin{example}
	Sur $(\mathbb{R}, \mathscr{B}(\mathbb{R}))$, on a la mesure de
	Dirac en $x_0 \in \mathbb{R}$, notée $\delta_{x_0}$, définie par:
	\begin{equation*}
		\delta_{x_0}(A) = \left\{
		\begin{array}{ll}
			1 & \text{ si } x_0 \in A \\
			0 & \text{ sinon }
		\end{array}
		\right.
	\end{equation*}
\end{example}

\begin{example}
	Sur $(\mathbb{R}, \mathscr{B}(\mathbb{R}))$, on a la mesure de comptage, notée $\nu$, définie par:
	\begin{equation*}
		\nu(A) = \left\{
		\begin{array}{ll}
			\#A     & \text{ si } A \text{ est fini } \\
			+\infty & \text{ sinon }
		\end{array}
		\right.
	\end{equation*}
\end{example}

\begin{proof}
	\begin{enumerate}
		\item L'espace de départ est bien une tribu car $\mathscr{P}(\mathbb{R})$ est une tribu sur $\mathbb{R}$.
		      L'espace d'arrivée est bien $\mathbb{R}^+ \cup \{+\infty\}$ car $\#A \in \mathbb{R}^+ \cup \{+\infty\}$.
		\item $\nu(\emptyset) = \#\emptyset = 0$
		\item Si $(A_n)_{n \in \mathbb{N}}$ est une suite de boréliens deux à deux disjoints:
		      \begin{equation*}
			      \mu(\bigcup\limits_{n \in \mathbb{N}} A_n) = \left\{\begin{array}{ll}
				      \#A     & \text{ si } \bigcup\limits_{n \in \mathbb{N}} A_n \text{ est fini } \\
				      +\infty & \text{ sinon }
			      \end{array}
			      \right.
		      \end{equation*}
		      $\bigcup\limits_{n \in \mathbb{N}} A_n$ est fini si $\exists k ,\, A_k$ infini (cas 1) ou si les éléments sont
		      finis mais tous non vides à partir d'un certain rang (cas 2).
		      \begin{enumerate}
			      \item Cas 1 : $\sum\limits_{n \in \mathbb{N}} \nu(A_n) \geq \nu(A_k) = +\infty$.
			      \item Cas 2 : $\sum\limits_{n \in \mathbb{N}} \nu(A_n) = +\infty$ car
			            $\forall n \in \mathbb{N}, \, \nu(A_n) \in \mathbb{N}$ et $\nu(A_n)$ ne stationne pas en 0. Donc il existe une suite
			            infinie d'éléments non vides, donc tels que $\nu(A_n) \geq 1$. Donc la somme diverge.
		      \end{enumerate}
		      Par contre si $\bigcup\limits_{n \in \mathbb{N}} A_n$ est fini, alors $A_n = \emptyset$ à partir d'un certain rang.
		      Donc le cardinal de $\bigcup\limits_{n \in \mathbb{N}} A_n$ est fini et $\sum\limits_{n \in \mathbb{N}} \nu(A_n) = \nu(\bigcup\limits_{n \in \mathbb{N}} A_n)$ car ils
		      sont deux a deux disjoints.
	\end{enumerate}
\end{proof}

\begin{prop}[Propriétés élémentaires]\label{prop:mesure:elementaire}
	Nous avons 5 propriétés élémentaires:
	\begin{enumerate}
		\item Croissance. Si $A$ et $B$ mesurables, avec $A \subset B$, alors $\mu(B) = \mu(A) + \mu(B \setminus A)$ et $\mu(A) \leq \mu(B)$. De plus,
		      si $\mu(B)$ est finie, alors $\mu(B\setminus A) = \mu(B) - \mu(A)$.
		\item Crible. Si $A$ et $B$ mesurables
		      \[\mu(A\cup B) + \mu(A \cap B) = \mu(A) + \mu(B)\]
		\item Continuité croissante. Soit $A_n$ une suite croissante d'ensembles mesurables $A_n \subset A_{n+1}$,
		      alors
		      \[\mu(\bigcup\limits_{n \in \mathbb{N}} A_n) = \lim\limits_{n \to \infty} \mu(A_n)\]
		\item Continuité décroissante. Soit $A_n$ une suite décroissante d'ensembles mesurables $A_{n+1}
			      \subset A_n$, telle que $\mu(A_0) < +\infty$, alors:
		      \[\mu(\bigcap\limits_{n \in \mathbb{N}} A_n) = \lim\limits_{n \to \infty} \mu(A_n)\]
		\item Sous-additivité. Soit $(A_n)_{n \in \mathbb{N}}$ une suite d'ensembles mesurables, alors:
		      \[\mu(\bigcup\limits_{n \in \mathbb{N}} A_n) \leq \sum\limits_{n \in \mathbb{N}} \mu(A_n)\]
	\end{enumerate}
\end{prop}

\begin{proof}
	Nous allons démontrer les propriétés dans l'ordre.
	\begin{enumerate}
		\item Soit $B = A \cup (B \setminus A)$, alors $A$ et $B \setminus A$ sont disjoints. Donc
		      \[\mu(B) = \mu(A) + \mu(B \setminus A)\]
		      Donc $\mu(B) \geq \mu(A)$. Donc $\mu(B) < \infty$ alors $\mu(A)< \infty$ et $\mu(B \setminus A) = \mu(B) - \mu(A)$.
		\item \begin{eqnarray*}
			      \mu(A) + \mu (B) &=& \mu( A \setminus (A \cap B)) + \mu(A \cap B) + \mu(B \setminus (A \cap B)) + \mu(A \cap B) \\
			      \mu(A\cup B)&=& \mu(A \setminus (A \cap B)) + \mu(A \cap B) + \mu(B \setminus (A \cap B))
		      \end{eqnarray*}
		\item On pose $B_0 = A_0$ et $B_n = A_n \setminus A_{n-1}$ pour $n \geq 1$. On a que les $B_n$ sont disjoints et
		      $\bigcup\limits_{n \in \mathbb{N}} B_n = \bigcup\limits_{n \in \mathbb{N}} A_n$. Donc
		      \begin{eqnarray*}
			      \bigcup\limits_{n \in \mathbb{N}} \mu(A_n) &=& \bigcup\limits_{n \in \mathbb{N}} \mu(B_n) \\
			      &=& \sum\limits_{n \in \mathbb{N}} \mu(B_n) \quad \text{par } \sigma\text{-additivité}
		      \end{eqnarray*}
		      Si l'un des $A_n$ vérifie $\mu(A_n) = +\infty$, alors $\mu(\bigcup\limits_{n \in \mathbb{N}} A_n) = +\infty$
		      et l'égalité est donc vraie.
		      \\
		      Si tous les $A_n$ sont finis, on a :
		      \begin{eqnarray*}
			      \mu(B_n) &=& \mu(A_n) - \mu(A_{n-1}) \\
			      \mu(\bigcup\limits_{n \in \mathbb{N}} A_n) &=& \sum\limits_{n \in \mathbb{N}} \left( \mu(A_n) - \mu(A_{n-1}) \right)+ \mu(A_0) \\
			      &=& \lim\limits_{N \to \infty} \mu(A_0) + \sum\limits_{n = 1}^N \left( \mu(A_n) - \mu(A_{n-1}) \right) \\
			      &=& \lim\limits_{N \to \infty} \mu(A_N)
		      \end{eqnarray*}
		\item On définit $C_n = A_0 \setminus A_n $, alors les $C_n$ sont croissants et mesurables. Donc
		      \begin{eqnarray*}
			      \mu(\bigcup\limits_{n \in \mathbb{N}} C_n) &=& \lim\limits_{n \to \infty} \mu(C_n) \\
			      &=& \lim\limits_{n \to \infty} \mu(A_0\setminus A_n) \\
			      &=& \lim\limits_{n \to \infty} \mu(A_0) - \mu(A_n) \\
			      &=& \mu(A_0) - \lim\limits_{n \to \infty} \mu(A_n)
		      \end{eqnarray*}
		      Or $\bigcup\limits_{n \in \mathbb{N}} C_n = A_0 \setminus \bigcap\limits_{n \in \mathbb{N}} A_n$ et donc
		      $\mu(\bigcup\limits_{n \in \mathbb{N}} C_n) = \mu(A_0) - \mu(\bigcap\limits_{n \in \mathbb{N}} A_n)$.
		\item On pose $D_n = A_n \cap \left( \bigcup\limits_{i=0}^{n-1} A_i \right)^c$. Les $D_n$ sont disjoints et mesurables par
		      $\sigma$-additivité.
		      \begin{eqnarray*}
			      \mu(\bigcup\limits_{n \in \mathbb{N}} A_n) &=& \mu(\bigcup\limits_{n \in \mathbb{N}} D_n) \\
			      &=& \sum\limits_{n \in \mathbb{N}} \mu(D_n) \quad \text{ par } \sigma\text{-additivité}
		      \end{eqnarray*}
		      et $D_n \subset A_n$ donc $\mu(D_n) \leq \mu(A_n) \leq \sum\limits_{n \in \mathbb{N}} \mu(A_n)$.
	\end{enumerate}
\end{proof}

\begin{remarque}
	Dans le (4) de la proposition \ref{prop:mesure:elementaire}, l'hypothèse $\mu(A_0) < +\infty$ est nécessaire. En effet, soit
	$\nu$ la mesure de comptage. Soit
	\[ A_n = \{ n, n+1, n+2, \dots \} \]
	On a $\bigcap\limits_{n \in \mathbb{N}} A_n = \emptyset$ et $\nu(A_n) = +\infty$ pour tout $n \in \mathbb{N}$.
\end{remarque}

\begin{definition}[Vocabulaire des mesures]
	Soit $\mu$ une mesure sur $(E, \mathscr{A})$.
	\begin{itemize}
		\item $(E, \mathscr{A}, \mu)$ est un espace mesuré si $(E,\mathscr{A})$ est un espace mesurable et $\mu$ est une mesure sur $ (E, \mathscr{A} ) $.
		\item $\mu$ est dite finie si $\mu(E) < +\infty$.
		\item $\mu$ est une probabilité si $\mu(E) = 1$.
		\item $\mu$ est $\sigma$-finie si $E = \bigcup\limits_{n \in \mathbb{N}} A_n$ avec $A_n \in \mathscr{A}$ et $\mu(A_n) < +\infty$.
		\item On dit que $x$ est un atome de $\mu$ si $\mu(\{x\}) > 0$.
		\item $\mu$ est diffuse si elle n'a pas d'atomes.
	\end{itemize}
\end{definition}

\subsection{Fonctions mesurables}

\begin{definition}
	Soient $(E, \mathscr{A})$ et $(F, \mathscr{B})$ deux espaces mesurables. Une fonction $f: E \to F$ est dite mesurable si
	$f^{-1}(B) \in \mathscr{A}$ pour tout $B \in \mathscr{B}$.
	Dans le cas ou $E$ et $F$ sont munis de leur tribus boréliennes (si elles existent), on dit que $f$ est borélienne.
\end{definition}


\begin{prop}
	Si $f: (E, \mathscr{A}) \to (F, \mathscr{B})$ et $g: (F, \mathscr{B}) \to (G, \mathscr{C})$
	sont mesurables, alors $g \circ f : (E, \mathscr{A}) \to (G, \mathscr{C})$ est mesurable.
\end{prop}

\begin{proof}
	Soitn $C \in \mathscr{C}$. On a que $g^{-1}(C) \in \mathscr{B}$ car $g$ est mesurable et
	$f^{-1}(g^{-1}(C)) \in \mathscr{A}$ car $f$ est mesurable.
	Comme $f^{-1}(g^{-1}(C)) = (g \circ f)^{-1}(C)$, on a que $g \circ f$ est mesurable.
\end{proof}


\begin{prop}
	Pour que $f : (E, \mathscr{A}) \to (F, \mathscr{B})$ soit mesurable, il suffit qu'il existe $\mathscr{C} \subset \mathscr{B}$ telle que:
	\begin{enumerate}
		\item $f^{-1}(C) \in \mathscr{A}$ pour tout $C \in \mathscr{C}$.
		\item $\sigma(\mathscr{C}) = \mathscr{B}$
	\end{enumerate}
\end{prop}

\begin{proof}
	Posons $\mathscr{G} = \{ B \in \mathscr{B} \, | \, f^{-1}(B) \in \mathscr{A} \}$. On a que $\mathscr{C} \subset \mathscr{G}$.
	Montonrs qu $\mathscr{G}$ est une tribu sur $F$.
	\begin{enumerate}
		\item $f^{-1}(\emptyset) = \emptyset \in \mathscr{A}$ donc $\emptyset \in \mathscr{G}$.
		\item Soit $B \in \mathscr{G}$, alors $f^{-1}(B^c) = (f^{-1}(B))^c \in \mathscr{A}$ donc $B^c \in \mathscr{G}$.
		\item Soit $(B_n)_{n \in \mathbb{N}}$ une suite d'éléments de $\mathscr{G}$, alors
		      $f^{-1}(\bigcup\limits_{n \in \mathbb{N}} B_n) = \bigcup\limits_{n \in \mathbb{N}} f^{-1}(B_n) \in \mathscr{A}$ donc
		      $\bigcup\limits_{n \in \mathbb{N}} B_n \in \mathscr{G}$.
	\end{enumerate}
	Ainsi $\mathscr{G}$ est une tribu sur $F$ et $\mathscr{C} \subset \mathscr{G}$.
	On a donc
	\[ \sigma(\mathscr{B}) = \sigma(\mathscr{C}) \subset \sigma(\mathscr{G}) = \mathscr{G} \]
	Donc $\mathscr{B} \subset \mathscr{G} \subset \mathscr{B}$, donc $\mathscr{B} = \mathscr{G}$.
	On a donc que $f$ est mesurable.
\end{proof}


\begin{example}[Application]
	Si $F, \mathscr{B}) = (\mathbb{R}, \mathscr{B}(\mathbb{R}))$, on peut prendre $\mathscr{C} = \{ ]-\infty,\   t[ \, | \, t \in \mathbb{R} \}$. On a que
				$\sigma(\mathscr{C}) = \mathscr{B}(\mathbb{R})$. Et donc il suffit d'étudier la mesurabilité de $f^{-1}(]-\infty, t[)$.
\end{example}


\begin{remarque}
	Si $f : (E,\mathscr{B}(E)) \rightarrow (F,\mathscr{B}(F))$, alors $f$ continue $\implies f$ mesurable.
\end{remarque}


\begin{prop}
	La fonction $\mathscr{1}_A : (E,\mathscr{A}) \rightarrow (\mathbb{R},\mathscr{B}(\mathbb{R}))$ est messurable si et seulement si $A \in \mathscr{A}$.
\end{prop}

\begin{proof}
	$f^{-1}(]-\infty, t] = \emptyset si t < 0, A^c si t \in [0,1[ E si t \geq 1$
	donc $f$ messirable ssi $A^c\in \mathscr{A}$ ssi $A\in \mathscr{A}$
\end{proof}

\begin{prop}
	$f_1 : (E,\mathscr{A}) \rightarrow (F_1,\mathscr{B_1}$
	$f_2 : (E,\mathscr{A})) \rightarrow (F_2,\mathscr{B_2}))$,
	$g: (E \times E,\mathscr{A} \oplus \mathscr{A}) \rightarrow (F_1 \times F_2,\mathscr{B_1} \oplus \mathscr{B_2})$
	$x\mapsto (f_1(x), f_2(x))$
	alors $g$ est mesurable si $f_1$ et $f_2$ le sont
\end{prop}

\begin{proof}
	$\mathscr{B_1} \oplus \mathscr{B_2} = \sigma \left\{ B_1 \times B_2, B_1 \in \mathscr{B_1}, B_2 \in \mathscr{B_2} \right\}...$
	$g^{-1}(B_1 \times B_2) = {x \in E, f_1(x) \in B_1 et f_2(x) \in B_2  }$
	$f_1^{-1}(B_1) \cap f_2{-1}(B_2)$
	donc $g^{-1}(B_1 \times B_2) \in \mathscr{A}$
\end{proof}



\begin{prop}
	Si $f$ et $g (E,\mathscr{A}) \rightarrow (F,\mathscr{B}$
	sont mesurables, alors les foncntions suivantes sont mesurables:
	\begin{itemize}
		\item $f + g$
		\item $fg$
		\item $\inf(f,g)$
		\item $f^+ = \sup(f,0)$
		\item $f^- = \sup(-f,0)$
	\end{itemize}
\end{prop}

\begin{proof}
	$(x,y) \mapsto x + y$
	$(\mathbb{R}^2, \mathscr{B}(\mathbb{R}) \oplus \mathscr{B}(\mathbb{R})  \mathscr{B} \rightarrow (R,   \mathscr{B}(\mathbb{R})) $
	est continue donc mesurable. De meme pour le reste.
\end{proof}


\begin{definition}
	On note $\Rbar = \R \cup \{ -\infty, \infty \}$
	On note $\Rbarp = \R^+ \cup \{  \infty \}$
	Dans $\Rbarp\ \forall a \in \Rbarp: $
	\begin{itemize}
		\item $a + (+\infty) = +\infty$
		\item
		      $a * (+\infty) = \left\{ \begin{array}{cc}
                      0       & \text{si} \  a = 0 \\
                      +\infty & \text{sinon}
			      \end{array}\right.$
	\end{itemize}
\end{definition}

\begin{definition}
	Si $a_n$ est une suite dans $\Rbar$. On définit
	\begin{itemize}
		\item $\sup a_n =
			      \left\{ \begin{array}{cc}
				      +\infty         & \text{si} \ +\infty \ \text{est dans la suite}   \\
				      +\infty         & \text{si} \ \forall M > 0 \ \exists N,\, a_n > M \\
				      \sup a_n \in \R & \text{si} \ a_n\  \text{est majoré}
			      \end{array}\right.$

		\item De même pour le $\inf$
		\item $\limsup a_n = notation chiante = \lim\limits_{n \rightarrow \infty}\downarrow \sup\limits_{k \geq n } a_k \in \Rbar$
		\item $\liminf a_n = notation chiante = \lim\limits_{n \to \infty}\uparrow \inf\limits_{k \geq n } a_k \in \Rbar$
	\end{itemize}
\end{definition}


\begin{remarque}
	On travaillera avec $\bor(\Rbar)$, les boreliens de $\Rbar$ qui est
	$$\sigma (\left\{[-\infty, a], a \in \R\right\})$$
\end{remarque}


\begin{remarque}
	$\limsup a_n$ est la plus grande valeur d'adherance de la suite $a_n$
	$\liminf a_n$ est la plus petite valeur d'adherance de la suite $a_n$
\end{remarque}

\begin{prop}
	Si $f_n$ est une suite de fonction messurables
	$$f_n (E,\triA) \rightarrow (\mathbb{R},\bor(\R))$$
	alors les fonctions suivantes sont mesurables:
	\begin{itemize}
		\item $\sup f_n x\mapsto f_n(x) $
		\item $\inf f_n$
		\item $\limsup f_n$
		\item $\liminf f_n$
	\end{itemize}
	En particulier si $f_n$ converge simplement dans $\Rbar$ alors
	$\lim f_n = \limsup f_n$ est mesurable. De plus $\{ x \in E, f_n(a) \text{converge}\}$ est mesurable.
\end{prop}

\begin{proof}
	$f(x) = \inf f_n (x)$
	\begin{eqnarray*}
		f^{-1}([-\infty, a[) &=& \{x \in E, \inf f_n < a\} \\
		&=& \{x \in E, \exists N \in \mathbb{N}, f_N (x) < a \}\\
		&=& \bigcup\limits_{n\in \mathbb{N}}f_N^{-1}([-\infty, a[))\\
		&=& \in \mathscr{A}
	\end{eqnarray*}
	de meme pour $\sup$

	\begin{eqnarray*}
		\liminf f_n &=& \lim\limits_{n \to \infty}\inf\limits_{k \geq n } a_k\\
		&=& \sup _{n\in N} \inf_{k \geq k} f_n\\
		&=& \in \mathscr{A}
	\end{eqnarray*}

	donc $\liminf$ et $\limsup$ sont mesurables.

	\begin{eqnarray*}
		\{x \in E, f_n \  \text{converge} \}  &=& (\liminf f_n - \limsup f_n)^{-1}(\{0\})\\
	\end{eqnarray*}
	et $\{ 0 \} \in \bor(\R)$ donc $\left\{ f_n \in E, f_n(x) \ \text{converge} \ \right\} \in \mathscr{A}$.
\end{proof}









\section{Intégration par rapport à une mesure}

\subsection{Intégrale d'une fonction mesurable positive}


On travaille avec $(E,\triA,\mu)$ un espace mesuré.

\begin{definition}[Fonction étagée]
	Soit $f: (E, \triA) \to (\R, \bor(\R))$ une fonction mesurable. On dit que $f$ est une fonction étagée
	si elle prend un nombre fini de valeurs.
	Si $\alpha_1, \dots, \alpha_n$ sont les valeurs prises par $f$, on note
	$A_i = f^{-1}(\{\alpha_i\})$ et on a
	$$ f = \sum_{i=1}^n \alpha_i \1_{A_i} $$
	on dit que c'est l'écriture canonique de $f$.
\end{definition}

\begin{remarque}
	$\1_{\Q}$ n'est pas étagée.
\end{remarque}


\begin{definition}[Intégrale d'une fonction positive]
	Si $f = \sum\limits_{i=1}^n \alpha_i \1_{A_i}$ est une fonction étagée positive, on définit
	$$ \int f d\mu = \sum_{i=1}^n \alpha_i \mu(A_i) \in [0, +\infty] $$
	Avec la convention $0 \cdot \infty = 0$.
\end{definition}


\begin{remarque}
	La valeur de $\int f d\mu$ ne dépend pas de l'écriture canonique de $f$, i.e. si
	$$ f = \sum_{i=1}^n \alpha_i \1_{A_i} = \sum_{j=1}^m \beta_j \1_{B_j} $$
	alors
	$$ \sum_{i=1}^n \alpha_i \mu(A_i) = \sum_{j=1}^m \beta_j \mu(B_j) $$
\end{remarque} %TODO: Add proof ?

\begin{prop}[Linéarité et croissance]
	Soient $f,g$ deux fonctions étagées positives.
	\begin{enumerate}
		\item $  \forall a, b \in \R, \int (af + bg) d\mu = a \int f d\mu + b \int g d\mu$
		\item Si $f \leq g$ alors $\int f d\mu \leq \int g d\mu$
	\end{enumerate}
\end{prop}

\begin{proof}
	\begin{itemize}
		\item $ f = \sum_{i=1}^n \alpha_i \1_{A_i} $ et $ g = \sum_{j=1}^m \alpha'_j \1_{A'_j} $
		      On introduit $B_{ij} = A_i \cap A'_j$, $\beta_{ik} = \alpha_i$ et $\beta'_{ik} = \alpha'_k$.
		      On a alors
		      $$ f = \sum_{i=1}^n \sum_{j=1}^m \beta_{ij} \1_{B_{ij}}, \ \int f \mu = \sum_{i,j} \beta_{ij} \mu(B_{ij}) $$
		      et
		      $$ g = \sum_{i=1}^n \sum_{j=1}^m \beta'_{ij} \1_{B_{ij}}, \ \int g \mu = \sum_{i,j} \beta'_{ij} \mu(B_{ij} $$
		      On a alors
		      \begin{eqnarray*}
			      \int (af + bg) d\mu & = & \int \left( \sum_{i=1}^n \sum_{j=1}^m (a\beta_{ij} + b\beta'_{ij}) \1_{B_{ij}} \right) d\mu \\
			      & = & \sum_{i,j} (a\beta_{ij} + b\beta'_{ij}) \mu(B_{ij}) \\
			      & = & a \sum_{i,j} \beta_{ij} \mu(B_{ij}) + b \sum_{i,j} \beta'_{ij} \mu(B_{ij}) \\
			      & = & a \int f d\mu + b \int g d\mu
		      \end{eqnarray*}
		\item Si $f \leq g$ alors $g - f $ est étagée positive.
		      \begin{eqnarray*}
			      \int g d\mu & = & \int (f + (g-f)) d\mu \\
			      & = & \int f d\mu + \underbrace{\int (g-f)}_{\geq 0} d\mu
		      \end{eqnarray*}
		      Donc $\int g d\mu \geq \int f d\mu$.
	\end{itemize}
\end{proof}

Soit  $\mathcal{E}_+$ l'ensemble des fonctions étagées positives.


\begin{definition}[Intégrale d'une fonction mesurable positive]
	Soit $f: (E, \triA) \to (\Rbarp, \bor(\Rbarp))$ une fonction mesurable positive.
	On définit
	$$ \int f d\mu = \sup\limits_{\substack{g \in \mathcal{E}_+ \\ g \leq f}} \int g d\mu $$
\end{definition}

\begin{remarque}
	L'ensemble en question n'est pas vide car $0 \in \mathcal{E}_+$ et $0 \leq f$. Et donc
	$$ \int f d\mu \geq \int 0 d\mu = 0 $$
\end{remarque}

\begin{remarque}
	Si $g$ est étagée positive,
	$$\sup\limits_{\substack{h \in \mathcal{E}_+ \\ h \leq g}} \int h d\mu \leq \int g d\mu \text{ intégrale définie précédemment}$$
	et de plus $h \in \mathcal{E}_+$ et $h \leq g$ et donc
	$$ \int g d\mu \leq \sup\limits_{\substack{h \in \mathcal{E}_+ \\ h \leq g}} \int h d\mu $$
	et donc on a l'égalité entre les deux définitions.
\end{remarque}

\begin{remarque}
	On notera:
	\begin{itemize}
		\item $\int f d\mu$
		\item $\int f(x) d\mu(x)$
		\item $\int f(x) \mu(dx)$
		\item $\int\limits_E f(x) \mu(dx)$
		\item $\int\limits_{x\in E} f(x) d\mu(x)$
		\item Et même $\mu(f)$
	\end{itemize}
\end{remarque}


\begin{prop}[Croissance et séparation] \label{prop:croi-et-sep}
	\begin{itemize}
		\item Si $f,g$ mesurables positives $f \leq g$ alors $\int f d\mu \leq \int g d\mu$
		\item Si $\mu(\left\{x : f(x) > 0\right\}) = 0$ alors $\int f d\mu = 0$
	\end{itemize}
\end{prop}

\begin{proof}
	\begin{itemize}
		\item $\left\{ h \in \mathcal{E}_+ : h \leq f \right\} \subset \left\{ h \in \mathcal{E}_+ : h \leq g \right\}$ donc
		      $$ \int f d\mu \leq \int g d\mu $$
		\item Soit $h$ étagée positive telle que $h \leq f$.
		      $h^{-1}(\overline{\R^{+*}}) \subset f^{-1}(\overline{\R^{+*}})$ et donc
		      $$ \mu(h^{-1}(\overline{\R^{+*}})) \leq \mu(f^{-1}(\overline{\R^{+*}})) = 0 $$
		      Donc $h = 0 \cdot \1_{A_i} + \sum_{j=1}^n \alpha_j \1_{A_j}$
		      Alors $A_i \subset h^{-1}(\{0\})$ et donc $\mu(A_i) = 0$.
		      Donc $\int h d\mu = 0$.
		      et donc $\int f d\mu = 0$.
	\end{itemize}
\end{proof}


\begin{theorem}[Convergence monotone] \label{thm:convergence_monotone}
	Soit $(f_n)_{n \in \N}$ une suite croissante de fonctions mesurables positives.\\
	On note $f = \lim\limits_{n \to \infty}\uparrow f_n$.
	Alors on a
	$$ \int f d\mu = \lim\limits_{n \to \infty} \int f_n d\mu $$
\end{theorem}

\begin{remarque}
	$f_n$ suite croissante de fonctions (et pas suite de fonctions croissantes ...).
	$$ \forall n \in \N, f_n \leq f_{n+1} $$
\end{remarque}


\begin{proof}
	\begin{itemize}
		\item $$f_n (x) \leq f_{n+1}(x)$$ et donc $\int f_n d\mu \leq \int f_{n+1} d\mu$.
		      et finalement $$\lim\limits_{n \to \infty} \int f_n d\mu \leq \int f d\mu$$.
		\item Soit $h \in \mathcal{E}_+$ telle que $h \leq f$, montrons que $\int h d\mu \leq \lim\limits_{n \to \infty} \int f_n d\mu$.\\
		      Soit $a \in [0,1[$
		      $$ E_n = \left\{ x \in E :  ah(x) \leq f_n(x) \right\} $$
		      $E_n$ est mesurable car $E_n = (ah-f_n)^{-1}(\R^-)$.\\
		      On a $f_n \to f$ et donc $a < 1$ et donc $ah < f$ et donc pour un
		      $n$ assez grand, $f_n \geq ah$.\\
		      Donc $E = \bigcup_{n \in \N} E_n$.\\
		      Or $ f_n \geq ah \1_{E_n} $ et donc
		      \begin{eqnarray*}
			      \int f_n d\mu \geq \int ah \1_{E_n} d\mu &=& \sum_{i=1}^k a \alpha_i \mu (A_i \cap E_n) \\
			      \text{car} \ ah\1_{E_n} 0 \sum_{i=1}^k a \alpha_i \1_{A_i \cap E_n} \  \text{est étagée positive}& \\
			      &=& a \sum_{i=1}^k \alpha_i \mu(A_i \cap E_n) \\
		      \end{eqnarray*}
		      or $E_n$ est une suite croissante d'ensembles avec $\bigcup_{n \in \N} E_n = E$ et donc
		      \begin{eqnarray*}
			      \lim\limits_{n \to \infty} \mu(A_i \cap E_n) &=& \mu(A_i)\\
			      \lim \limits_{n \to \infty} \int f_n d\mu &\geq& a \int h d\mu
		      \end{eqnarray*}
		      Comme c'est vrai pour tout $a \in [0,1[$, on a
		      \begin{eqnarray*}
			      \lim \limits_{n \to \infty} \int f_n d\mu &\geq&  \int h d\mu\\
			      \lim \limits_{n \to \infty} \int f_n d\mu &\geq& \sup\limits_{\substack{h \in \mathcal{E}_+ \\ h \leq f}} \int h d\mu = \int f d\mu
		      \end{eqnarray*}
	\end{itemize}
\end{proof}


\begin{example}[Contre-exemple à la convergence monotone pour une suite non croissante]
	$f_n = \1_{[n, \infty[}$. $f_n \to 0, \ \forall x $.\\
	Et on a  $\int f_n d\lambda = \infty$ or $\int 0 d\lambda = 0$.
\end{example}

\begin{prop}
	Soit $f$ mesurable positive.\\
	Alors il existe $f_n$ suite croissante de fonctions étagées positives telles
	$$\forall x \in E, f_n(x) \to f(x)$$
\end{prop}


\begin{proof} %TODO: Add graphics
	$$n \geq 1, i \in \left\{0, \dots, 2^n - 1\right\}$$
	\begin{eqnarray*}
		A_{n} &=& \left\{ x \in E : f(x) \geq n \right\} \in \triA \\
		B_{n,i} &=& \left\{ x \in E : \frac{i}{2^n} \leq f(x) < \frac{i+1}{2^n} \right\} \in \triA \\
		f_n & = & n \1_{A_n} + \sum_{i=0}^{2^n - 1} \frac{i}{2^n} \1_{B_{n,i}}
	\end{eqnarray*}

	En général, $A_n$ et $B_{n,i}$ ne sont pas des intervalles.

	Par construction $f_n \leq fn$ et $f_n$ est une suite croissante, c'est pour cette raison qu'on a subdivisé avec $2^n$ intervalles, et pas $n$ intervalles.\\
	A-t-on $f_n \to f$ ?

	\begin{eqnarray*}
		f_n(x) =  2^{-1} i_n(x)\underbrace{\1_{B_n,i_n(x)}(x)}_{=1} &\text{ Si } \1_n(x) \text{ est tel que } x \in B_{n,i_n(x)} & \\
		& 2^{-n}i_n(x) \leq f_n(x) \leq 2^{-n}(i_n(x) + 1) &\\
		& \text{donc } 2_{-n}i_n(x) \to f(x) &
	\end{eqnarray*}
	Et donc $f_n(x) \to f(x)$
\end{proof}

\begin{remarque}
	Les fonctions données par la proposition précédente vérifient les hypothèses du théorème \ref{thm:convergence_monotone}, donc
	$$ \int f_n d\mu \to \int f d\mu $$
\end{remarque}


\begin{prop}[Linéarité]
	Soient $f,g$ mesurables positives et $a, b\geq 0$.
	$$ \int (af + bg) d\mu = a\int f d\mu + b\int g d\mu $$
\end{prop}


\begin{proof}
	$$f_n \in \mathcal{E}_+, f_n \uparrow \to f$$
	$$ g_n \in \mathcal{E}_+, g_n \uparrow \to g$$
	$$ \underbrace{\int a f_n + b g_n \,d\mu}_{\text{Par } \ref{thm:convergence_monotone} \to \int a f + b g\,d\mu}
		= a \underbrace{\int f_n d\mu}_{ \text{Par } \ref{thm:convergence_monotone} \to \int f d\mu}
		+ b \underbrace{\int g_n d\mu}_{ \text{Par } \ref{thm:convergence_monotone} \to \int g d\mu} $$
\end{proof}



\begin{prop} [TCM pour les suites]
	Soit $f_n$ une suite de fonctions mesurables positives.\\
	Alors,
	$$ \int \sum\limits_n^{\infty} f_n d\mu = \sum\limits_n^{\infty} \int f_n d\mu $$
\end{prop}

\begin{proof}
	On regarde $S_n(x) = \sum\limits_{k=1}^n f_k(x)$ est une suite de fonctions mesurables positives.
	Et $S_{n+1} - S_n = f_{n+1} > 0 $ et donc $S_n$ est une suite croissante de fonctions mesurables positives.
	D'après le \ref{thm:convergence_monotone}, on a
	\begin{eqnarray*}
		\int \lim\limits_{n \to \infty} S_n (x)d\mu (x)&=& \lim\limits_{n \to \infty} \int S_n(x) d\mu(x) \\
		\int \sum\limits_{n = 1}^{\infty} S_n (x)d\mu (x)&=& \lim\limits_{n \to \infty} \int \sum\limits_{k = 1}^n S_n(x) d\mu(x) \\
		&=& \lim\limits_{n \to \infty} \sum\limits_{k = 1}^n\int  f_k(x) d\mu(x) \\
		&=& \sum\limits_{n = 1}^{\infty} \int  f_k(x) d\mu(x) \\
	\end{eqnarray*}
\end{proof}


\begin{definition}[$\mu$-presque partout]
	Soit $P$ une propriété sur $x \in E$ ($P(x) \in \{Vrai, Faux\}$)\\
	On dit que $P$ est vraie $\mu$-presque partout ($\mu-pp$) si
	$$ \left\{x \in E : P(x)  = Faux \right\} \in \triA \text{ et } \mu\left(\left\{x \in E : P(x)  = Faux \right\}\right) = 0 $$
	ou s'il existe $B \in \triA$ tel que $\left\{ x \in E : P(x) = Faux \right\} \subset B$ et $\mu(B) = 0$
\end{definition}

\begin{definition}[Mesure à densité]
	Soit $f$ une fonction mesurable positive. \\
	On définit
	\begin{eqnarray*}
		\nu : \triA & \to &\Rbarp                  \\
		A     & \mapsto & \int \1_A f d\mu \\
		&& = \int_A fd\mu
	\end{eqnarray*}
	Alors $\nu$ est une mesure sur $(E, \triA)$ et on l'appelle mesure à densité par rapport à $\mu$.
\end{definition}

\begin{proof}
	\begin{itemize}
		\item L'ensemble de départ est mesurable.
		\item $\nu(\emptyset) = \int \1_{\emptyset} f d\mu = \int 0 d \mu = 0$
		\item $A_n$ suite d'ensembles disjoints on regarde
		      \begin{eqnarray*}
			      \nu\left(\bigcup A_n\right) &=& \int \1_{\bigcup A_n} f d\mu \\
			      &=& \int \left( \sum_{k=1}^{\infty} \1_{A_k} \right) f d\mu \\
			      &=& \int \sum_{k=1}^{\infty} \1_{A_k} f d\mu \\
			      &=& \sum_{k=1}^{\infty} \int \1_{A_k} f d\mu
		      \end{eqnarray*}
		      d'après le TCM pour les séries, car $\forall k \geq 1, \ \1_{A_k}f$ est mesurable positive.
	\end{itemize}
\end{proof}

\begin{remarque} \label{rem:transfert}
	On veut veut, pour $g$ mesurable positive, montrer que
	$$ \int g d\nu = \int g f d\mu $$
\end{remarque}

\begin{proof}
	\begin{itemize}
		\item Vrai pour $\1_A$ ?
		\item Vrai pour $g$ étagée positive ?
		\item Vrai pour  $g$ étagée positive ?
	\end{itemize}

	\begin{itemize}
		\item Si $g = \1_A$, avec $A \in \mathscr{F}$, alors
		      \begin{eqnarray*}
			      \int g d \nu &=& \int \1_A d\nu  = \nu(A) \\
			      &=& \int \1_A f d\mu = \int g f d\mu
		      \end{eqnarray*}
		\item Si $g = \sum_{i=1}^n \alpha_i \1_{A_i}$, alors
		      \begin{eqnarray*}
			      \int g d \nu = \sum_{i=1}^n \alpha_i \1_{A_i} \nu(A_i)  &=& \sum_{i=1}^n \alpha_i \int f \1_{A_i} d \mu \\
			      &=&\int \left(\sum_{i=1}^n \alpha_i \1_{A_i}\right) f d \mu \\
			      &=& \int g f d \mu
		      \end{eqnarray*}
		\item Si $g$ est étagée positive, alors
		      On prend $g_n \uparrow g$ une suite de fonctions étagées positives.
		      Alors \begin{eqnarray*}
			      \int g d \nu &=& \lim\limits_{n \to \infty} \int g_n d \nu  \text{ par le théorème \ref{thm:convergence_monotone}}\\
			      &=& \lim\limits_{n \to \infty} \int g_n f d \mu
		      \end{eqnarray*}
		      On remarque que $g_n f$ est une suite croissante de fonctions mesurables positives et donc
		      on peut appliquer le théorème \ref{thm:convergence_monotone} et donc
		      \begin{eqnarray*}
			      \lim\limits_{n \to \infty} \int g_n f d \mu & = & \int  \lim\limits_{n \to \infty} g_n f d \mu \\
			      &=& \int g f d \mu
		      \end{eqnarray*}
	\end{itemize}
\end{proof}

\begin{example}[la normale sur $\R$]
	(Même proba sur  $(\R, \bor(\R))$, qui est la loi de $X ~ \mathscr{N}(0,1)$)
	\begin{eqnarray*}
		\nu(B) = \Pro_\lambda(B) &= &\Pro(X \in B)  \\
		&=& \int \1_B(x) \frac{1}{\sqrt{2\pi}} e^{-\frac{X^2}{2}} d \lambda(x)\\
		&=&\int \1_B f d\lambda
	\end{eqnarray*}

	Donc $\nu$ a la même densité que $f$ par rapport à la mesure de Lebesgue:
	$$f (x) = \frac{1}{\sqrt{2\pi}} e^{-\frac{X^2}{2}} $$
\end{example}




\begin{prop}[Inégalité de Markov]
	Soit $f$ une fonction mesurable positive.
	\begin{itemize}
		\item Pour tout $a > 0$ on a
		      $$ \mu\left(\left\{ x \in E : f(x) \geq a \right\}\right) \leq \frac{1}{a} \int f d\mu $$
		\item Si $\int f d\mu < \infty$ alors $f < \infty \ \mu-pp$
		\item $\int f d\mu = 0$ si et seulement si $f = 0 \ \mu-pp$
		\item $f=g \ \mu-pp \implies \int f d\mu = \int g d\mu$
	\end{itemize}
\end{prop}

\begin{proof}
	\begin{itemize}
		\item $f \geq a \1_{f \geq a}$
		      $$\int f d \mu \geq a \mu \left( \left\{ x \in E : f(x) \geq a \right\} \right) $$
		\item $A_k = \left\{ x \in E : f(x) \geq a \right\}$
		      On a $A_{\infty} = \bigcup_{k=0}^{\infty} A_k$ intersection décroissante.
		      \begin{eqnarray*}
			      \mu(A_1) &=& 1 \int f d \mu \text{ d'après  le (1)}\\
			      &<& \infty \text{ par hypothèse}
		      \end{eqnarray*}
		      donc \begin{eqnarray*}
			      \mu {A_{\infty}} &=& \lim\limits_{n \to \infty} \mu(A_n) \\
			      \mu(A_{n} &\leq& \frac{1}{n} \int f d u
		      \end{eqnarray*}

		      Donc $\mu (An) \to 0$ et donc $\mu(A_{\infty}) = 0$.
		\item $f= 0 \ \mu-pp \implies \int f d\mu = 0$  par  \ref{prop:croi-et-sep}.\\
		      $\leftarrow$ Soit  $f$ mesurable positive telle que $\int f d\mu = 0$.\\
		      $B_n = \left\{ x \in E : f(x) \geq \frac{1}{n} \right\}$\\
		      $\bigcup B_n$ est une suite croissante. \\
		      $\bigcup\limits_{n \in \N} = \left\{ x \in E : f(x) > 0 \right\}$\\

		      \begin{eqnarray*} %TODO: Add prof
		      \end{eqnarray*}

	\end{itemize}
\end{proof}


\begin{theorem}[Lemme de Fatou]
	Soit $f_n$ est une suite de fonctions mesurables positives.\\
	Alors
	$$ \int \liminf f_n d\mu \leq \liminf \int f_n d\mu $$
\end{theorem}

\begin{proof}
	$\liminf f_n = \lim\limits_{n \to \infty} \uparrow \inf_{k \geq n} f_k$\\
	On regarde $g_n = \inf_{k \geq n} f_k$ est une suite croissante de fonctions mesurables positives.\\
	Donc d'après le théorème \ref{thm:convergence_monotone}, on a
	$$ \lim\limits_{n \to \infty} \int g_n d\mu = \int \lim\limits_{n \to \infty} g_n d\mu = \int \liminf f_n d\mu $$
	Si $p \geq n$, alors $g_n \leq f_p$ et donc
	$$ \int g_n d\mu \leq \int f_p d\mu $$
	et donc
	$$ \forall p \geq n, \int f_p d\mu \geq \int g_p d\mu $$
	Et donc $\forall n \in \N$:
	$$ \inf_{p \geq n} \int f_p d\mu \geq \int g_n d\mu $$
	Et en passant à la limite, on
	$$ \liminf \int f_n d\mu \geq \int \liminf f_n d\mu $$
\end{proof}


\begin{example} $([0,1], \bor([0,1]),\lambda)$\\
	\begin{eqnarray*}
		f_{2k} &=& \1_{[\frac{1}{2}, 1]}\\
		f_{2k+1} &=& \1_{[0,\frac{1}{2}]}\\
		\liminf f_n &=& \1_{\{\frac{1}{2}\}}\\
		\int f_n d \lambda &=& \frac{1}{2}
	\end{eqnarray*}
	$$ \frac{1}{2} = \liminf \int f_n d \lambda \geq \int \liminf f_n d \lambda = 0 $$
\end{example}


\begin{theorem}[Égalité Reinmann-Lebesgue sur un segment, fonctions positives]
	Soit $f$ mesurable positive:
	$ f ([a,b], \bor([a,b])) \to \R, \bor(\R)$.\\
	On suppose $f$ Reinmann intégrable sur $[a,b]$.\\
	$$ \int_a^b f(x) dx = \int_{[a,b]} f d\lambda $$
\end{theorem}

\begin{proof} %TODO: Add proof





\end{proof}

\subsection{Exemples de calculs d'intégrales}

\begin{example}
	La mesure de Dirac
	$$ \int f d \delta_x = f(x) $$
\end{example}

\begin{proof}
	\begin{itemize}
		\item $f$ indicatrice
		\item $f$ étagée
		\item $f$ mesurable positive
	\end{itemize}
	En utilisant ce schéma, on peut montrer l'égalité, comme pour la remarque \ref{rem:transfert}. %TODO: Add details
\end{proof}

\begin{example} %TODO Add \nu_c definition
	Mesure de comptage:
	$$ \int f d \nu_c = \sum_{n \in \N} f(n)$$
\end{example}



\subsection{Fonctions intégrables}

On travaille sur $(E, \triA, \mu)$.

\begin{definition}
	Soit $f$ mesurable, $f : (E, \triA) \to (\R, \bor(\R))$.\\
	On dit que $f$ est intégrable par rapport à $\mu$ si :
	$$\int f^+ d\mu < +\infty \text{ et } \int f^- d\mu < +\infty$$
	où $f^+ = \max(f,0)$ et $f^- = -\min(f,0)$.
	On définit alors
	$$ \int f d\mu = \int f^+ d\mu - \int f^- d\mu $$
\end{definition}

\begin{remarque}
	De façon équivalente, on dit que $f$ est intégrable si
	$$ \int |f| d\mu < +\infty$$
\end{remarque}

\begin{remarque}
	$$ f = f^+ - f^- $$
	$$ |f| = f^+ + f^- $$
\end{remarque}

\begin{remarque}
	$f^+$ et $f^-$ sont mesurables positives, donc $\int f^+d\mu$ et $\int f^- d\mu$ sont définies.
\end{remarque}

\begin{remarque}
	Si $f \geq 0$ on retrouve bien la définition originale.
\end{remarque}

\begin{remarque}
	Par contre, si $\int f^+ d\mu = +\infty$ on dit que $f$ n'est pas intégrable, même si elle est positive.
\end{remarque}


\begin{definition}
	On note $\Li$ l'espace des fonctions intégrables.
\end{definition}

\begin{prop}[Propriétés de l'intégrale]
	\begin{itemize}
		\item Inégalité triangulaire. $\left| \int f d\mu \right| \leq \int |f| d \mu$.
		\item Linéarité: $\Li$ est un espace vectoriel.
		\item Croissance: si $f,g \in \Li$ et $f\leq g$ alors $\int f d\mu \leq \int g d\mu$
		\item Si $f,g \in \Li$ $f = g \mupp$ alors $\int f d\mu = \int g d\mu$.
	\end{itemize}
\end{prop}

\begin{proof}
	\begin{itemize}
		\item  \begin{eqnarray*}
			      |\int f d\mu| &=& \left|\int f^+ d\mu - \int f^- d\mu \right| \\
			      &\leq& \left|\int f^+ d\mu \right| + \left|\int f^- d\mu \right| \\
			      &\leq& \int f^+ d\mu + \int f^- d\mu  \\
			      \text{linéarité fonctions positives } &\leq& \int f^+ f^- d\mu  \\
			      &\leq& \int |f| d\mu
		      \end{eqnarray*}
		\item %TODO
		\item $f \leq g$, $g = f + (g-f)$
		      $$\int g d\mu = \int f d\mu + \int (g-f) d \mu$$
		      Or $(g-f)^- = 0$, donc $\int (g-f) d\mu > 0$.
		      Donc $\int f d\mu \leq \int g d\mu$.
		\item Si $f = g$ $\mupp$, alors $f^+ = g^+$ $\mupp$ et $f^- = g^-$ $\mu$pp.\\
		      Donc $\int f d\mu = \int f^+ \mu + \int f^-  d\mu = \int g^+ \mu + \int g^-  d\mu = \int g d\mu$
	\end{itemize}
\end{proof}

\begin{definition}[Intégrale des fonctions complexes]
	$$ F: (E, \triA) \to (\C, \bor(\C)) \text{ mesurable.}$$
	Ce qui est équivalent à dire que $Re(f)$ et $Im(f)$ sont mesurables.\\
	On dit que $f$ est intégrable et on note
	$$ f\in \LiC$$
	si $$\int |f| d\mu < +\infty$$
	On pose
	$$ \int f d\mu = \int Re(f) d\mu + \int Im(f) d\mu $$
\end{definition}


\begin{prop}
	\begin{itemize}
		\item Inégalité triangulaire. $\left| \int f d\mu \right| \leq \int |f| d \mu$.
		\item Linéarité: $\LiC$ est un espace vectoriel.
		\item Si $f,g \in \Li$ $f = g \mupp$ alors $\int f d\mu = \int g d\mu$.
	\end{itemize}
\end{prop}


\begin{proof}
	\begin{itemize}
		\item $$\forall b \in \C, \ |b| = \sup_{a_1, a_2 \in \R \\ a_1^2 + a_2^2 = 1 } a_1 Re(b) + a_2 Im(b)$$
		      % TODO
		\item
	\end{itemize}
\end{proof}



\begin{theorem}[Convergence dominée]
	Soit $f_n$ une suite de fonctions dans $\Li$ avec:
	\begin{itemize}
		\item Il existe $f$ mesurable à valeurs dans $\R$ telle que $f_n \to f, \ \mupp$.
		\item Il existe $g : (E,A) \to (\R^+, \bor(\R^+))$ mesurable positive avec $\int g d\mu < +\infty$ telle que
		      $$\forall n \in\N,\ |f_n(x)| \leq g(x)\ \mupp$$
	\end{itemize}

	Alors $f \in \Li$ et
	$$\lim_{n\to \infty} \int f_n d\mu= \int f d \mu$$
	et de plus
	$$ \int |f_n - f | d \mu \to 0 $$
\end{theorem}


\begin{example}[Contre-exemple sans domination]
	$$f_n =\frac{1}{n} \1_{[0,n[} \to 0 \ \mupp$$
	$$ \forall n, \ \int f_n d\mu = 1 \to 1 \neq 0 = \int 0 d\mu$$
\end{example}

\subsection{Intégrales dépendant d'un paramètre}

\begin{theorem}[Continuité sous le signe intégrale]
	Soit $f : U \times E \to \R$ où $(U,D)$ est un espace métrique.\\
	On suppose:
	\begin{itemize}
		\item $\forall u \in U,\ f (u, \cdot) : (E, \triA) \to (\R, \bor(\R))$ mesurable.
		\item $\mu-pp$ (en $x$) $f(\cdot, x) : U  \to \R$ est continue en $u_0 \in U$.
		\item Il existe $g$ intégrable telle que
		      $$\forall u \in U, \ \left| f(u,x) \right| \leq g(x) \mu-pp \text{ en } x$$
	\end{itemize}
	Alors la fonction
	$$F (u) = \int f(u,x) d\mu(x)$$
	est bien définie pour tout $u \in U$ et elle est continue en $u_0 \in U$.
\end{theorem}


\begin{proof}
	Il faut montrer que si $u_n$ et une suite avec $u_n \to u_0, \ n > 0$, alors $F(u_n) \to F(u_0)$.\\
	F est bien définie car $\forall u \in U, f(u,\cdot)$ mesurable et $|f(u,\cdot)| \leq g(\cdot)$ qui est intégrable.\\

	Posons $f_n(x) = f(u_n,x)$ et $f_n(x) \to f(u_0,x)$.\\
	$$|f_n(x)|= |f(u_n, x)| \leq g(x) \mu-pp$$
	D'après le TCD , $\underbrace{\int f_n(x)d\mu(x)}_{=F(u_n)} \to  \underbrace{\int f(u_0, x)d\mu(x)}_{= F(u_0)}$
\end{proof}

\begin{example}
	Soit $\mu$ une mesure diffuse sur $(\R, \bor(\R))$ et $\phi \in \Li(\R, \bor(\R), \mu)$.
	On définit \begin{eqnarray*}
		F(u) &=&  \int_{]-\infty, u]} \phi(x)d\mu(x)\\
		& = & \int \phi \1_{]-\infty, u]}(x)d\mu(x)
	\end{eqnarray*}
	alors $F$ est continue.

	\begin{proof}
		On pose $f(u,x) = \phi(x)\1_{]-\infty, u]}(x)$
		Il suffit de vérifier que $f(\cdot, x)$ est continue et qu'elle est dominée. Le reste est trivial: \\
		\begin{itemize}
			\item $u \mapsto f(u,x) = \left\{
				      \begin{array}{l}
					      \phi(x) \text{ si } u \geq x \\
					      0 \text{ sinon}
				      \end{array}
				      \right.$ \\
			      est continue en $\R\setminus\{x\}$. Comme $\mu$ est diffuse, $\mu \{x\} = 0$ et donc $f(\cdot, x)$ est continue $\mu-pp$.
			\item $|f(u,x)| = |\phi(x)\1_{]-\infty, u]}(x)| \leq |\phi(x)|$ qui est intégrable.
		\end{itemize}
	\end{proof}
\end{example}


\begin{theorem}[Dérivation sous le signe intégrale]
	On suppose que $U = I$ est un intervalle ouvert de $\R$ et $u_0 \in I$.\\
	$$ f: I\times E \to \R$$
	\begin{itemize}
		\item $\forall u \in I, \ f(u, \cdot) \in \Li_R(E,\triA, \mu)$
		\item $\mu-pp,\  f(\cdot, x)$ est dérivable en $u_0 \in I$ de dérivée $\diffp{f}{u} (u_0, x)$.
		\item Il existe $g\in \Li$ telle que
		      $$ \forall u\in I |f(u,x) - f(u_0,x)| \leq g(x)|u-u_0| \mu-pp$$
	\end{itemize}

	Alors $F(u) = \int f(u,x)d\mu$ est dérivable au point $u_0$ et sa dérivée est $F'(u_0) = \int \diffp{f}{u}(u_0,x)d\mu(x)$
\end{theorem}

\begin{remarque}
	La fonction $\diffp{f}{u}(u_0,\cdot)$ n'est définie que $\mu-pp$. Il suffit de
	la prolonger n'importe comment et cela suffit a définir $\int \diffp{f}{u}(u_0,x)d\mu(x)$.
	Comme le prolongement se fait sur un ensemble de mesure nulle, cela ne change pas la valeur de l'intégrale,
	c'est pour cela que l'on peut donner une liberté absolue pour le prolongement, par exemple en lui donnant la valeur 0.
\end{remarque}

\begin{proof}
	Soit $u_n \to u_0, \ n > 0, \ u_n \neq u_0$. \\
	On regarde
	\begin{eqnarray*}
		\frac{F(u_n)-F(u_0)}{u_n-u_0} &=& \frac{1}{u_n-u_0} \int f(u_n,x) -(u_0,x)df\mu(x) \\
		&=& \int \frac{f(u_n,x) - f(u_0,x)}{u_n-u_0}d\mu(x)
	\end{eqnarray*}
	\begin{itemize}
		\item $\frac{f(u_n,x)- f(u_0,x)}{u_n-u_0} \to \diffp{f}{u}(u_0,x)$
		\item $\left|\frac{f(u_n,x)- f(u_0,x)}{u_n-u_0} \right| \leq g(x)$
	\end{itemize}
	Donc d'après le TCD on a %TDOD: Add ref
	$$ \frac{F(u_n)- F(u_0)}{u_n-u_0} \to \int \diffp{f}{u}(u_0,x) d\mu(x)$$

\end{proof}

\begin{remarque}
	On peu changer les hypothèses 2 et 3 pour avoir une forme plus pratique:
	\begin{itemize}
		\item $\mu-pp\  f(\cdot, x)$ est dérivable sur $I$.
		\item Il existe $g\in \Li$ telle que
		      $$ \forall u\in I \left|\diffp{f}{u}(u,x)\right| \leq g(x) \mu-pp$$
		      Ceci implique , par le théorème des accroissements finis, la majoration de l'hypothèse 3.
	\end{itemize}
	et dans e cas on a que $F$ est dérivable sur $I$ tout entier.
\end{remarque}

\begin{remarque}
	Si $f$ est à valeurs complexes cela marche aussi.
\end{remarque}


\begin{example}[Trasformée de Fourier]
	Si $\phi \in \Li$ on définie sa transformée de Fourier:
	$\hat{\phi} = \int e^{iux}\phi(x) d\lambda(x)$
	alors $\hat{\phi}$ est bien définie dans $\R$ et continue (par le théorème de continuité sous le signe intégrale).\\
	Si de plus on a $\int |x\phi(x)|d\lambda(x)<\infty$ alors $\hat{\phi}$ est dérivable sur $\R$ de dérivée:
	$$\hat{\phi}'(x)= \int ixe^{iux}\phi(x)d\lambda(x)$$
	%TODO add proof
\end{example}



\begin{theorem}[Régularité $C^k$ sous le signe intégrale]
	Soit $f: I \times E \to \R$ où $I$ est un intervalle ouvert de $\R$\\

	\begin{itemize}
		\item $\forall u \in I, \ f(u,\cdot) \in \Li(E,\triA,\mu)$
		\item $\mu-pp u \ \mapsto f(u,x)$ est $C^k$ sur $I$.
		\item Il existe $g_k\in \Li$ telle que
		      $$ \forall u\in I \left|\diffp[i]{f}{u}(u,x)\right| \leq g_k(x) \mu-pp$$
	\end{itemize}

	Alors $F(u) = \int f(u,x)d\mu$ est $C^k$ sur $I$, avec $F^{(i)}(u) = \int \diffp[i]{f}{u}(u,x)d\mu(x)$

\end{theorem}


\section{Mesure produits}


\subsection{Espace produit, tribu produit}



\begin{definition}[Tribu produit]
	Soient $(E,\triA)$ et $(F,\triB)$ deux espaces mesurables on dñefinit la tribu produit.
	$$\triA \xor \triB = \sigma \left(\left\{ A \times B \mid A\in \triA, B \in \triB \right\}\right)$$º
\end{definition}

\begin{remarque}
	Les ensembles de type $A\times B, A\in \triA, B\in\triB$ sont appelés rectangles ou pavés mesurables.
\end{remarque}



\begin{definition}
	On étend la définition à $n$ tribus:

	$$\triA_1 \xor \dots \xor \triA_n = \sigma \left(\left\{ A_0 \times \dots \times A_n \mid A_0\in \triA_0, \dots , A_n \in \triA_n \right\}\right)$$

	Cette définition est associative.
\end{definition}

\begin{prop}
	$\bor(\R^2) = \bor(\R) \xor \bor(\R)$
\end{prop}

\begin{proof}
	% TODO
\end{proof}

\begin{notation}
	Pour $C \subset E \times F$, on note:
	$$ C_x = \left\{ y \in F \mid (x,y) \in C \right\} $$
	$$ C^y = \left\{ x \in E \mid (x,y) \in C \right\} $$
	Pour $f: E \times F \to G$:
	$$f_x(y) = f(x,y) \quad (\1_C)_x = \1_{C_x}$$
	$$f^y(x) = f(x,y) \quad (\1_C)^y = \1_{C^y} $$
\end{notation}

\begin{prop}
	\begin{itemize}
		\item Si $c \in \triA \xor \triB$, alors
		      $$\forall x \in E,\ C_x \in \triB$$
		      $$\forall y \in F,\ C_y \in \triA$$
		\item Soit $f (E\times F, \triA \xor \triB) \to (C, \triC)$ mesurable alors:
		      $$\forall x \in E, \ f_x (F, \triB)  \to (C, \triC) \text{ mesurable}$$
		      $$\forall y \in F, \ f_y (E, \triB)  \to (C, \triC) \text{ mesurable}$$
	\end{itemize}
\end{prop}


\begin{proof}
	\begin{itemize}
		\item Soit $x \in E$ montrons que $\forall C \in \triA \xor \triB \ C_x \in \triB$.$\forall C \in \triA \xor \triB \ C_x \in \triB$.\\
		      On regarde $\triT = \left\{ C \in \triA \times  \triB , C_x \in \triB \right\}$ et montrons que c'est une tribu:
		      \begin{itemize}
			      \item $\emptyset_x = \emptyset \in \triB$
			      \item Soit $C$ tel que $C_x \in \triB $ %TODO
		      \end{itemize}

	\end{itemize}
\end{proof}



\subsection{Construction de la mesure produit}

\begin{theorem}
	Soient $\mu$ et $\nu$ deux mesures $\sigma$-finies sur $(E, \triA)$ et $(F, \triB)$ respectivement.

	\begin{itemize}
		\item Il existe une unique mesure $\sigma$-finie sur ($E \times F, \triA \xor \triB)$ telle que $\forall A \in \triA, B \in \triB, \mu \xor \nu(A \times B) = \mu(A) \nu(B)$.
		\item $\forall C \in \triA \xor \triB. $
		      $$ \mu \xor \nu(C) = \int_E \nu(C_x) d \mu(x) = \int_F \mu(C^y) d \nu(y) $$
	\end{itemize}
\end{theorem}

\begin{proof}
	\begin{itemize}
		\item  Unicité:\\
		      $\mu$ et $\nu$ sont $\sigma$-finies, on se donne donc $E_n \uparrow E$ et $F_n \uparrow F$ tels que $\mu(E_n) < +\infty$ et $\nu(F_n) < +\infty$.\\
		      On pose $C_n = E_n \times F_n$ et on a que $C_n \uparrow E \times F$.\\
		      Soient $m_1$ et $m_2$ deux mesures qui vérifient les propriétés de la proposition.\\
		      \begin{eqnarray*}
			      m_1(E_n \times F_n) &=& \mu(E_n) \nu(F_n) < +\infty \\
			      & = & m_2(E_n \times F_n)
		      \end{eqnarray*}

		      \begin{itemize}
			      \item $m_1$ et $m_2$ coincident sur $\set{A \times B \mid A \in \triA, B \in \triB}$ qui engendre $\triA \xor \triB$ et stable par intersections finies.
			      \item $m_1 (C_n) = m_2(C_n) < +\infty$ et $C_n \uparrow E \times F$ donc $m_1 = m_2$ d'après le théorème d'unicité des mesures.
		      \end{itemize}
		      %TODO: It's really long, idk if it's worth it. If someone wants to do it, go ahead.
	\end{itemize}
\end{proof}

\begin{remarque}
	On peu itérer $\mu_1 \xor \mu_2 \xor \dots \xor \mu_n = \mu_1 \xor (\mu_2 \xor (\dots \xor (\mu_{n-1} \xor \mu_n)\dots)$ et c'est associatif.
	$$ \mu_1 \xor \mu_2 \xor \dots \xor \mu_n(A_1 \times \dots \times A_n) = \prod_{i=1}^n \mu_i(A_i) $$
\end{remarque}


\begin{example}
	La mesure de Lebesgue sur $\R^n$ est le produit des mesures de Lebesgue sur $\R$:

	$$\lambda_n = \underbrace{\lambda \xor \dots \xor \lambda}_{n \text{ fois}}$$
	et elle vérifie:
	$$\lambda_n([a_1, b_1] \times \dots \times [a_n, b_n]) = \prod_{i=1}^n (b_i - a_i)$$
\end{example}

\begin{remarque}
	Si $X \indep Y$, alors $\mathbb{P}_{X,Y} = \mathbb{P}_X \xor \mathbb{P}_Y$.
	%TODO: Add random probability stuff
\end{remarque}

\subsection{Les théorèmes de Fubini}

\begin{theorem}[Fubini-Tonelli]
	Soient $\mu$ et $\nu$ deux mesures $\sigma$-finies sur $(E, \triA)$ et $(F, \triB)$ respectivement.
	Soit $f: (E \times F, \triA \xor \triB) \to ([0, +\infty], \mathcal{B}([0, +\infty]))$ une fonction mesurable.

	\begin{itemize}
		\item Les fonctions :
		      $$ x \mapsto \int f(x, y) d \nu(y)$$
		      $$ y \mapsto \int f(x, y) d \mu(x)$$
		      sont respectivement $\triA$ et $\triB$ mesurables.
		\item On a :
		      \begin{eqnarray*}
			      \int_{E \times F} f d(\mu \xor \nu) & = & \int_E  \int_F f(x, y) d \nu(y) d \mu(x) \\
			      & = & \int_F  \int_E f(x, y) d \mu(x) d \nu(y)
		      \end{eqnarray*}
	\end{itemize}
\end{theorem}


\begin{proof}
	%TODO
\end{proof}


\begin{theorem}[Fubini-Lebesgue]
	Soit $f \in \Li(E \times F, \triA \xor \triB, \mu \xor \nu)$.

	\begin{itemize}
		\item $$ \mu-pp, \ y \mapsto f(x, y) \in \Li(F, \triB, \nu)$$
		      $$ \nu-pp, \ x \mapsto f(x, y) \in \Li(E, \triA, \mu)$$
		\item On a :
		      $\mu-pp, \ y \mapsto \int f(x, y) d \nu(y)$ est bien définie et est dans $\Li(E, \triA, \mu)$.\\
		      $\nu-pp, \ x \mapsto \int f(x, y) d \mu(x)$ est bien définie et est dans $\Li(F, \triB, \nu)$.
		\item On a :
		      \begin{eqnarray*}
			      \int_{E \times F} f d(\mu \xor \nu) & = & \int_E  \int_F f(x, y) d \nu(y) d \mu(x) \\
			      & = & \int_F  \int_E f(x, y) d \mu(x) d \nu(y)
		      \end{eqnarray*}
	\end{itemize}
\end{theorem}


\begin{proof} %TODO: Make pretty
	On regarde $|f|$, qui est mesurable positive d'après Fubinni-Tonelli. On a donc :
	$$ \int |f| d (\mu \xor \nu) = \int \int |f| d \nu (y) d \mu (x) < + \infty$$
	par hypothèse. \\
	On a donc $ x   \mapsto \int |f|(x,y) d \nu(y)$ est finie $\mu-pp$ car son intégrale est finie. \\
	i.e: $\mu-pp, \int |f| d \nu(y) < + \infty, |f|(x,y) \in \Li(F, \triB, \nu)$ \\
	De plus $\int \left[ \int |f|(x,y) d    \nu(y)   \right] d \mu(x) < + \infty $.
	Donc $x \mapsto \int f (x,y) d \nu(y)$ est bien définie et est dans $\Li(E, \triA, \mu)$
	car $\int \left| \int f (x,y) d \nu(y) \right| d \mu(x) < \int \int |f| d \nu(y) d \mu(x) < + \infty$.
	De même pour les deux autres fonctions.

	$f = f^+ - f^-$, donc on a :
	\begin{eqnarray*}
		\int f d (\mu \xor \nu) & = & \int f^+ d (\mu \xor \nu) - \int f^- d (\mu \xor \nu) \\
		& = & \int \int f^+(x,y) d \nu(y) d \mu(x) - \int \int f^-(x,y) d \nu(y) d \mu(x) \text{ (Fubini-Tonelli)} \\
		& = & \int \left( \int f^+(x,y) d \nu(y) - \int f^-(x,y) d \nu(y) \right) d \mu(x) \\
		&=& \int \int f^+(x,y) - f^-(x,y) d \nu(y) d \mu(x) \\
		&=& \int \int f(x,y) d \nu(y) d \mu(x)
	\end{eqnarray*}
	De même dans l'autre sens.
\end{proof}

\begin{remarque}
	L'hypothèse $f \in \Li(E \times F, \triA \xor \triB, \mu \xor \nu)$ est indispensable.
\end{remarque}



\section{La formule de changement de variables}


\begin{prop}[Formule de changement de variable linéaire]
	Soit $b \in \R^d, \ M \in \mathcal{M}_d(\R)$ inversible et
	$$ \begin{array}{rcl}
			f : \R^d & \longrightarrow & \R^d   \\
			x        & \longmapsto     & Mx + b
		\end{array} $$
	Alors, pour tout borélien $A \subset \R^d$,
	$$ \lambda_d(f(A)) = |\det(M)| \lambda_d(A) $$
\end{prop}

\begin{remarque}
	Si $M$ n'est pas inversible, $\lambda_d(f(A)) = 0$, car l'image est incluse dans un hyperplan $H$
	de $\R^d$, qui est donc de mesure nulle.
\end{remarque}

\begin{proof}
	On admet que la mesure de Lebesgue est invariante par translation et c'est la seule a multiplication près (exo). On peut donc supposer
	$b = 0$. Soit $A \subset \R^d$ borélien.

	$B \mapsto \lambda_d(f(B))$ est une mesure qui est invariante par translation.

	$f (\emptyset) = \emptyset$ et $f$ est bijective, donc l'image d'un d'une famille deux à deux disjoints est aussi deux à deux disjoints, et donc elle est
	$\sigma$-additive.
	\begin{eqnarray*}
		\lambda_d(f([B+a])) &=& \lambda_d(f(B) + f(a)) \\
		&=& \lambda_d(f(B))
	\end{eqnarray*}
	alors elle est un multiple de $\lambda_d$.

	Il suffit de vérifier que $\lambda_d(f([0,1]^d)) = |\det(M)|$.

	\begin{itemize}
		\item Si $M$ est diagonale $M = \begin{pmatrix} a_1 & & \\ & \ddots & \\ & & a_d \end{pmatrix}$, alors
		      $$ \lambda_d(f([0,1]^d)) = \lambda_d([0, a_1] \times \cdots \times [0, a_d]) =\left| \prod_{i=1}^d a_i \right|= |\det(M)| $$
		\item Si $M$ est orthogonale, le coefficient vaut 1 car $M$ conserve la boule unité.
		\item Si M est symétrique définie positive, le coefficient vaut $\det(M)$ car $M$ est diagonale après un changement de base.
		\item Dans le cas, $M = PS$ avec $P$ orthogonale et $S$ symétrique définie positive, avec $S =\sqrt{M^tM}$ et $P = M S^{-1}$.
		      Donc le coefficient vaut $|\det(P)| |\det(S)| = |\det(M)|$.
	\end{itemize}
\end{proof}


\begin{rappel}
	$U, D$ ouverts de $\R^d$, $\phi : U \to D$ est un $C^1$-difféomorphisme si $\phi$ est bijective et $C^1$ sur $U$ et $\phi^{-1}$ est $C^1$ sur $D$.
	Dans ce cas $\forall u \in U, \ \phi'(u)$ est inversible.
\end{rappel}


\begin{theorem}[Changement de variables]
	Soit $\phi: U \to D$ un $C^1$-difféomorphisme alors pour toute fonction $f : D \to \R^+$mesurable positive:
	$$\int_D f(x) d\lambda(x) = \int_U f(\phi(u)) |J_\phi (u)| d u $$
	où $J_\phi =\det(\phi(u))$ est le jacobien de $\phi$ en $u$.
\end{theorem}



\section{Espace $L^2$}

\subsection{Définition et premières propriétés}

\begin{definition}
	On définit lénsemble des fonction (réelles ou complexes) de carré intégrable sur un intervalle $(E, \triA, \mu)$ noté
	$\Ld (E, \triA, \mu) = \set{ f: (E, \triA) \to (\R, \bor(\R)) \mid \int_E \abs{f}^2 d \mu < \infty }$.
\end{definition}


\begin{prop}
	Sur $\Ld (E, \triA, \mu)$, la notion $f \sim g \ssi f = g \text{ p.p.}$ définit une relation d'équivalence.
\end{prop}

\begin{definition}
	On définit l'espace quotient $L^2(E, \triA, \mu) = \Ld (E, \triA, \mu) / \sim$.
\end{definition}

\begin{remarque}
	En fait on fait le travail deans tous les espaces $L^p$.

	$$ \Lp^p (E, \triA, \mu) = \set{ f: (E, \triA) \to (\R, \bor(\R)) \mid \int_E \abs{f}^p d \mu < \infty }$$
	et en particulier dans $L^1$.
\end{remarque}

\begin{remarque}
	Dans la suite on identifie (abusivement) un élément de $L^1$ avec un représentant dans $\Lp^1$.
	En particulier ils ont la même intégrale.

	Sur $L^2$, on note $\norm{f}_2 = \sqrt{\int_E \abs{f}^2 d \mu}$
\end{remarque}


\begin{theorem}[Inégalité de Cauchy-Schwarz]
	Soient $f, g$ mesurables sur $(E, \triA, \mu)$, alors
	$$\int \abs{fg} d \mu \leq \norm{f}_2 \norm{g}_2$$
	En particulier, $fg \in L^1$ si $f, g \in L^2$.\\
	Et il y a égalité si  $f$ et $g$ sont colinéaires :
	$$\exists \lambda \in \R, \ f = \lambda g \text{ ou } g = \lambda f$$
\end{theorem}

\begin{proof}
	Si $\norm{f}_2 = 0$, alors $f = 0$ $\mu-pp$ donc $fg = 0$ $\mu-pp$ et l'inégalité est vraie et de
	même si $\norm{g}_2 = 0$.

	Supposons donc que $\norm{f}_2 > 0$ et $\norm{g}_2 > 0$.\\
	Soit $t \in \R$, alors on regarde
	$$ 0 \leq \int (f + tg)^2 d \mu = \int_E f^2 d \mu + 2t \int fg d \mu + t^2 \int g^2 d \mu$$
	avec $fg \in L^1$ car $\abs{fg} \leq \frac{f^2g^2}{2}$.
	C'est un polynôme de degré 2 en $t$ qui est positif pour tout $t$ donc son discriminant est négatif.
	$$  \Delta = (2 \int fg d \mu)^2 - 4 \int f^2 d \mu \int g^2 d \mu \leq 0$$
	$$ \int fg d \mu \leq \sqrt{\int f^2 d \mu \int g^2 d \mu}$$
	Le cas d'égalité est immédiat : $\Delta = 0$ Donc :
	$$\exists t \text {tel que} \int (f + tg)^2 d \mu  = 0$$
	$$ f + tg = 0 \mu-pp$$
	$f$ et $g$ sont colinéaires $\mu-pp$.

	Si $\norm{f}_2 = +  \infty$ ou $\norm{g}_2 = +  \infty$, l'inégalité est évidente.
\end{proof}

\begin{coro}
	Si $\mu$ est une mesure finie, alors $L^2(E, \triA, \mu) \subset L^1(E, \triA, \mu)$.
\end{coro}

\begin{proof}
    \begin{eqnarray*}
        \int \abs{f} d \mu \leq \int \abs{f}*1 d \mu &\leq& \sqrt{\int f^2 d \mu} \sqrt{\int 1^2 d \mu} \\
                                                     & \leq & \norm{f}_2 \sqrt{\mu(E)} < \infty
    \end{eqnarray*}
\end{proof}





\section{Espaces de Hilbert}

\subsection{Définitions, premières propriétés}

\begin{definition}
	Soit $H$ un $\R$ ou $\C$-espace vectoriel.
	Un produit scalaire est une application $\sprod{\cdot}{\cdot} : H \times H \to \R$ ou $\C$ telle que
	\begin{itemize}
		\item $\forall y \in H, x \mapsto \sprod{x}{y}$ est linéaire
		\item $\forall x,y \in H, \sprod{x}{y} = \overline{\sprod{y}{x}}$
		\item $\forall x \in H, \sprod{x}{x} \geq 0$ et $\sprod{x}{x} = 0 \iff x = 0$
	\end{itemize}

	Si $\sprod{\cdot}{\cdot}$ vérifie seulement les deux premières propriétés, on dit que c'est un produit hermitien.
\end{definition}

\begin{remarque}
	$\sprod{x}{\lambda y} = \overline{\lambda} \sprod{x}{y} $.
	$y \mapsto \sprod{x}{y}$ est antilinéaire.
	$(x,y) \mapsto \sprod{x}{y}$ dite sesquilinéaire (linéaire en la première variable et antilinéaire en la seconde).
	On note en général $\norm{x}_2 = \sqrt{\sprod{x}{x}}$. On montrera que c'est une norme.
\end{remarque}

\begin{remarque}
	\begin{eqnarray*}
		\norm{x + y}_2^2 &=& \sprod{x + y}{x + y}\\
		&=& \sprod{x}{x} + \sprod{x}{y} + \sprod{y}{x} + \sprod{y}{y}\\
		&=& \norm{x}_2^2 + 2\Re \sprod{x}{y} + \norm{y}_2^2
	\end{eqnarray*}
\end{remarque}

\begin{prop}[Inégalité de Cauchy-Schwarz]
	Soit $H$ un espace de Hilbert, alors
	$$ \forall x,y \in H, \abs{\sprod{x}{y}} \leq \norm{x}_2 \norm{y}_2 $$
\end{prop}

\begin{proof}
	\begin{itemize}
		\item Si $\norm y = 0, \ 0 \leq 0$.
		\item Sinon on regarde
		      $$ 0 \leq \norm{x + ty}^2 = \norm{x}^2 + 2t\Re \sprod{x}{y} + t^2 \norm{y}^2 $$
		      qui est un polynôme du second degré en $t$.
		      On pose $\Delta = 4\Re \sprod{x}{y}^2 - 4\norm{x}^2\norm{y}^2 \leq 0$.
		      Donc $\abs{\Re \sprod{x}{y}} \leq \norm x \norm y$.

		      Si on est dans un $\R$-espace vectoriel, on a fini. Sinon:
		      \begin{eqnarray*}
			      \sprod x y &=& \abs{\sprod x y} e^{i\theta} \\
			      \abs{\sprod x y} &=& e^{-i\theta} \sprod x y \\
			      &=&    \sprod {e^{-i\theta} x} y
		      \end{eqnarray*}
		      Or $\abs{\sprod x y} = \abs {\Re ( \sprod {e^{-i\theta} x} y )} \leq \norm {e^{-i\theta} x} \norm y = \norm x \norm y$.
	\end{itemize}
\end{proof}


\begin{coro}
	$\norm{x} = \sqrt{\sprod{x}{x}}$ définit une norme sur $H$ si $\sprod{\cdot}{\cdot}$ est un produit scalaire.
\end{coro}

\begin{proof}
	\begin{itemize}
		\item Par définition $\norm{x} \geq 0$.
		\item \begin{eqnarray*}
			      \norm{x + y}^2 &=& \sprod{x+y}{x+y} \\
			      &=&\norm x^2 + 2\Re \sprod{x}{y} + \norm{y}^2 \\
			      &\leq& \norm{x}^2 + 2\abs{\sprod{x}{y}} + \norm{y}^2 \\
			      &\leq& \norm{x}^2 + 2\norm{x}\norm{y} + \norm{y}^2 \text{ par Cauchy-Schwarz} \\
			      &\leq& (\norm{x} + \norm{y})^2
		      \end{eqnarray*}
		\item Soit $\lambda$ un scalaire:
		      \begin{eqnarray*}
			      \norm{\lambda x}^2 &=& \sprod{\lambda x}{\lambda x} \\
			      &=&\lambda \overline \lambda \sprod{x}{x} \\
			      &=& \abs{\lambda}^2 \norm{x}^2 \\
			      \norm{\lambda x} &=& \abs{\lambda} \norm{x}
		      \end{eqnarray*}
		\item $\norm{x} = 0 \implies \sprod{x}{x} = 0 \implies x = 0$ par définition du produit scalaire.
	\end{itemize}
\end{proof}

\begin{prop}[Identités remarquables]
	Soient $x,y \in H$, $\sprod{\cdot}{\cdot}$ un produit scalaire sur $H$ et $H$ un $\C$-espace vectoriel.

	\begin{itemize}
		\item Identité de Polarisation:
		      $$\sprod{x}{y} = \frac{1}{4} \left( \norm{x+y}^2 - \norm{x-y}^2 + i\norm{x+iy}^2 - i\norm{x-iy}^2 \right)$$
		\item Dans un $\R$-espace vectoriel, on a:
		      $$\sprod{x}{y} = \frac{1}{4} \left( \norm{x+y}^2 - \norm{x-y}^2 \right)$$
		\item Identité de la médiane / du parallélogramme:
		      $$\norm{x+y}^2 + \norm{x-y}^2 = 2\left( \norm{x}^2 + \norm{y}^2 \right)$$
		\item Pythagore: Si $\sprod{x}{y} = 0$, alors
		      $$ \norm{x+y}^2 = \norm{x}^2 + \norm{y}^2$$
	\end{itemize}
\end{prop}

\begin{definition}
	On dit que $x$ et $y$ sont orthogonaux si $\sprod{x}{y} = 0$.
\end{definition}

\begin{proof}
	$$\norm{x+y}^2  = \norm{x}^2 + 2\Re \sprod{x}{y} + \norm{y}^2 $$
	$$\norm{x-y}^2  = \norm{x}^2 - 2\Re \sprod{x}{y} + \norm{y}^2 $$

	Par différence $\Re \left(\sprod{x}{y}\right) = \frac{1}{4} \left( \norm{x+y}^2 - \norm{x-y}^2 \right)$
	\begin{itemize}
		\item
		      \begin{eqnarray*}
			      \norm{x+y}^2  &=& \norm{x}^2 + 2\underbrace{\Re \sprod{x}{y} }_{= \Re (-i\sprod{x}{y}) = \Im \sprod{x}{y}} + \norm{y}^2 \\
			      \norm{x-iy}^2 &=& \norm{x}^2 - 2\Im \sprod{x}{y} + \norm{y}^2 \\
			      \Im \sprod{x}{y} &=& \frac{1}{4} \left( \norm{x+iy}^2 - \norm{x-iy}^2 \right)\\
			      \sprod{x}{y} &=& \Re \sprod{x}{y} + i \Im \sprod{x}{y}
		      \end{eqnarray*}
		      D'où l'identité de polarisation.
		\item On démontre la deuxième identité et¡n utilisant la différence, car sur un $\R$-espace vectoriel, $\Re \left(\sprod{x}{y}\right) = \sprod{x}{y}$.
		\item L'identité de la médiane est immédiate en additionnant les deux premières équations.
		\item Et enfin, si $\sprod{x}{y} = 0$, alors $\norm{x+y}^2 = \norm{x}^2 + \norm{y}^2$.
	\end{itemize} %TODO: Maybe add figure
\end{proof}


\begin{remarque}
	Dans un $\C$-espace vectoriel, le théorème de Pythagore n'admet pas de réciproque.
\end{remarque}

\begin{definition}
	$H$ muni de son produit scalaire est un espace de Hilbert s'il est complet pour la norme associée au produit scalaire.
\end{definition}


\subsection{Projection orthogonale dans un espace de Hilbert}

\begin{definition}
	$F$ un sous espace de $H$, on définit l'othogonal de $F$ par $F^\perp = \{y \in H, \forall x \in F, \sprod{x}{y} = 0\}$.
\end{definition}

\begin{prop}
	$F, G$ deux sous-espaces vectoriels de $H$, alors:
	\begin{itemize}
		\item $F^\perp$ est fermé.
		\item Si $F \subset G$ alors $G^\perp \subset F^\perp$.
		\item $F^\perp = \overline{F}^\perp$.
	\end{itemize}
\end{prop}


\begin{proof}
	\begin{itemize}
		\item Soit $x_n \in F^\perp$ tel que $x_n \to x$, montrons que $a \in F^\perp$.
		      Soit $y \in F$, montrons que $\sprod{x}{y} = 0$.
		      $$ \abs{\sprod{x_n-x}{y}} \leq  \underbrace{\norm{x_n-x}}_{\to 0} \norm{y} $$
		      $$ \underbrace{\sprod{x_n}{y}}_{=0} - \sprod{x}{y}  \to 0 $$
		      donc la suite stationne en 0, donc $\sprod{x}{y} = 0$.

		\item Soit $x \in G^\perp$, montrons que $x \in F^\perp$.\\
		      Soit $y \in F$ montrons que $\sprod{x}{y} = 0$. Or $F \subset G$ donc $y \in G$ et donc $\sprod{x}{y} = 0$.
		\item Par le 2 $\overline{F}^\perp \subset F^\perp$.\\
		      Montrons que $F^\perp \subset \overline{F}^\perp$.\\
		      Soit $x \in F^\perp$, montrons que $x \in \overline{F}^\perp$.\\
		      Soit $y \in \overline{F}$, montrons que $\sprod{x}{y} = 0$.\\
		      Soit $y_n \in F$ tel que $y_n \to y$, on a
		      $$ \abs{\sprod{x}{y- y_n}} \leq \norm{x} \norm{y-y_n} \to 0 $$
		      or
		      $$  \abs{\sprod{x}{y- y_n}} = \sprod{x}{y} -  \underbrace{\sprod{x}{y_n}}_{=0} $$
		      Donc $\sprod{x}{y} = 0$.
	\end{itemize}
\end{proof}

%TODO: Maybe eqnarray this?
\begin{remarque}
	D'après Cauchy-Schwarz,
	$$ \forall x \in H, \  x \mapsto \sprod{x}{y} \text{ est continue} $$
	$$ \forall y \in H, \  y \mapsto \sprod{x}{y} \text{ est continue} $$
\end{remarque}

\begin{theorem}[Projection orthogonale sur un sous espace fermé]
	Soit $F$ un sous espace fermé de $H$.

	\begin{itemize}
		\item $\forall x \in H, \exists ! y \in F, \norm{x-y} = \dist{x} {F}$.\\
		      Où $\dist{x} {F} = \inf_{y \in F} \norm{x-y}$.\\
		      On note $y = p_F(x)$ cet unique point, cela définit une application $p_F : H \to F$.
		\item $p_F(x)$ est l'unique vecteur de $F$ tel que $x - p_F(x) \in F^\perp$.
		\item $\forall x \in H,\ \norm{x}^2 = \norm{p_F(x)}^2 + \norm{x - p_F(x)}^2$.
		\item $p_F$ est linéaire et continue.
		\item $F$ et $F^\perp$ sont dits supplémentaires orthogonaux.\\
		      $$ H = F + F^\perp \text{ et } F \cap F^\perp = \{0\} $$
		      Et $$\forall x \in F , \forall y \in F^\perp, \sprod{x}{y} = 0$$
		      On note cela $H = F \oplus F^\perp$.
	\end{itemize}
\end{theorem}


\begin{proof}

	\begin{itemize}
		\item Soit $y_n \in F$ tel que $\norm{x-y_n} \to \dist{x}{F}$. Comme $H$ est complet, il suffit de montrer que $y_n$ est de Cauchy.\\
		      On prend $a = x - y_n$ et $b = x - y_m$.
		      \begin{eqnarray*}
			      \norm{a + b}^2  + \norm{a - b}^2 & = & 2 \norm{a}^2 + 2 \norm{b}^2\\
			      \norm {a-b}^2 & = & 2 \left( \norm{a}^2 +  \norm{b}^2 \right) -  \norm{a+b}^2\\
			      \norm {y_n - y_m}^2 & = & 2 \left( \norm{x-y_n}^2 +  \norm{x-y_m}^2 \right) - 4 \norm{x - \frac{y_n + y_m}{2}}^2 \\
			      & \leq & 2 \left( \norm{x-y_n}^2 +  \norm{x-y_m}^2 \right) - 4 {\dist x F}^2  %TODO: Underlines
		      \end{eqnarray*}

		      Si $N$ est assez grand pour que $\forall n \geq N, \abs {\norm{x-y_n}^2 - \dist{x}{F}^2} \leq\frac{\varepsilon}{4}$.\\
		      Si $n,m \geq N$ on a
		      \begin{eqnarray*}
			      \norm {y_n - y_m}^2 & \leq & 2 \left( \dist{x}{F}^2 +   \frac \varepsilon 4 + \dist{x}{F}^2 +   \frac \varepsilon 4 \right) - 4 {\dist x F}^2 \\
			      &\leq & \varepsilon
		      \end{eqnarray*}
		      $y_n$ est donc de Cauchy et donc converge vers un point $y \in H$, $y_n \in F$ donc $y \in F$ car $F$ est fermé.\\

		      Unicité: Supposons que $y$ et $y'$ soient deux points de $F$ qui réalisent le minimum.\\
		      $$ \norm{y - y'}^2 = 2 \left( {\dist x F}^2 + {\dist x F}^2 \right) - 4 {\dist x F}^2 \leq 0 $$
		      Donc $y = y'$.

		\item Montrons que $x - p_F(x) \in F^\perp$.\\
		      Soit $z \in F$, montrons que $\sprod{x - p_F(x)}{z} = 0$.\\
		      On pose $y = p_F(x) \in F$. Soit $t\in \Rp$, on a $y + t z \in F$ Donc
		      \begin{eqnarray*}
			      {\dist x F}^2 & \leq & \norm{x - (y + t z)}^2\\
			      & \leq & \norm{x - tz - y}^2\\
			      & = & \norm{x - y}^2 - 2\Re \sprod{x - y}{tz} + t^2 \norm{z}^2\\
			      & \leq & {\dist x F}^2 - 2t \Re \sprod{x - y}{z} + t^2 \norm{z}^2
		      \end{eqnarray*}

		      Donc $2t\Re \sprod{x - y}{z} + t^2 \norm{z}^2 \geq 0$ pour tout $t \in \Rp$.
		      En particulier pour $t = 0$ on a $\Re \sprod{x - y}{z} = 0$ \\
		      En utilisant $itz$ à la place de $tz$ on trove $\Im \sprod{x - y}{z} = 0$\\

		      Unicité: Supposons $y$ et $y'$ deux points de $F$ tels que $\forall z \in F, \sprod{x - y}{z} = 0$ et $\sprod{x - y'}{z} = 0$.\\
		      $\forall x \in F, \sprod{y - y'}{z} = 0$, or $y - y' \in F$ donc $\sprod{y - y'}{y - y'} = 0$ donc $y = y'$.
		\item D'après 2, $x - p_F(x) \perp p_F(x)$, donc par Pythagore,
		      $$ \norm{x}^2 = \norm{p_F(x)}^2 + \norm{x - p_F(x)}^2 $$
		\item Montrons que $p:F( x + \lambda y) = p_F(x) + \lambda p_F(y)$.\\
		      Montrons que $p_F(x) + \lambda p_F(y)$ vérifie 2.\\
		      Montrons que $x + \lambda y - p_F(x) - \lambda p_F(y) \in F^\perp$.\\
		      $$ \underbrace{x - p_F(x)}_{\in F^\perp} + \lambda\left( \underbrace{y - p_F(y)}_{\in F^\perp} \right) \in F^\perp $$
		      Montrons que $p_F$ est continue.\\
		      Comme $p_F$ est linéaire, il suffit de montrer que $p_F$ est continue en 0.\\
		      $p_F$ continue \ssi $\exists M, \ \norm{p_F(x)} \leq M \norm{x}$.\\

		      D'après 3, $\norm{p_F(x)}^2 = \norm{x}^2 - \norm{x - p_F(x)}^2 \leq \norm{x}^2$. Donc $p_F$ est continue et
		      $$ \left| \norm{p_F(x)} \right| \leq 1 $$ %TODO: command for this

		\item $x = \underbrace{p_F(x)}_{\in F} + \underbrace{(x - p_F(x))}_{\in F^\perp \text { d'après 2}}$.
	\end{itemize}
\end{proof}

\begin{remarque}
	Si $F$ est de dimension finie, $(e_1, \ldots, e_n)$ une base orthonormée de $F$, alors $p_F(x) = \sum_{i=1}^n \sprod{x}{e_i} e_i$.
\end{remarque}

\begin{remarque}
	Si $F$ n'est pas fermé rien ne fonctionne.
\end{remarque}

\begin{coro}
	\begin{itemize}
		\item Si $F$ est fermé, alors ${F^\perp}^\perp = F$.
		\item Si $F$ est dense \ssi $F^\perp = \set 0$.
	\end{itemize}
\end{coro}

\begin{proof}
	\begin{itemize}
		\item C'est le point 5 du théorème.
		\item $\overline{F}^\perp = F^\perp$

		      \begin{itemize}
			      \item Si $F$ est dense, $\overline{F}^\perp = H^\perp = \set 0$.
			      \item Sinon $\overline{F} \subsetneq H$  et $ \overline{F} \oplus \overline{F}^\perp = H$.
			            Donc $\overline{F}^\perp \neq \set 0$.
		      \end{itemize}
	\end{itemize}

\end{proof}

\subsection{Théorème de représentation) de Riesz}

\begin{theorem}[Théorème de représentation de Riesz]
	Soit $H$ un espace de Hilbert, $\phi : H \to \C$ une forme \textbf{linéaire continue}.\\
	Alors il existe un unique $y \in H$ tel que $\forall x \in H, \phi(x) = \sprod{x}{y}$.
	De plus $\left|\norm{\phi}\right| = \norm{y}$.
\end{theorem}


\begin{proof}
	Si $\phi = 0$ alors $y = 0$ convient.\\
	Si ce n'est pas le cas on regarde $F = \phi^{-1}(\set 0)$, $\phi$ est continue donc $F$ est fermé.\\
	$F \subsetneq H$ car $\phi \neq 0$.\\
	En fait $\dim F^\perp = 1$\\
	Si $x,y \in F^\perp,\ x \neq 0, y \neq 0$ alors $\sprod{x}{y} \neq 0$ car $x \neq 0$ et $y \neq 0$.\\
	%TODO
\end{proof}



\subsection{Bases orthonormales}


\begin{definition}
	On dit que $H$ est séparable s'il existe une partie dénombrable dense de $H$. On encore une suite de points dense dans $H$.
\end{definition}


\begin{example}
	$\R$ est séparable ($\Q$ est dense dans $\R$ et dénombrable).
\end{example}

\begin{example}
	$\Ld$ est séparable.
\end{example}

\begin{proof}
	On a vu que les fonctions continues à support compact sont denses dans $\Ld$.


	Toute fonction continue à support compact peut être approchée par une fonction étagée à valeurs rationnelles,
	avec pour ensemble de niveaux des pavés à coordonnées rationnelles.
	$$ \sum_{k=-n^2}^{n^2} f_{\epsilon} \left(\frac k n\right) \1_{[\frac k n, \frac {k+1} n[} $$
	avec $f_{\epsilon}$ un rationnel $\frac {\epsilon} 4$-proche de $f\left(\frac k n\right)$.

	\begin{itemize}
		\item approche $f$ en $\norm{\cdot}_2$.
	\end{itemize}

	%TODO
\end{proof}

\begin{example}
	$\Ld(\N)$ et $\Q^{(\N)}$ l'ensemble des suites nulles à partir d'un certain rang à valeurs rationnelles.
\end{example}

\begin{definition}[Base orthonormale / hilbertienne]
	On dit que $(e_n)_{n\in \N}$ est une base orthonormale/hilbertienne de $H$ si
	\begin{itemize}
		\item $\norm {e_n} = 1, \ \sprod {e_i} {e_j} = \delta_{ij} = \begin{cases} 1 & \text{si } i=j \\ 0 & \text{sinon} \end{cases}$
		\item L'ensemble des combinaisons linéaires finies d'éléments $(e_n)_{n\in \N}$ est dense dans $H$.
	\end{itemize}
\end{definition}


\begin{example}
	$\Ld(\N)$  et $\sprod  a b = \sum\limits_{k=1}^n a_k \overline{b_k}$.
	$e_k = (0, \ldots, 0, 1, 0, \ldots)$ avec $1$ en position $k$.
	$$ \sprod {e_i} {e_j} = \sum e_i \overline{e_j} = \delta_{ij} $$

	L'ensemble des combinaisons linéaires finies d'éléments de $(e_n)_{n\in \N}$ est $\C^{(\N)}$ l'ensemble des suites nulles à partir d'un certain rang, donc dense dans $\Ld(\N)$.
\end{example}

\begin{example}
	%TODO
\end{example}


\begin{theorem}
	Si $H$ est un espace de Hilbert séparable, alors $H$ admet une base hilbertienne.
\end{theorem}

\begin{proof}
	Soit $A$ une partie dénombrable dense de $H$.

	\begin{itemize}
		\item On construit avec $A$ une famille orthonormale.
		\item On vérifie que c'est une base hilbertienne.
	\end{itemize}

	$A = \set { a_n }_{n\in \N}$.

	\begin{itemize}
		\item Montrons le premier point:\\

		      Montrons que $\forall n \in \N, \exists p \leq n \exists f_0, \dots, f_p \subset \set {a_0, \dots, a_n}$ tel que $(f_0, \dots, f_p)$ est libre dans $H$ et $\vect {f_0, \dots, f_p} = \vect {a_0, \dots, a_n}$.

		      On le montre par récurrence sur $n$.

		      \begin{itemize}
			      \item $n=0$ : $f_0 = a_0$ si $a_0 \neq 0$. $(a_0)$ est libre, sinon $p = -1$ (la famille est vide).
			      \item Hérédité :
			            \begin{itemize}
				            \item Si $a_{n+1} \in \vect {a_0, \dots , a_n}$ on garde $(f_0, \dots, f_p)$ convient.
				            \item Si $a_{n+1} \notin \vect {a_0, \dots , a_n}$, alors $(f_0, \dots, f_n, a_{n+1})$ convient.
			            \end{itemize}
		      \end{itemize}

		      On travaille avec $f_0, \dots, f_p$.
		\item Soit $F$ un sous espace vectoriel fermé. Supposons que $(e_1, \dots, e_n)$ une base orthonormale de $F$.

		      \begin{eqnarray*}
			      p_F: H & \to & H \\
			      x & \mapsto & \sum_{k=1}^n \sprod x {e_k} e_k
		      \end{eqnarray*}
		      vérifie $x - p_F \in F^{\perp}$. Donc c'est bien la projection orthogonale sur $F$.

		      Posons $F_n = \vect {f_0, \dots, f_n}$.
		      alors:
		      \begin{eqnarray*}
			      e_0 &=& \frac {f_0} {\norm {f_0}} \\
			      e_1 &=& \frac {f_1 - p_{F_0} (f_1)} {\norm {f_1 - p_{F_0} (f_1)}} \\
			      &\vdots& \\
			      e_n &=& \frac {f_n - p_{F_{n-1}} (f_n)} {\norm {f_n - p_{F_{n-1}} (f_n)}}\\
			      &=& \frac {f_n - \sum\limits_{k=0}^{n-1} \sprod {f_n} {e_k} e_k} {\norm {f_n - \sum\limits_{k=0}^{n-1} \sprod {f_n} {e_k} e_k}} \\
			      &\vdots&
		      \end{eqnarray*}

		      On vérifie que $\norm {e_n} = 1$ et par récurrence $\forall n \in \N, \forall i < n, \sprod {e_n} {e_i} = 0$.


		      Or $e_n \in F_n$ et $\vect {(e_n)_{n\in \N}} \supset A$ Et $\vect {(e_n)_{n\in \N}}$ est l'ensemble des combinaisons linéaires finies. Or $A$ est dense dans $H$, donc
		      $\vect {(e_n)_{n\in \N}} = H$.
	\end{itemize}
\end{proof}


\begin{prop}[Inégalité de Bessel]
	Soit $(e_n)_{n\in \N}$ une famille orthonormale de $H$.
	Alors $\forall x \in H, \sum_{n=0}^{\infty} \abs{\sprod x {e_n}}^2 \leq \norm x^2$.
\end{prop}


\begin{proof}
	%TODO
\end{proof}


\begin{prop}[Identité de Parseval]
	$H$ espace de Hilbert séparable, $(e_n)_{n\in \N}$ une base hilbertienne de $H$,
	alors:
	$$ \forall x \in H, x = \sum_{n=0}^{\infty} \sprod x {e_n} e_n $$
	$$ {\norm x}^2 = \sum_{n=0}^{\infty} \abs{\sprod x {e_n}}^2 $$
	$$ \forall x,y \in H, \sprod x y = \sum_{n=0}^{\infty} \sprod x {e_n} \overline{\sprod y {e_n}} $$
\end{prop}

\begin{proof}
	%TODO
\end{proof}



\begin{prop}[Unicité des coefficients]
	Sous les mêmes hypothèses, si
	$$ \sum_{n=0}^{k} \lambda_n e_n \to_{k \to \infty} x $$
	alors $\forall n \in \N, \lambda_n = \sprod x {e_n}$.

	On dit que $(\lambda_n)$ est la suite des coefficients de $x$ dans la base (hilbertienne) $(e_n)$.
\end{prop}


\begin{proof}
	%TODO
\end{proof}



\section{Séries de Fourier}

\subsection{Définitions}


\begin{definition}
	On note $\T = \R / \Z$ et $\Li(\T, \bor (\T), \lambda_{\T})$ pour $p = 1,2$ l'espace des fonctions mesurables de $\R$ dans $\C$ 1-périodiques, munie de la mesure de Lebesgue restreinte à $[0,1[$
	telles que leur restriction à $[0,1[$ soit dans $\Lp^p([0,1[), \bor([0,1[), \lambda)$ pour $f \in \Lp_{\C}^p(\T)$.

	$$ {\norm f}_p = \left( \int_{\T} \abs{f}^p d\lambda_{\T} \right)^{\frac{1}{p}} $$
\end{definition}

\begin{definition}
	Soit $f \in \Li(\T)$, on note
	$$ c_n(f) = \int_{[0,1]} f(t) e^{-2i\pi nt} dt $$
	le coefficient de Fourier d'ordre $n$ de $f$.

	On appelle la série de Fourier de $f$ la série
	$$ \sum_{n \in \Z} c_n(f) e^{2i\pi nt} $$
	vue comme une série de fonctions abstraite.
	$$ S_N(f) = \sum_{n = -N}^{N} c_n(f) e^{2i\pi nt} $$
	les sommes partielles de la série de Fourier de $f$.
\end{definition}

\begin{prop}[Lemme de Riemann-Lebesgue]
	Si $f \in \Li(\T)$, alors $\lim_{n \to \infty} c_n(f) = 0$.
\end{prop}

\begin{proof}
	%TODO
\end{proof}

\begin{remarque}
	Comme on travaille avec des fonctions $1$-périodiques
	$$ c_n(f) = \int_0^1 f(t) e^{-2i\pi nt} dt  = \int_{\frac{1}{2}}^{\frac{1}{2}} f(t) e^{-2i\pi nt} dt $$
\end{remarque}



\begin{definition}[Coefficients de Fourier réels]
	On piet travailler avec (de preference si $f$ est a valeurs réelles)
	$$ a_0() = \int_0^1 f(t) dt $$
	$$ a_n(f) = 2\int_0^1 f(t) \cos(2\pi nt) dt $$
	$$ b_n(f) = 2\int_0^1 f(t) \sin(2\pi nt) dt $$
	$\forall n \in \N^*$, on a
	$$ a_n(f) = c_n(f) + c_{-n}(f) $$
	$$ b_n(f) = i(c_n(f) - c_{-n}(f)) $$
	et
	$$ c_n(f) = \frac{a_n(f) - ib_n(f)}{2} $$
	$$ c_{-n}(f) = \frac{a_n(f) + ib_n(f)}{2} $$
\end{definition}


\begin{prop}[Parité]
	\begin{itemize}
		\item Si $f$ est paire, alors $b_n(f) = 0$ et $c_{-n}(f) = c_n(f)$.
		\item Si $f$ est impaire, alors $a_n(f) = 0$ et $c_{-n}(f) = -c_n(f)$.
	\end{itemize}
\end{prop}



