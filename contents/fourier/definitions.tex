\subsection{Définitions}


\begin{definition}
	On note $\T = \R / \Z$ et $\Lp^p(\T, \bor (\T), \lambda_{\T})$ pour $p = 1,2$ l'espace des fonctions mesurables de $\R$ dans $\C$ 1-périodiques, munie de la mesure de Lebesgue restreinte à $[0,1[$
	telles que leur restriction à $[0,1[$ soit dans $\Lp^p([0,1[, \bor([0,1[), \lambda)$ pour $f \in \Lp_{\C}^p(\T)$.

	$$ {\norm f}_p = \left( \int_{\T} \abs{f}^p d\lambda_{\T} \right)^{\frac{1}{p}} $$
\end{definition}

\begin{definition}
	Soit $f \in \Li(\T)$, on note
	$$ c_n(f) = \int_{[0,1]} f(t) e^{-2i\pi nt} dt $$
	le coefficient de Fourier d'ordre $n$ de $f$.

	On appelle la série de Fourier de $f$ la série
	$$ \sum_{n \in \Z} c_n(f) e^{2i\pi nt} $$
	vue comme une série de fonctions abstraite.
	$$ S_N(f) = \sum_{n = -N}^{N} c_n(f) e^{2i\pi nt} $$
	les sommes partielles de la série de Fourier de $f$.
\end{definition}

\begin{prop}[Lemme de Riemann-Lebesgue]
	Si $f \in \Li(\T)$, alors $\lim_{n \to \infty} c_n(f) = 0$.
\end{prop}

\begin{proof}
	%TODO
\end{proof}

\begin{remarque}
	Comme on travaille avec des fonctions $1$-périodiques
	$$ c_n(f) = \int_0^1 f(t) e^{-2i\pi nt} dt  = \int_{\frac{1}{2}}^{\frac{-1}{2}} f(t) e^{-2i\pi nt} dt $$
\end{remarque}



\begin{definition}[Coefficients de Fourier réels]
	On peut travailler avec (de préférence si $f$ est à valeurs réelles)
	$$ a_0(f) = \int_0^1 f(t) dt $$
	$$ a_n(f) = 2\int_0^1 f(t) \cos(2\pi nt) dt $$
	$$ b_n(f) = 2\int_0^1 f(t) \sin(2\pi nt) dt $$
	$\forall n \in \N^*$, on a
	$$ a_n(f) = c_n(f) + c_{-n}(f) $$
	$$ b_n(f) = i(c_n(f) - c_{-n}(f)) $$
	et
	$$ c_n(f) = \frac{a_n(f) - ib_n(f)}{2} $$
	$$ c_{-n}(f) = \frac{a_n(f) + ib_n(f)}{2} $$
\end{definition}


\begin{prop}[Parité]
	\begin{itemize}
		\item Si $f$ est paire, alors $b_n(f) = 0$ et $c_{-n}(f) = c_n(f)$.
		\item Si $f$ est impaire, alors $a_n(f) = 0$ et $c_{-n}(f) = -c_n(f)$.
	\end{itemize}
\end{prop}
