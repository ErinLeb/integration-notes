\subsection{Fonctions mesurables}

\begin{definition}
	Soient $(E, \mathscr{A})$ et $(F, \mathscr{B})$ deux espaces mesurables. Une fonction $f: E \to F$ est dite mesurable si
	$f^{-1}(B) \in \mathscr{A}$ pour tout $B \in \mathscr{B}$.
	Dans le cas ou $E$ et $F$ sont munis de leur tribus boréliennes (si elles existent), on dit que $f$ est borélienne.
\end{definition}


\begin{prop}
	Si $f: (E, \mathscr{A}) \to (F, \mathscr{B})$ et $g: (F, \mathscr{B}) \to (G, \mathscr{C})$
	sont mesurables, alors $g \circ f : (E, \mathscr{A}) \to (G, \mathscr{C})$ est mesurable.
\end{prop}

\begin{proof}
	Soitn $C \in \mathscr{C}$. On a que $g^{-1}(C) \in \mathscr{B}$ car $g$ est mesurable et
	$f^{-1}(g^{-1}(C)) \in \mathscr{A}$ car $f$ est mesurable.
	Comme $f^{-1}(g^{-1}(C)) = (g \circ f)^{-1}(C)$, on a que $g \circ f$ est mesurable.
\end{proof}


\begin{prop}
	Pour que $f : (E, \mathscr{A}) \to (F, \mathscr{B})$ soit mesurable, il suffit qu'il existe $\mathscr{C} \subset \mathscr{B}$ telle que:
	\begin{enumerate}
		\item $f^{-1}(C) \in \mathscr{A}$ pour tout $C \in \mathscr{C}$.
		\item $\sigma(\mathscr{C}) = \mathscr{B}$
	\end{enumerate}
\end{prop}

\begin{proof}
	Posons $\mathscr{G} = \{ B \in \mathscr{B} \, | \, f^{-1}(B) \in \mathscr{A} \}$. On a que $\mathscr{C} \subset \mathscr{G}$.
	Montonrs qu $\mathscr{G}$ est une tribu sur $F$.
	\begin{enumerate}
		\item $f^{-1}(\emptyset) = \emptyset \in \mathscr{A}$ donc $\emptyset \in \mathscr{G}$.
		\item Soit $B \in \mathscr{G}$, alors $f^{-1}(B^c) = (f^{-1}(B))^c \in \mathscr{A}$ donc $B^c \in \mathscr{G}$.
		\item Soit $(B_n)_{n \in \mathbb{N}}$ une suite d'éléments de $\mathscr{G}$, alors
		      $f^{-1}(\bigcup\limits_{n \in \mathbb{N}} B_n) = \bigcup\limits_{n \in \mathbb{N}} f^{-1}(B_n) \in \mathscr{A}$ donc
		      $\bigcup\limits_{n \in \mathbb{N}} B_n \in \mathscr{G}$.
	\end{enumerate}
	Ainsi $\mathscr{G}$ est une tribu sur $F$ et $\mathscr{C} \subset \mathscr{G}$.
	On a donc
	\[ \sigma(\mathscr{B}) = \sigma(\mathscr{C}) \subset \sigma(\mathscr{G}) = \mathscr{G} \]
	Donc $\mathscr{B} \subset \mathscr{G} \subset \mathscr{B}$, donc $\mathscr{B} = \mathscr{G}$.
	On a donc que $f$ est mesurable.
\end{proof}


\begin{example}[Application]
	Si $F, \mathscr{B}) = (\mathbb{R}, \mathscr{B}(\mathbb{R}))$, on peut prendre $\mathscr{C} = \{ ]-\infty,\   t[ \, | \, t \in \mathbb{R} \}$. On a que
				$\sigma(\mathscr{C}) = \mathscr{B}(\mathbb{R})$. Et donc il suffit d'étudier la mesurabilité de $f^{-1}(]-\infty, t[)$.
\end{example}
