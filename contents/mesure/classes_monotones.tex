\subsection{Classes monotones}

\begin{definition}
	$\mathscr{M} \subset \mathscr{P}(E)$ est une classe monotone si
	\begin{itemize}
		\item $E\in \mathscr{M}$
		\item si $A,B \in \mathscr{M} et A \subset B alors B\setminus A \in \mathscr{M}$
		\item si $A_{n\in\mathbb{N}} \in \mathscr{M}$ est croissante par l'inclusion, alors $\cup A_n \in \mathscr{M}$
	\end{itemize}
\end{definition}

\begin{remarque}
	Toute tribu est une classe monotone.
\end{remarque}

\begin{definition}
	On définit la classe monotone engendrée par $\mathscr{C} \subset \mathscr{P}(E)$
	$$ \mathscr{M}(\mathscr{C}) = \bigcup\limits_{\mathscr{M} \text{classe monotone et} \mathscr{C} \subset \mathscr{M}} \mathscr{} $$
\end{definition}

\begin{proof}
	Toute intersection de classes monotones est une classe monotone: c'est la même demonstration que pour la proposition \ref{proof:intersection_tribus} sur les tribus.
\end{proof}

\begin{theorem}[Lemme de classe monotone]
	Si $\mathscr{C}$ est stable par intersections finies, alors
	$$ \mathscr{M}\mathscr{C} = \sigma(\mathscr{C}) $$
\end{theorem}


\begin{proof}
	\begin{itemize}
		\item
		      $$ \mathscr{M}\mathscr{C} \subset \sigma(\mathscr{C}) $$
		      car $\sigma(\mathscr{C})$ contient $\mathscr{C}$ et $\sigma(\mathscr{C}) $ est une
		      tribu car elle est une classe monotone.
		\item
		      $$\sigma(\mathscr{C})  \subset \mathscr{M}\mathscr{C}$$
		      On a  que $\mathscr{M}\mathscr{C}$ contient $\mathscr{C}$. Il
		      suffit de montrerque$\mathscr{M}\mathscr{C}$ est une tribu. \\

		      \begin{itemize}
			      \item $E \in \mathscr{M}\mathscr{C}$
			      \item  $\mathscr{M}\mathscr{C}$ est stable par passage au complementaire car
			            on peut prendre $B = E$ et alors $A \subset B$ et $B\setminus A = A^c$
			      \item Stabilité par union dénombrable: \\

			            Idée: $\cup A_n = \cup_{n\in mathbb{N}} B_n, avec B_n = \cup_{k<n} A_k$ et l'union est croissante.
			            Il faut montrer que $\mathscr{M}\mathscr{C}$ est stable par union finie.
			            On peut montrer que $\mathscr{M}\mathscr{C}$ stable par intersection finie. On peut juste montrer qu'elle est stable par intersection de 2 éléments.

			            Fixons $A \in \mathscr{M}\mathscr{C}$. On regarde $\mathscr{M}_A = \{B \in \mathscr{M}\mathscr{C} , A \cap B \in \mathscr{M}\mathscr{C}\}$
			            On veut montrer que $\mathscr{M}_A = \mathscr{M}\mathscr{C}$.
			            On va montrer que $\mathscr{M}_A$ est une classe monotone. On a deja $\mathscr{C} \subset \mathscr{M}_A$.
			            \begin{itemize}
				            \item $E \in \mathscr{M}_A ?$.  On a $A \cap E = A \in \mathscr{C} \subset \mathscr{M}_A$
				            \item Si $B \in \mathscr{M}_A$ , $B' \in \mathscr{M}_A$, et $B \subset B'$ montrons que $B'\setminus B \in \mathscr{M}_A$
				                  $A \cap (B'\setminus B) = \underbrace{(A \cap B')}{\in   \mathscr{M}\mathscr{C}} \underbrace{\setminus (A\cap B)}{\mathscr{M}\mathscr{C}} \in \mathscr{M}\mathscr{C}$
				                  et $A\cap B \subset A \cap B'$
				            \item Soit $B_n$ a valeur dans $\mathscr{M}_A$ une suite croissante:
				                  $$ A \cap (\cup B_n) = \cup (A \cap B_n)$$

				                  donc $A \cap (\cup B_n) \in \mathscr{M}\mathscr{C}$ et $\cup B_n \in \mathscr{M}_A $
			            \end{itemize}

			            Donc $\mathscr{M}_A$ est une classe monotone, donc $mathscr{M}_A = mathscr{M}\mathscr{C}$ qui est équivalent à :
			            $$ \forall A \in \mathscr{C}, \forall B \in \mathscr{M}\mathscr{C}, A \cap B \in \mathscr{M}\mathscr{C} (*)$$

			            Soit $B\in \mathscr{M}\mathscr{C}$.
			            $$ mathscr{M}_B = \{ A \in mathscr{M}\mathscr{C}, A \cap B \in mathscr{M}\mathscr{C}\} $$
			            On sait que $\mathscr{C}\subset\mathscr{M}_B$
			            En effet, $A \in \mathscr{C}$, on a bien $B |in mathscr{M}\mathscr{C}$, donc $A\cap B \in\mathscr{M}\mathscr{C}$ d'apres (*).
			            de plus, $\mathscr{M}_B$ est une classe monotone, meme preuve que ci-dessous (car dans cette preuve on n'utilise pas que $A \in \mathscr{C}$).
			            donc $\mathscr{M}_B = \mathscr{M}\mathscr{C}$ et donc :
			            $$ \forall B \in \mathscr{M}\mathscr{C}, \forall A \in \mathscr{M}\mathscr{C}, A \cap B \in \mathscr{M}\mathscr{C}$$
		      \end{itemize}
	\end{itemize}
\end{proof}


\begin{theorem}[Lemme d'unicité des mesures]
	Soit $\mu$ et $\nu$ deux mesures sur $(E,\mathscr{A})$. On suppose qu'il existe
	$\mathscr{C}$ stable pr intersections finies tel que $\sigma(\mathscr{C}) = \mathscr{A}$
	et $\forall C \in \mathscr{C}, \mu(C) = \nu(C)$ alors:
	\begin{itemize}
		\item Si $\mu(E) = \nu (E) < +\infty$, alors $\mu = \nu$.
		\item S'il existe une suite $E_n\in \mathscr{C}$ croissante avec
		      $E = \cup E_n$ et $\mu(E_n) = \nu(E_n)< +\infty$, alors $\mu = \nu$.
	\end{itemize}
\end{theorem}

\begin{proof}
	\begin{itemize}
		\item On regarde $\mathscr{G} = \{A \in \mathscr{A}, \mu(A) = \nu(A)\}$. Montrons que c'est une classe monotone:
		      \begin{itemize}
			      \item $\mu (E) = \nu (E), $donc $E\in \mathscr{G}$.
			      \item Soit $A,B \in \mathscr{G}$, $A\subset B $, vérifions que $B\setminus A \in \mathscr{G}$ :
			            \begin{eqnarray*}
				            \mu(A \setminus B ) &=& \mu(B)- \mu(A) \\
				            &=& \nu (B) - \nu(A) = \nu (B\setminus A)
			            \end{eqnarray*}
			      \item Si $A_n \in \mathscr{G}$ est une suite coroissante,
			            \begin{eqnarray*}
				            \mu(\cup A_n) &=& \lim\limits_{n \to \infty} \mu(A_n) \\
				            &=& \lim\limits_{n \to \infty} \nu(A_n) \\
				            &=& \nu(\cup A_n) \\
			            \end{eqnarray*}
		      \end{itemize}
		      Donc $\mathscr{G}$ est une classe monotone que contient $\mathscr{C}$.
		      Donc $\mathscr{G} \supset\mathscr{M}\mathscr{C} = \sigma(\mathscr{C}) = \mathscr{A}$ et
		      donc $\mathscr{G} = \mathscr{A}$ eto donc $\mu = \nu$.
		\item
		      On restraint $\mu$ à $E_n$, $\mu_n A \mapsto \mu(E_n \cap A)$
		      Alors $\mu_n$ est une mesure et $\mu_n < +\infty$

		      On a
		      \begin{eqnarray*}
			      \forall C \in \mathscr{C}, \mu (E_n \cap C) &=& \nu (E_n \cap C) \\
			      &=& \nu_n(C)
		      \end{eqnarray*}
		      donc $\mu_n = \nu_n$ et d'apres le (1) on a $\forall A \in \mathscr{A}, \nu(A) = \lim \mu_n(A)$
	\end{itemize}
\end{proof}


\begin{remarque}
	D'après ce théorème, la fonction de répartition caractérise la loi.
	$$ \mathscr{C} = \{ ]-\infty, x], x \in \mathbb{R} \} $$
	$$ \sigma(\mathscr{C}) = \mathscr{B}(\mathbb{R}) \text{et} \mathscr{C} \text{stable par intersections finies}$$
	si $\forall t \in \mathbb{R}, F_X(t) = F_Y(t)$ alors $\forall B \in \mathscr{B}(\mathbb{R}), \mathbb{P}_X(B) = \mathbb{P}_Y(B)$
\end{remarque}

\begin{prop}[Opérations sur les mesures]
	$(E, \mathscr{A}, \mu)$ un espace mesuré.
	\begin{itemize}
		\item Si $B \in \mathscr{A}$, on définit $\mu_B : A \mapsto \mu(A \cap B)$ qui est une mesure.
		\item $\lambda \in \mathbb{R}^+$ alors $\lambda \mu$ est une mesure.
		\item $\nu$ une msure sur $(E, \mathscr{A})$, alors $\mu + \nu$ est une mesure.
	\end{itemize}
\end{prop}


\begin{theorem}[Mesure de Lebesgue]
	Il existe une unique mesure $\lambda$ sur $(\mathbb{R}, \mathscr{B}(\mathbb{R}))$ telle que
	$$ \forall a < b, \lambda([a,b]) = b-a $$
	On l'appelle la mesure de Lebesgue sur $(\mathbb{R}, \mathscr{B}(\mathbb{R}))$.
\end{theorem}

\begin{proof}
	\begin{itemize}
		\item Existence: Admise
		\item Unicité: \\
		      $\mathscr{C} = \{ [a,b], a < b \} \cup \emptyset$ et stable par intersections finies et $\sigma (\mathscr{C}) = \mathscr{B}(\mathbb{R})$.
		      donc si $\mu([a,b]) = b-a = \nu([a,b])$ alors $\mu = \nu$, par le lemme d'unicité des mesures.
	\end{itemize}
\end{proof}
