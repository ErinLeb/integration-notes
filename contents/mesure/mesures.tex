
\subsection{Mesures}

On se donne $(E, \mathscr{A})$ un espace mesurable.

\begin{definition}[Mesure]
	On dit que $\mu : \mathscr{A} \to \mathbb{R}^+\cup \{+\infty\}$ est une mesure sur $(E, \mathscr{A})$ si:
	\begin{enumerate}
		\item $\mu(\emptyset) = 0$
		\item $\mu$ est $\sigma$-additive, c'est à dire que si $(A_n)_{n \in \mathbb{N}}$ est une suite d'éléments de $\mathscr{A}$
		      tels que $A_i \cap A_j = \emptyset$ pour $i \neq j$ (deux à deux disjoints), alors
		      \begin{equation*}
			      \mu\left(\bigcup\limits_{n \in \mathbb{N}} A_n\right) = \sum\limits_{n \in \mathbb{N}} \mu(A_n)
		      \end{equation*}
	\end{enumerate}
\end{definition}

\begin{remarque}
	On appelle meurables les ensembles qui sont dans $\mathscr{A}$.
\end{remarque}

\begin{remarque}
	Comme $\mu (A_n) \in \mathbb{R}^+ \cup \{+\infty\}$, la somme $\sum\limits_{n \in \mathbb{N}} \mu(A_n)$ est bien définie.
\end{remarque}

\begin{remarque}
	$\mu$ mesurable donne l'additivité (finie). Cependant, la reciproque est fausse.
	\begin{example}
		Soit $m: \mathscr{P}(\mathbb{N}) \to \mathbb{R}^+$
		\begin{equation*}
			m(A) = \left\{
			\begin{array}{ll}
				0       & \text{ si } A \text{ est fini } \\
				+\infty & \text{ sinon }
			\end{array}
			\right.
		\end{equation*}
		est additive mais pas $\sigma$-additive.
	\end{example}
\end{remarque}

\begin{example}
	Sur $(\mathbb{R}, \mathscr{B}(\mathbb{R}))$, on a la mesure de
	Dirac en $x_0 \in \mathbb{R}$, notée $\delta_{x_0}$, définie par:
	\begin{equation*}
		\delta_{x_0}(A) = \left\{
		\begin{array}{ll}
			1 & \text{ si } x_0 \in A \\
			0 & \text{ sinon }
		\end{array}
		\right.
	\end{equation*}
\end{example}

\begin{example}
	Sur $\mathbb{R}, \mathscr{B}(\mathbb{R})$, on a la mesure de comptable, notée $\nu$, définie par:
	\begin{equation*}
		\nu(A) = \left\{
		\begin{array}{ll}
			\#A     & \text{ si } A \text{ est fini } \\
			+\infty & \text{ sinon }
		\end{array}
		\right.
	\end{equation*}
\end{example}

\begin{proof}
	\begin{enumerate}
		\item Le espace de départ est bien une tribu car $\mathscr{P}(\mathbb{R})$ est une tribu sur $\mathbb{R}$.Le
		      espace d'arrivée est bien $\mathbb{R}^+ \cup \{+\infty\}$ car $\#A \in \mathbb{R}^+ \cup \{+\infty\}$.
		\item $\nu(\emptyset) = \#\emptyset = 0$
		\item Si $(A_n)_{n \in \mathbb{N}}$ est une suite de boréliens disjoints:
		      \begin{equation*}
			      \mu(\bigcup\limits_{n \in \mathbb{N}} A_n) = \left\{\begin{array}{ll}
				      \#A     & \text{ si } \bigcup\limits_{n \in \mathbb{N}} A_n \text{ est fini } \\
				      +\infty & \text{ sinon }
			      \end{array}
			      \right.
		      \end{equation*}
		      $\bigcup\limits_{n \in \mathbb{N}} A_n$ est fini si $\exists k ,\, A_k$ infini (cas 1) ou si les elements sont
		      finis mais tous non vides a partir d'un certain rang (cas 2).
		      \begin{enumerate}
			      \item Cas 1 : $\sum\limits_{n \in \mathbb{N}} \nu(A_n) \geq \nu(A_k) = +\infty$.
			      \item Cas 2 : $\sum\limits_{n \in \mathbb{N}} \nu(A_n) = +\infty$ car
			            $\forall n \in \mathbb{N}, \, \nu(A_n) \in \mathbb{N}$ et $\nu(A_n)$ ne stationne pas en 0l Donc il existe une suite
			            infinie d'éléments non vides, donc tels que $\nu(A_n) \geq 1$. Donc la some diverge.
		      \end{enumerate}
		      Par contre si $\bigcup\limits_{n \in \mathbb{N}} A_n$ est fini, alors $A_n = \emptyset$ à partir d'un certain rang.
		      Donc le cardinal de $\bigcup\limits_{n \in \mathbb{N}} A_n$ est fini et $\sum\limits_{n \in \mathbb{N}} \nu(A_n) = \nu(\bigcup\limits_{n \in \mathbb{N}} A_n)$ car ils
		      sont deux a deux disjoints.
	\end{enumerate}
\end{proof}

\begin{prop}[Propriétés élémentaires]\label{prop:mesure:elementaire}
	Nous avons 5 propriétés élémentaires:
	\begin{enumerate}
		\item Croissance. Si $A$ et $B$ mesurables, avec $A \subset B$, alors $\mu(B) = \mu(A) + \mu(B \setminus A)$. De plus,
		      si $\mu(B)$ est finie, alors $\mu(B\setminus A) = \mu(B) - \mu(A)$.
		\item Crible. Si $A$ et $B$ mesurables
		      \[\mu(A\cup B) + \mu(A \cap B) = \mu(A) + \mu(B)\]
		\item Continuité croissante. Soit $A_n$ une suite croissante d'ensembles mesurables $A_n \subset A_{n+1}$,
		      alors
		      \[\mu(\bigcup\limits_{n \in \mathbb{N}} A_n) = \lim\limits_{n \to \infty} \mu(A_n)\]
		\item Continuité décroissante. Soit $A_n$ une suite décroissante d'ensembles mesurables $A_{n+1}
			      \subset A_n$, telle que $\mu(A_0) < +\infty$, alors:
		      \[\mu(\bigcap\limits_{n \in \mathbb{N}} A_n) = \lim\limits_{n \to \infty} \mu(A_n)\]
		\item Sous-aditivité. Soit $(A_n)_{n \in \mathbb{N}}$ une suite d'ensembles mesurables, alors:
		      \[\mu(\bigcup\limits_{n \in \mathbb{N}} A_n) \leq \sum\limits_{n \in \mathbb{N}} \mu(A_n)\]
	\end{enumerate}
\end{prop}

\begin{proof}
	Nous allons démontrer les propriétés dans l'ordre.
	\begin{enumerate}
		\item Soit $B = A \cup (B \setminus A)$, alors $A$ et $B \setminus A$ sont disjoints. Donc
		      \[\mu(B) = \mu(A) + \mu(B \setminus A)\]
		      Donc $\mu(B) \geq \mu(A)$. Donc $\mu(B) < \infty$ alors $\mu(A)< \infty$ et $\mu(B \setminus A) = \mu(B) - \mu(A)$.
		\item \begin{eqnarray*}
			      \mu(A) + \mu (B) &=& \mu( A \setminus (A \cap B)) + \mu(A \cap B) + \mu(A \setminus B) \\
			      \mu(A\cup B)&=& \mu(A \setminus (A \cap B)) + \mu(A \cap B) + \mu(B \setminus (A \cap B))
		      \end{eqnarray*}
		\item On pose $B_0 = A_0$ et $B_n = A_n \setminus A_{n-1}$ pour $n \geq 1$. On a que les $B_n$ sont disjoints et
		      $\bigcup\limits_{n \in \mathbb{N}} B_n = \bigcup\limits_{n \in \mathbb{N}} A_n$. Donc
		      \begin{eqnarray*}
			      \bigcup\limits_{n \in \mathbb{N}} \mu(A_n) &=& \bigcup\limits_{n \in \mathbb{N}} \mu(B_n) \\
			      &=& \sum\limits_{n \in \mathbb{N}} \mu(B_n) \text{par }\quad \sigma\text{-additivité}
		      \end{eqnarray*}
		      Si l'un des $A_n$ vérifie $\mu(A_n) = +\infty$, alors $\mu(\bigcup\limits_{n \in \mathbb{N}} A_n) = +\infty$
		      et l'égalité est donc vraie.
		      \\
		      Si tous les $A_n$ sont finis, on a :
		      \begin{eqnarray*}
			      \mu(B_n) &=& \mu(A_n) - \mu(A_{n-1}) \\
			      \mu(\bigcup\limits_{n \in \mathbb{N}} A_n) &=& \sum\limits_{n \in \mathbb{N}} \left( \mu(A_n) - \mu(A_{n-1}) \right)+ \mu(A_0) \\
			      &=& \lim\limits_{n \to \infty} \mu(A_0) + \sum\limits_{n = 1}^N \left( \mu(A_n) - \mu(A_{n-1}) \right) \\
			      &=& \lim\limits_{n \to \infty} \mu(A_n)
		      \end{eqnarray*}
		\item On définit $C_n = A_0 \setminus A_n $, alors les $C_n$ sont croissants et mesurables. Donc
		      \begin{eqnarray*}
			      \mu(\bigcup\limits_{n \in \mathbb{N}} C_n) &=& \lim\limits_{n \to \infty} \mu(C_n) \\
			      &=& \lim\limits_{n \to \infty} \mu(A_0\setminus A_n) \\
			      &=& \lim\limits_{n \to \infty} \mu(A_0) - \mu(A_n) \\
			      &=& \mu(A_0) - \lim\limits_{n \to \infty} \mu(A_n)
		      \end{eqnarray*}
		      Or $\bigcup\limits_{n \in \mathbb{N}} C_n = A_0 \setminus \bigcup\limits_{n \in \mathbb{N}} A_n$ et donc
		      $\mu(\bigcup\limits_{n \in \mathbb{N}} C_n) = \mu(A_0) - \mu(\bigcup\limits_{n \in \mathbb{N}} A_n)$.
		\item On pose $D_n = A_n \cap \left( \bigcup\limits_{i=0}^{n-1} A_i \right)^c$. Les $D_n$ sont disjoints et mesurables par
		      $\sigma$-additivité.
		      \begin{eqnarray*}
			      \mu(\bigcup\limits_{n \in \mathbb{N}} A_n) &=& \mu(\bigcup\limits_{n \in \mathbb{N}} D_n) \\
			      &=& \sum\limits_{n \in \mathbb{N}} \mu(D_n) \quad \text{ par } \sigma\text{-additivité}
		      \end{eqnarray*}
		      et $D_n \subset A_n$ donc $\mu(D_n) \leq \mu(A_n) \leq \sum\limits_{n \in \mathbb{N}} \mu(A_n)$.
	\end{enumerate}
\end{proof}

\begin{remarque}
	Dans le (4)\ref{prop:mesure:elementaire}, l'hypothèse $\mu(A_0) < +\infty$ est nécessaire. En effet, soit
	$\nu$ la mesure de comptage. Soit
	\[ A_n = \{ n, n+1, n+2, \dots \} \]
	On a $\bigcup\limits_{n \in \mathbb{N}} A_n = \emptyset$ et $\nu(A_n) = +\infty$ pour tout $n \in \mathbb{N}$.
\end{remarque}

\begin{definition}[Vocabulaire des mesures]
	Soit $\mu$ une mesure sur $(E, \mathscr{A})$.
	\begin{itemize}
		\item $(E,A\mu)$ est un space mesuré si $(E,\mathscr{A})$ est un espace mesurable et $\mu$ est une mesure sur $ (E, \mathscr{A} ) $.
		\item $\mu$ est dite finie si $\mu(E) < +\infty$.
		\item $\mu$ est une probabilité si $\mu(E) = 1$.
		\item $\mu$ est $\sigma$-finie si $E = \bigcup\limits_{n \in \mathbb{N}} A_n$ avec $A_n \in \mathscr{A}$ et $\mu(A_n) < +\infty$.
		\item On dit que $x$ est un atome de $\mu$ si $\mu(\{x\}) > 0$.
		\item $\mu$ est diffuse si elle n'a pas d'atomes.
	\end{itemize}
\end{definition}
