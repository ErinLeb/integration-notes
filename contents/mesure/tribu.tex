
\subsection{Tribus}


\begin{definition}[Tribu]
	Soit $E$ un ensemble. Une tribu (ou $\sigma$-algèbre) $\mathscr{A}$ sur $E$ est une partie de $\mathscr{P}(E)$ vérifiant:
	\begin{enumerate}
		\item $E \in \mathscr{A}$
		\item $A \in \mathscr{A} \implies A^c \in \mathscr{A}$
		\item $ \forall n \in \mathbb{N}, A_n \in \mathscr{A} \implies \bigcup\limits_{n \in \mathbb{N}} A_n \in \mathscr{A}$
	\end{enumerate}
\end{definition}


\begin{prop}
	Toute intersection de tribus sur $E$ est une tribu sur $E$.
\end{prop}

\begin{proof}\label{proof:intersection_tribus}
	Si $\forall i \in I$, $\mathscr{A}_i$ est une tribu sur $E$. On a:
	\begin{itemize}
		\item $\forall  i \in I, E \in \mathscr{A}_i \implies E \in \bigcap\limits_{i \in I} \mathscr{A}_i$
		\item $\forall i \in I, A \in \mathscr{A}_i \implies \forall i \in I, \, A^c \in \mathscr{A}_i
			      \implies A^c \in \bigcap\limits_{i \in I} \mathscr{A}_i$
		\item $\forall i \in I, \forall n \in \mathbb{N}, A_n \in \mathscr{A}_i \implies
			      \forall i \in I, \,\bigcup\limits_{n \in \mathbb{N}} A_n \in \mathscr{A}_i \implies
			      \bigcup\limits_{n \in \mathbb{N}} A_n \in \bigcap\limits_{i \in I} \mathscr{A}_i$
	\end{itemize}
\end{proof}

\begin{definition}
	Soit $\mathscr{C}$ un sous ensemble de $\mathscr{P}(E)$. On note la tribu
	engrendrée par $\mathscr{C}$, $\sigma(\mathscr{C})$ avec
	\begin{equation*}
		\sigma(\mathscr{C}) = \bigcap\limits_{\mathscr{A} \text{ tribu}, \mathscr{C} \subset \mathscr{A}} \mathscr{A}
	\end{equation*}
\end{definition}

\begin{remarque}
	$\sigma(\mathscr{C})$ est bien une tribu comme intersection non vide de tribus,
	car $\mathscr{P}(E)$ est une tribu sur $E$ qui contient $\mathscr{C}$.
\end{remarque}

\begin{definition}[Tribu borélienne]
	On note $\Omega$ l'ensemble des ouverts de $\mathbb{R}$. La tribu borélienne sur
	$\mathbb{R}$ est la tribu $\sigma(\Omega)$, notée $\mathscr{B}(\mathbb{R})$.
\end{definition}


\begin{remarque}
	On peut étendre cette définition à tout espace topologique (ou moins fort, tout espace métrique). En particulier
	$\mathbb{R}^d$.
\end{remarque}

\begin{remarque}
	$\mathscr{B}(\mathbb{R})$ est aussi engendrée par :
	\begin{itemize}
		\item $\{ ]-\infty, a[\,,\, a \in \mathbb{R} \}$
		\item $\{ ]a, b[\,,\, a < b \in \mathbb{R} \}$
		\item $\{ ]a, b[\,,\, a < b \in \mathbb{Q} \}$
	\end{itemize}
\end{remarque}


\begin{definition}[Tribu produit]
	Si $(E_1, \mathscr{A}_1)$ et $(E_2, \mathscr{A}_2)$ sont deux espaces mesurables (couple ensemble-tribu compatible)
	on note $\mathscr{A}_1 \otimes \mathscr{A}_2$ la tribu sur $E_1 \times E_2$ engendrée
	par les rectangles $A_1 \times A_2$ avec $A_1 \in \mathscr{A}_1$ et $A_2 \in \mathscr{A}_2$.

\end{definition}


